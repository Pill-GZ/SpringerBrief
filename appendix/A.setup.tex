{\bf Software availability.}
U-PASS runs as an R Shiny application. It is a free, open source software under the MIT license.
A live instance of the application is hosted at 
\url{https://power.stat.lsa.umich.edu/u-pass/}.
The source code can be obtained from the repository hosting service Github, by running in the computer's terminal:
\begin{verbatim}
  clone https://github.com/Pill-GZ/U-PASS.git
\end{verbatim}
or by downloading directly from the GitHub page: 
\url{https://github.com/Pill-GZ/U-PASS}.

Should the user choose to run the application from their local machine, we recommend downloading the source code, and follow the next two steps of this user guide.

\bigskip
\noindent
{\bf Installation and dependencies.}
We have collected the required R packages inside the R script
\texttt{install\_required\_packages.R}.
These packages can be installed by navigating to the project folder, and running in the computer's terminal:
\begin{verbatim}
  Rscript install_required_packages.R 
\end{verbatim}
or by running the following command from inside R (RStudio):
\begin{verbatim}
  source("install_required_packages.R")    
\end{verbatim}
The U-PASS software itself requries no installation.

\bigskip
\noindent
{\bf Start/terminate the application.}
The application can be started by running in the computer's terminal:
\begin{verbatim}
  Rscript -e 'library(methods);
              shiny::runApp("./", launch.browser=TRUE)'
\end{verbatim}
or by running the following command from inside R (RStudio):
\begin{verbatim}
  shiny::runApp()
\end{verbatim}
The application can be terminated by simply closing the browser (or browser tab).
Alternatively, the application can be terminated by pressing \texttt{Ctrl} + \texttt{C} in the terminal, or by clicking on the red stop button in Rstudio.
