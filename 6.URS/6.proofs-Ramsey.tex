

% We start by providing a general result on the structure of arbitrary correlation matrices in this section, which helps us establish the `only if' part of Theorem \ref{thm:Gaussian-weak-dependence}. 
% Its proof uses the Ramsey Theorem from graph theory, which we briefly review next.

Given any integer $k\ge 1$, there is always an integer $R(k,k)$ called the {\em Ramsey number}:
\begin{equation}\label{eq:Ramsey-number}
k\le R(k,k)\le \binom{2k-2}{k-1}
\end{equation}
such that the following property holds:
every undirected graph with at least $R(k,k)$ vertices will contain {\em either} a clique of size $k$, or an {\em independent set} of $k$ nodes. 
Recall that a clique is a complete sub-graph where all pairs of nodes are connected, and an independent set is a set of nodes where no two nodes are connected.

This result is a consequence of the celebrated work of \citet{ramsey2009problem}, which 
gave birth to Ramsey Theory \citep[see e.g.,][]{conlon2015recent}.  
The Ramsey Theorem and the upper bound \eqref{eq:Ramsey-number} \citep[established first in][]{erdos1935combinatorial} are at the heart of the proof of the following result.  A recent improvement on the upper bound is given by \cite{sah:2020}.
%An excellent introduction to Ramsey theory is given in \url{http://math.mit.edu/~fox/MAT307-lecture05.pdf}. 

\begin{proposition} \label{prop:lower-bound-correlation-Ramsey}
  Fix $\gamma\in(0,1)$ and let $P = \left(\rho(i,j)\right)_{n\times n}$ be an arbitrary correlation
  matrix. If 
  \begin{equation}\label{eq:Ramsey-the-k-def}
   k:= \lfloor \log_2({n})/2 \rfloor  \ge \lceil 1/\gamma \rceil + 1,
  \end{equation}
  then there is a set of $k$ indices $K = \{l_1, \ldots, l_k\}\subseteq \{1,\ldots,n\}$ 
  such that 
  \begin{equation} \label{eq:lower-bound-correlation-Ramsey}
      \rho(i,j) \ge -\gamma, \mbox{ for all } i,j\in K.
  \end{equation}
\end{proposition}

\begin{proof}%[Proof of Proposition \ref{prop:lower-bound-correlation-Ramsey}]
By using \eqref{eq:Ramsey-number} and a refinement of the Stirling's formula, 
we will show at the end of the proof that for $k \le \log_2({n})/2$, we have 
\begin{equation}\label{eq:Ramsey-bounds}
 R(k,k) \le n,
\end{equation}
where $R(k,k)$ is the Ramsey number.  

Now, construct a graph with vertices $\{1,\dots,n\}$ such that there is an edge between nodes $i$ and $j$ if and only if $\rho(i,j) \ge -\gamma$. 
In view of \eqref{eq:Ramsey-bounds} and Ramsey's theorem (see e.g., Theorem 1 in \cite{fox2009lecture} or \cite{conlon2015recent} for a recent survey on Ramsey theory), there is a subset of $k$ nodes $K =\{l_1,\dots,l_k\}$, which is either a {\em complete graph} or an {\em independent set}.  Recall that in a
complete graph, every two nodes are connected with an edge; while in independent sets, no two nodes are connected.

If $K$ is a complete graph, then by our construction of the graph, Relation \eqref{eq:lower-bound-correlation-Ramsey} holds. 

Now, suppose that $K$ is a set of independent nodes.  This means, again by the construction of our graph, that
$$
\rho(i,j) < -\gamma,\quad\mbox{for all }i\not= j\in K.
$$
Let $Z_i,\ i \in K$ be zero-mean random variables such that 
$\rho(i,j) = \E [Z_iZ_j]$. Observe that
\begin{equation} \label{eq:Ramsey-proof-contradiction}
    \var\left( \sum_{i\in K} Z_i\right) 
    = \sum_{i\in K} \var(Z_i) + \sum_{\substack{i\not=j\\i,j \in K}} \cov(Z_i, Z_j) 
    <  k - k(k-1)\gamma,
\end{equation}
since $\var(Z_i)=1$ and $\rho(i,j)<-\gamma$ for $i\neq j$.
By our assumption, $k\ge \left(\lceil 1/\gamma \rceil + 1\right)$, or equivalently, $(k-1) \ge 1/\gamma$, the variance in \eqref{eq:Ramsey-proof-contradiction} is negative. 
This is a contradiction showing that there are no independent sets $K$ with cardinality $k$.

To complete the proof, it remains to show that Relation \eqref{eq:Ramsey-bounds} holds.
In view of the upper bound on the Ramsey numbers \eqref{eq:Ramsey-number}, it 
is enough to show that $k \le \log_2(\sqrt{n})$ implies
$$
\binom{2k-2}{k-1} \le n.
$$
This follows from a refinement of the Stirling formula, due to \citet{robbins1955remark}:
$$
 \sqrt{2\pi} m^{m+1/2} e^{-m} e^{\frac{1}{(12 m +1)}} \le  m! \le \sqrt{2\pi} m^{m+1/2} e^{-m} 
 e^{\frac{1}{12 m}}.
$$
Indeed, letting $\widetilde k:= k-1$, and applying the above upper and lower bounds 
to the  terms $(2\widetilde k)!$ and $\widetilde k!$, respectively, we obtain:
\begin{align*}
\binom{2k-2}{k-1} \equiv \frac{(2\widetilde k)!}{ (\widetilde k!)^2 }
\le \frac{2^{2\widetilde k}}{\sqrt{\pi \widetilde k}}\exp\left \{ \frac{1}{24 \widetilde k} -
\frac{2}{ 12 \widetilde k +1}\right\} < 2^{2 k}
\end{align*}
where the last two inequalities follow by simply dropping positive factors less than $1$.
Since $2k \le \log_2(n)$, the above bound implies Relation \eqref{eq:Ramsey-bounds} 
and the proof is complete.
\end{proof}

Using Proposition \ref{prop:lower-bound-correlation-Ramsey}, we establish the key lemma used in the proof of Theorem \ref{thm:Gaussian-weak-dependence}.


\begin{lemma} \label{lemma:positive-correlation}
  Let $c\in(0,1)$, and $P = \left(\rho(i,j)\right)_{(n+1)\times(n+1)}$ be a correlation matrix such that \begin{equation} \label{eq:positive-correlation-lemma-condition}
      \rho(1,j) > c \quad \mbox{for all } j = 1,\ldots,n+1.
  \end{equation}
  If $n \ge 2^{2\lceil2/c^2\rceil+4}$, then there is a set of indices $K = \{l_1, \ldots, l_k\}\subseteq \{2,\ldots,n+1\}$ of cardinality $k = |K| = \lfloor\log_2{\sqrt{n}}\rfloor$, such that 
  \begin{equation} \label{eq:positive-correltation-lemma-conclusion}
      \rho(i,j) > \frac{c^2}{2} \quad\mbox{for all } i,j\in K.
  \end{equation}
  That is, all entries of the $k\times k$ sub-correlation matrix $P_K:=\left(\rho(i,j)\right)_{i,j\in K}$ are larger than $c^2/2$.
\end{lemma}

\begin{proof}[Lemma \ref{lemma:positive-correlation}]
    Let $Z_1, \ldots, Z_{n+1}$ be random variables with covariance matrix $P$.
    Denote $\rho_j = \rho(1,j)$ and define 
    \begin{equation}
      R_j = 
      \begin{cases}
        \frac{1}{\sqrt{1-\rho_j^2}}\left(Z_j - \rho_j Z_1\right), &\mbox{if } \rho_j<1,\\
        R^* &\mbox{if } \rho_j=1,
      \end{cases}
    \end{equation}
    where $R^*$ is an arbitrary zero-mean, unit-variance random variable.
    It is easy to see that $\var(R_j) = 1$, and
    \begin{align*}
    \cov\left(Z_i, Z_j\right) &= \cov\left(\rho_i Z_1 + \sqrt{1-\rho_i^2} R_i, \; \rho_j Z_1 + \sqrt{1-\rho_j^2} R_j\right) \\
        &= \rho_i\rho_j + \sqrt{1-\rho_i^2}\sqrt{1-\rho_j^2} \;\cov\left(R_i, R_j\right) \\
        &> c^2 + \min\left\{\cov\left(R_i, R_j\right), 0\right\}.
    \end{align*}
    
    Therefore, Relation \eqref{eq:positive-correltation-lemma-conclusion} would hold if we can find a set of indices $K = \{l_1,\ldots,l_k\}$ such that $\cov\left(R_i,R_j\right)\ge -c^2/2$ for all $i,j\in K$, where $k=|K|=\lfloor\log_2\sqrt{n}\rfloor$.
    This, however, follows from Proposition \ref{prop:lower-bound-correlation-Ramsey} applied to $\left(R_j\right)_{j=2}^{n+1}$ with $\gamma = c^2/2$, provided that 
    $$
    k = \lfloor\log_2\sqrt{n}\rfloor \ge \lceil 2/c^2 \rceil + 1.
    $$
    The last inequality indeed follows form the assumption that $n \ge 2^{2\lceil2/c^2\rceil+4}$.
\end{proof}


