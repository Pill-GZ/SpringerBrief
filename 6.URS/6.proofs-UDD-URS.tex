
Recall that our objective is to show \eqref{eq:URS-condition}. 
We will do so in two stages; namely, we will prove that for all $\delta>0$, we have 
\begin{equation} \label{eq:URS-condition-upper-side}
    \P\left[\frac{M_{S_p}}{u_{|S_p|}} > 1+\delta\right] \to 0,
\end{equation}
and
\begin{equation} \label{eq:URS-condition-lower-side}
    \P\left[\frac{M_{S_p}}{u_{|S_p|}} < 1-\delta\right] \to 0,
\end{equation}
for any sequence of subsets $S_p$ such that $|S_p|\to\infty$.
Although the first step \eqref{eq:URS-condition-upper-side} was already shown in Proposition \ref{prop:rapid-varying-tails}, regardless of the dependence structure, we provide in this section a more refined result. 
Specifically, the following result states that for the AGG model, the constant $\delta$ in Proposition \ref{prop:rapid-varying-tails} can be replaced by a vanishing sequence $c_p\to 0$.

\begin{lemma}[Upper tails of AGG maxima] \label{lemma:AGG-maxima-upper-tails}
Let ${\cal E}$ be an array with marginal distribution $F\in\text{AGG}(\nu)$, $\nu>0$. If we pick
\begin{equation} \label{eq:choice-of-c_p}
    c_p = \frac{u_{p\log{p}}}{u_p} - 1,    
\end{equation} 
where $u_p = F^{\leftarrow}(1-1/p)$, then we have $c_p>0$, $c_p\to 0$, and
\begin{equation} \label{eq:AGG-max-upper-bound}
    \P\left[\frac{M_p}{u_p}-(1+c_p) > 0\right] \to 0.
\end{equation}
\end{lemma}
The proof can be found in Section \ref{subsec:bounding-upper-tails-of-maxima} below.

Since Lemma \ref{lemma:AGG-maxima-upper-tails} holds regardless of the dependence structure, the same conclusions hold if one replaces $M_p$ by $M_{S_p} = \max_{j\in S_p}\epsilon(j)$ and $p$ by $q = q(p)=|S_p|$, where $S_p$ is any sequence of sets such that $q \equiv |S_p| \to \infty$.
This entails \eqref{eq:URS-condition-upper-side}.

On the other hand, the proof of \eqref{eq:URS-condition-lower-side} uses a more elaborate argument based on the Sudakov-Fernique bound.
We proceed by first bounding the probability by an expectation. 
For all $\delta>0$, we have
\begin{align}
    \P\left[\frac{M_{S_p}}{u_q}<1-\delta\right] 
        &= \P\left[-\left(\frac{M_{S_p}}{u_q} - (1+c_q)\right) > \delta + c_q\right] \nonumber \\
        %&\le \P\left[\max{\left\{-\left(\frac{M_{S_p}}{u_q} - (1+c_q)\right),0\right\}} > \delta + c_q\right] \nonumber \\
        &\le \P\left[\left(\frac{M_{S_p}}{u_q} - (1+c_q)\right)_->\delta+c_q\right] \nonumber \\
        &\le \frac{1}{\delta + c_q}\E\left[\left(\frac{M_{S_p}}{u_q} - (1+c_q)\right)_-\right], \label{eq:Gaussian-maxima-lower-expectation-bound}
\end{align}
where $(x)_-:=\max\{-x,0\}$ and the last line follows from the Markov inequality.
The next result shows that the upper bound in \eqref{eq:Gaussian-maxima-lower-expectation-bound} vanishes.
\begin{lemma} \label{lemma:Gaussian-maxima-lower-expectation}
  Let ${\cal E}$ be a Gaussian UDD  array  and 
  $S_p\subseteq\{1,\ldots,p\}$ be an arbitrary sequence of sets 
  such that $q = q(p) = |S_p|\to\infty$.  Then, for $M_{S_p}:= \max_{j\in S_p} \epsilon_p(j)$ and $c_q$ as in \eqref{eq:choice-of-c_p}, we have
  \begin{equation} \label{eq:Gaussian-maxima-lower-expectation}
    \E\left[\left(\frac{M_{S_p}}{u_q} - (1+c_q)\right)_-\right] \to 0,\ \ \quad \mbox{ as }p\to \infty.
  \end{equation}
\end{lemma}
The proof of the lemma is given in Section \ref{subsec:bounding-lower-tails-of-maxima} below.


Going back to the proof of Theorem \ref{thm:Gaussian-weak-dependence}, we observe that Relations \eqref{eq:Gaussian-maxima-lower-expectation-bound} and \eqref{eq:Gaussian-maxima-lower-expectation} imply \eqref{eq:URS-condition-lower-side}, which completes the proof of the `if' part of
Theorem \ref{thm:Gaussian-weak-dependence}. \qed 

\begin{remark} Only the Sudakov-Fernique minorization argument used in the proof of Lemma \ref{lemma:Gaussian-maxima-lower-expectation}, relies on the Gaussian assumption. We expect the techniques and results here to be useful in extending Theorem \ref{thm:Gaussian-weak-dependence} to more general class of distributions, say, the AGG model.
\end{remark}


\subsection{Bounding the upper tails of AGG maxima}
\label{subsec:bounding-upper-tails-of-maxima}


\begin{proof}[Lemma \ref{lemma:AGG-maxima-upper-tails}]
Recall by \eqref{eq:AGG-quantiles} that 
\begin{equation*}
    u_q\sim\left(\nu\log{q}\right)^{1/\nu}, \quad q\to\infty,
\end{equation*}
so that
\begin{equation} %\label{eq:choice-of-c_p}
c_p  = \frac{u_{p\log{p}}}{u_p} -1 = \left(\frac{\log{p}+\log{\log{p}}}{\log{p}}\right)^{1/\nu}(1+o(1)) - 1 \rightarrow 0 \quad \mbox{as } p\to\infty.
\end{equation} 
By the union bound, we have
\begin{align}
    \P\left[\frac{M_p}{u_p} > 1+c_p\right] 
        &\le \sum_{j=1}^p \P\left[\frac{\epsilon_p(j)}{u_p} > 1+c_p\right] 
        = p \overline{F}\left(u_{p\log{p}}\right) \label{eq:AGG-maxima-upper-tails-proof-1} \\
        &= p \overline{F}\left(F^{\leftarrow}\left(1-\frac{1}{p\log{p}}\right)\right) \le \frac{1}{\log{p}} \rightarrow 0. \nonumber
\end{align}
where the last inequality follows from the fact that $F\left(F^{\leftarrow}(u)\right)\ge u$ for all $u\in[0,1]$.
\end{proof}

In addition to Lemma \ref{lemma:AGG-maxima-upper-tails}, which says the upper tail vanishes in probability, we will also prepare a result which states that the upper tail also vanishes in expectation.

\begin{lemma} \label{lemma:AGG-max-uniform-integrability}
Let $M_p$ and $c_p$ be as in Lemma \ref{lemma:AGG-maxima-upper-tails}, and denote 
$$
\xi_p := \frac{M_p}{(1+c_p)u_p}.
$$
Then there exist $p_0, t_0 > 0$, and an absolute constant $C>0$ such that
\begin{equation} \label{eq:lemma-AGG-uniform-integrability}
    \P\left[\xi_p > t \right]\le \exp{\{-Ct^\nu\}}, \quad \text{for all} \quad p>p_0,\ t>t_0.
\end{equation}
In particular, the set of random variables $\{\left(\xi_p\right)_+,p\in\N\}$ is uniformly integrable.
\end{lemma}
\begin{proof}[Lemma \ref{lemma:AGG-max-uniform-integrability}]
Recalling that $(1+c_p)u_p = u_{p\log{p}}$, and by applying the union bound as in \eqref{eq:AGG-maxima-upper-tails-proof-1}, we have
\begin{align}
    \log \P\left[\xi_p > t\right] 
        &\le \log p + \log{\overline{F}\left(u_{p\log{p}}t\right)} \nonumber \\
        &\le \log p - \frac{1}{\nu}\left(u_{p\log{p}}t\right)^\nu(1-\delta). \label{eq:lemma-AGG-uniform-integrability-proof-1}
\end{align}
for $t > t_0(\delta)>0$, where $\delta\in(0,1)$ is an arbitrarily small number fixed in advance. 
This follows from the assumption that $F\in\text{AGG}(\nu)$ and Definition \ref{def:AGG} of the AGG distribution.
Using in \eqref{eq:lemma-AGG-uniform-integrability-proof-1} the explicit expressions for the quantiles in \eqref{eq:AGG-quantiles}, we obtain
\begin{equation} \label{eq:lemma-AGG-uniform-integrability-proof-2}
    \log \P\left[\xi_p > t\right] \le \log p - \underbrace{\left(1+o(1)\right)(1-\delta)t^\nu}_{\text{greater than }1\text{ for large }t}\log{p} - t^\nu\underbrace{\log{\log{p}}\left(1+o(1)\right)(1-\delta)}_{\text{greater than }C\text{ for large }p}.
\end{equation}
For large $t$, we have $\left(1+o(1)\right)(1-\delta)t^\nu > 1$, so that the sum of the first two terms on the right-hand side of \eqref{eq:lemma-AGG-uniform-integrability-proof-2} is negative.
Also, for $p$ larger than some constant $p_0(\delta)$, we have $\log{\log{p}}\left(1+o(1)\right)(1-\delta) > C$ for some constant $C$ that does not depend on $p$.
Therefore \eqref{eq:lemma-AGG-uniform-integrability} holds for $t>t_0(\delta)$ and $p>p_0(\delta)$, and the proof is complete.
\end{proof}

\begin{corollary} \label{cor:AGG-max-upper-bound-expectation}
The upper tails of AGG maxima vanish in expectation, i.e.,
    \begin{equation} \label{eq:AGG-max-bound-upper-expectation}
    \E\left[\left(\frac{M_p}{u_p} - (1+c_p)\right)_+\right]
    \to 0 \quad\text{as }\; p\to\infty,
\end{equation}
where $(a)_+ := \max\{a,0\}$.
\end{corollary}

\begin{proof}[Corollary \ref{cor:AGG-max-upper-bound-expectation}]
Since $c_p\ge0$ is a sequence converging to 0, we have $c_p < 1$ for $p \ge p_0$. Hence for any $t>0$, we have
\begin{align}
    \P\left[\left(\frac{M_p}{u_p} - (1+c_p)\right)_+ > t\right] 
    &= \P\left[(1+c_p)\left(\xi_p-1\right)_+ > t\right] \nonumber \\
    &\le \P\left[\left(\xi_p-1\right)_+ > t/2\right] 
    \le \P\left[\xi_p > t/2 \right]. \label{eq:AGG-max-bound-upper-expectation-proof}
\end{align}
By Lemma \ref{lemma:AGG-max-uniform-integrability}, $\{\left(\xi_p\right)_+\}$ is u.i., therefore by Relation \eqref{eq:AGG-max-bound-upper-expectation-proof}, $\{\left(M_p/u_p - (1+c_p)\right)_+,\;p\in\N\}$ is u.i. as well.
Since by Lemma \ref{lemma:AGG-maxima-upper-tails}, $\left(M_p/u_p - (1+c_p)\right)_+\to 0$ in probability, Relation \eqref{eq:AGG-max-bound-upper-expectation} follows from the established uniform integrability \citep[see, e.g., Theorem 6.6.1 in][]{resnick:1999book}.
\end{proof}


\subsection{Bounding the lower tails of Gaussian maxima}
\label{subsec:bounding-lower-tails-of-maxima}

The main goal of this section is to establish the following result. 

\begin{proposition} \label{prop:Gaussian-maxima-expectation-lower-bound}
For every UDD Gaussian array $\cal E$, and any sequence of subsets
$S_p\subseteq\{1,\ldots,p\}$ such that $q = q(p) = |S_p|\to \infty$, we have
\begin{equation} \label{eq:AGG-max-bound-expectation}
    \liminf_{p\to\infty} \E\left[\frac{M_{S_p}}{u_q}\right] \ge 1,
\end{equation}
where $M_S = \max_{j\in S}\epsilon(j)$.
\end{proposition}
% We suppressed dependence on $p$ for both $S = S_p$ and $s = s(p) = |S_p|$ for convenience of notation.

We will first show that Lemma \ref{lemma:Gaussian-maxima-lower-expectation}, which is the key to the proof of the `if' part of 
Theorem \ref{thm:Gaussian-weak-dependence}, follows immediately from this proposition.

\begin{proof}[Lemma \ref{lemma:Gaussian-maxima-lower-expectation}]
We start with the identity
$$
\E\left[\frac{M_{S_p}}{u_q}-(1+c_q)\right] = \E\left[\left(\frac{M_{S_p}}{u_q}-(1+c_q)\right)_+\right] - \E\left[\left(\frac{M_{S_p}}{u_q}-(1+c_q)\right)_-\right].
$$
By re-arranging terms and taking limsup/liminf, we obtain
\begin{align} \label{e:lim-sup-vanishes}
    0 \le &\limsup_{p\to\infty}\E\left[\left(\frac{M_{S_p}}{u_q}-(1+c_q)\right)_-\right]\nonumber \\
        \le & \limsup_{p\to\infty}\E\left[\left(\frac{M_{S_p}}{u_q}-(1+c_q)\right)_+\right] - \liminf_{p\to\infty}\E\left[\frac{M_{S_p}}{u_q}-(1+c_q)\right]\\
        = & - \liminf_{p\to\infty}\E\left[\frac{M_{S_p}}{u_q}-(1+c_q)\right],
        \label{e:lim-inf-bound}
\end{align}
where the last equality follows from the fact that the lim-sup in \eqref{e:lim-sup-vanishes} vanishes by Corollary \ref{cor:AGG-max-upper-bound-expectation}.
On the other hand, since $c_q\to 0$, we have
$$
\liminf_{p\to\infty}\E\left[\frac{M_{S_p}}{u_q}-(1+c_q)\right] 
= \liminf_{p\to\infty}\E\left[\frac{M_{S_p}}{u_q}-1\right] \ge 0,
$$
where the last inequality follows from Proposition \ref{prop:Gaussian-maxima-expectation-lower-bound}.  This shows that 
the right-hand side of \eqref{e:lim-inf-bound} is non-positive and hence
\eqref{eq:Gaussian-maxima-lower-expectation} holds. 
\end{proof}

The following interesting fact about the relationship between the upper quantiles and the expectation of iid maxima will be needed for the 
proof of  Proposition \ref{prop:Gaussian-maxima-expectation-lower-bound}.
%The following lemma may be of independent interest.
% The following result applies to AGG models as well as a wider range of models, provided that tail densities are monotone discreasing.

\begin{lemma} \label{lemma:expectation-lower}
Let $(X_i)_{i=1}^p$ be $p$ iid random variables with distribution $F$ such that $\E[(X_i)_-]$ exists, i.e.,
$$
\E[\max\{-X_i, 0\}] < \infty.
$$
Let $M_p = \max_{i=1,\ldots,p}X_i$. Assume that $F$ has a density $f$, which is eventually decreasing. 
More precisely, we suppose there exists a 
$C_0$ such that $0<F(C_0)<1$, and $f(x_1) \ge f(x_2)$ whenever $C_0 < x_1 \le x_2$. 
Under these assumptions, we have,
$$
\liminf_{p\to\infty}\frac{\E M_p}{u_{p+1}} \ge 1,
$$
where $u_{p+1} = F^{\leftarrow}(1 - 1/(p+1))$.
\end{lemma}

\begin{proof}%[Proof of Lemma \ref{lemma:expectation-lower}]
% For ease of exposition we let $C_0 = 0$; the case where $C_0 > 0$ can be handled using exactly the same arguments.
The idea comes from an argument in the monograph of \cite{boucheron:lugosi:massart:2013}.  Write 
$$
X_i = F^{\leftarrow}(U_i)
$$
where $U_i$ are iid uniform random variables on $(0,1)$.
Denote $M^{U}_p$ as the maximum of the $U_i$'s, we have $\E M_p = \E\left[F^{\leftarrow}(M^{U}_p)\right]$, and by conditioning, we obtain
\begin{align} \label{eq:lemma:expectation-lower-proof-1}
    \E M_p &= \E\left[F^{\leftarrow}(M^{U}_p)\;\big|\;M^{U}_p \ge F(C_0)\right] \P\left[M^{U}_p \ge F(C_0)\right] + \nonumber \\ 
           &\quad\quad +\E\left[F^{\leftarrow}(M^{U}_p)\;\big|\;M^{U}_p < F(C_0)\right] \P\left[M^{U}_p < F(C_0)\right]. 
\end{align} 
Focus on the first term in the summation. Since $f$ is decreasing beyond $C_0$, $F$ is concave on $(C_0, \infty)$, and $F^{\leftarrow}$ is convex on $(F(C_0), 1)$. By Jensen's inequality, we have
\begin{equation*}
    \E\left[F^{\leftarrow}(M^{U}_p)\;\big|\;M^{U}_p \ge F(C_0)\right] 
        \ge F^{\leftarrow}\left(\E[M^{U}_p\;|\;M^{U}_p\ge F(C_0)]\right).
\end{equation*}
Using the fact that $M_p^U$ is the maximum of iid Uniform$(0,1)$ random variables, with a direct calculation one can show that 
\begin{equation*}
    F^{\leftarrow}\left(\E[M^{U}_p\;|\;M^{U}_p\ge F(C_0)]\right)
    = F^{\leftarrow}\left(\left(1-\frac{1}{p+1}\right)\left(\frac{1-F(C_0)^{p+1}}{1-F(C_0)^{p}}\right)\right),
\end{equation*}
and hence
\begin{align} \label{e:lemma:expectation-lower-bound-1.5}
    \E\left[F^{\leftarrow}(M^{U}_p)\;\big|\;M^{U}_p \ge F(C_0)\right] 
        &\ge F^{\leftarrow}\left(\left(1-\frac{1}{p+1}\right)\left(\frac{1-F(C_0)^{p+1}}{1-F(C_0)^{p}}\right)\right) \nonumber\\
        &\ge F^{\leftarrow}\left(1-\frac{1}{p+1}\right) = u_{p+1}.
\end{align}
Now, focus on the second term in \eqref{eq:lemma:expectation-lower-proof-1}. Since 
$\P[M^{U}_p\le m\big|\;M^{U}_p < F(C_0)] = \left(m/F(C_0)\right)^p \le m/F(C_0)$ for $m\le F(C_0)$, we have
$$\left(M^{U}_p\;\big|\;M^{U}_p < F(C_0)\right)\stackrel{\text{d}}{\ge} \left(U_1\;\big|\;U_1 < F(C_0)\right),$$
where and the latter is the uniform distribution on $(0,F(C_0))$.
Therefore, for the second term of the sum in \eqref{eq:lemma:expectation-lower-proof-1}, by the 
monotonicity of $F^{\leftarrow}$, we obtain
\begin{align} \label{e:lemma:expectation-lower-bound-1.75}
    \E\left[F^{\leftarrow}(M^{U}_p)\;\big|\;M^{U}_p < F(C_0)\right] 
        &\ge \E\left[F^{\leftarrow}(U_1)\;\big|\;U_1 < F(C_0)\right] \nonumber \\
        &= \E\left[X_1\;\big|\;X_1 < C_0\right].
\end{align}
Finally, since $\P\left[M^{U}_p < F(C_0)\right] = F(C_0)^p = 1 - \P\left[M^{U}_p \ge F(C_0)\right]$, by \eqref{e:lemma:expectation-lower-bound-1.5} and \eqref{e:lemma:expectation-lower-bound-1.75}, we have % \eqref{eq:AGG-quantiles}, 
\begin{align*}
    \frac{\E M_p}{u_{p+1}} 
        &\ge \left(1 - F(C_0)^p\right) + \frac{\E\left[X_1\;\big|\;X_1 < C_0\right]}{u_{p+1}} F(C_0)^p.
\end{align*}
The conclusion follows since the right-hand-side of the last inequality converges to 1.
\end{proof}


We are now ready to prove Proposition \ref{prop:Gaussian-maxima-expectation-lower-bound}. 
This is where the UDD dependence assumption is used.

\begin{proof}[Proposition \ref{prop:Gaussian-maxima-expectation-lower-bound}]
% Assume without loss of generality that the $\epsilon(j)$'s have unit variances.
Recall that ${\cal E} = \{\epsilon_p(i),\ i=1,\cdots,p,\ p\in\N\}$ is a Gaussian array with standardized marginals.
Define the canonical (pseudo) metric on $S_p \subset\{1,\cdots,p\}$,
$$
d(i,j) = \sqrt{\E\left[\left(\epsilon(i)-\epsilon(j)\right)^2\right]}.
$$
It can be easily checked that the canonical metric takes values between 0 and 2.
For an arbitrary $\delta\in(0,1)$, take $\gamma = \sqrt{2(1-\delta)}$, and
let $\mathcal{N}$ be a $\gamma$-packing of $S_p$. That is, let $\mathcal{N}$ be a subset of $S_p$, such that for any $i,j\in\mathcal{N}$, $i\neq j$, we have $d(i,j)\ge\gamma$, i.e.,
\begin{equation}\label{e:gamma-packing-def}
d(i,j) = \sqrt{2\left(1-\Sigma_p(i,j)\right)} \ge \gamma = \sqrt{2(1-\delta)},
\end{equation}
or equivalently, $\Sigma_p(i,j) \le \delta$.
We claim that we can find a $\gamma$-packing $\mathcal{N}$ whose number of elements is at least 
\begin{equation} \label{eq:packing-number-lower-bound}
    |\mathcal{N}| \ge q/N(\delta).
\end{equation}
Indeed, $\mathcal{N}$ can be constructed iteratively as follows:
\begin{itemize}    \raggedright
    \item[{\bf 1:}] Set $S_p^{(1)}:=S_p$ and $\mathcal{N}:=\{j_1\}$, where $j_1\in S_p^{(1)}$ is an arbitrary element. Set $k:=1$.\\
    \item[{\bf 2:}] Set $S_p^{(k+1)}:=S_p^{(k)}\setminus B_\gamma(j_k)$, where
    $$
    B_\gamma(j_k) := \{i\in S_p: \;d(i,j_k) < \gamma \equiv \sqrt{2(1-\delta)}\}.
    $$
    \item[{\bf 3:}] If $S_p^{(k)} \neq \emptyset$, pick an arbitrary $j_{k+1}\in S_p^{(k)}$, set $\mathcal{N}:=\mathcal{N}\cup\{j_{k+1}\}$, and $k:=k+1$, go to step 2; otherwise, stop.
\end{itemize}
By the definition of UDD (see Definition \ref{def:UDD}), there are at most $N(\delta)$ coordinates whose covariance with $\epsilon(j)$ exceed $\delta$. 
Therefore at each iteration, $\left|B_\gamma(j_k)\right|\le N(\delta)$, and hence
$$
\left|S_p^{(k+1)}\right| \ge \left|S_p^{(k)}\right| - \left| B_\gamma(j_k)\right| \ge q - kN(\delta).
$$
The construction can continue for at least $q/N(\delta)$ iterations, and we have $|\mathcal{N}| \ge \lfloor q/N(\delta) \rfloor$ as desired.
    
Now we define on this $\gamma$-packing $\mathcal{N}$ an independent Gaussian process $\left(\eta(j)\right)_{j\in\mathcal{N}}$, 
$$\eta(j) = \frac{\gamma}{\sqrt{2}}Z(j) \quad j\in \mathcal{N},$$
where $Z(j)$'s are iid standard normal random variables.
Observe that by the definition of $\gamma$-packing in \eqref{e:gamma-packing-def}, the increments of the new process are smaller than those of the original process in the following sense, 
$$
\E\left[\left(\eta(i)-\eta(j)\right)^2\right] = \gamma^2 \le d^2(i,j) = \E\left[\left(\epsilon(i)-\epsilon(j)\right)^2\right]
$$
for all $i\neq j$, $i,j\in\mathcal{N}$. Applying the Sudakov-Fernique inequality (see Theorem \ref{th:Sudakov-Fernique})
to  $\left(\eta(j)\right)_{j\in\mathcal{N}}$ and  $\left(\epsilon(j)\right)_{j\in\mathcal{N}}$, we have
\begin{equation}\label{eq:Sudakov-1}
\E\left[\max_{j\in\mathcal{N}}\eta(j)\right] \le \E\left[\max_{j\in\mathcal{N}}\epsilon(j)\right] \le \E\left[\max_{j\in{S_p}}\epsilon(j)\right].
\end{equation}
Since the $\left(\eta(j)\right)_{j\in\mathcal{N}}$ are independent Gaussians, Lemma \ref{lemma:expectation-lower} yields the lower bound,
\begin{equation}\label{eq:Sudakov-2}
\liminf_{p\to\infty} \E\left[\frac{\max_{j\in\mathcal{N}}\eta(j)}{u_{|\mathcal{N}|}}\right] \ge \frac{\gamma}{\sqrt{2}} = \sqrt{1-\delta}.
\end{equation}
Using \eqref{eq:packing-number-lower-bound} and the expressions \eqref{eq:AGG-quantiles} for the quantiles 
of AGG models (with $\nu=2$ here), we have
\begin{equation}\label{eq:Sudakov-3}
\frac{u_{|\mathcal{N}|}}{u_q} 
\ge \left(\frac{\log q-\log{N(\delta)}}{\log{q}}\right)^{1/2}\left(1+o(1)\right)\to 1,
\end{equation}
since $N(\delta)$ does not depend on $q= q(p)\to \infty$.

By combining \eqref{eq:Sudakov-1}, \eqref{eq:Sudakov-2} and \eqref{eq:Sudakov-3}, we conclude that
\begin{align*}
    \liminf_{p\to\infty} \E\left[\frac{\max_{j\in{S_p}}\epsilon(j)}{u_q}\right] 
    &\ge \liminf_{p\to\infty} \E\left[\frac{\max_{j\in\mathcal{N}}\eta(j)}{u_q}\right] &\text{by } \eqref{eq:Sudakov-1}\\
    &\ge \liminf_{p\to\infty} \E\left[\frac{\max_{j\in\mathcal{N}}\eta(j)}{u_{|\mathcal{N}|}}\right]  &\text{by } \eqref{eq:Sudakov-3} \\
    &\ge \sqrt{1-\delta}.  &\text{by } \eqref{eq:Sudakov-2}
\end{align*} 
Since $\delta>0$ is arbitrary, \eqref{eq:AGG-max-bound-expectation} follows as desired.
\end{proof}

% The current proof of Theorem \ref{thm:Gaussian-weak-dependence} relies on Slepian's lemma \cite{slepian1962one}. 
% \begin{lemma}[Slepian's Lemma] \label{lemma:Slepian}
% For two zero-mean Gaussian sequences $\left(\eta(j)\right)_{j=1}^p$ and $\left(z(j)\right)_{j=1}^p$, if
% $$
% \E[\eta(j)^2] \le \E[z(j)^2] \quad \text{and} \quad \E[\eta(i)\eta(j)] \le \E[z(i)z(j)]
% $$
% for all $i,j\in \{1,\ldots,p\}$, then 
% $$
% \P\left[\max_{j\in\{1,\ldots,p\}}\eta(j) \le u\right] \ge \P\left[\max_{j\in\{1,\ldots,p\}}z(j) \le u\right]
% $$
% for all $u\in\mathbb R$.
% \end{lemma}

