For a study with a fixed budget, i.e., a fixed total number of subjects $n$, the researcher is free to choose the fraction of cases $\phi_1$ to be included in the study.
A natural question is how this budget should be allocated to maximize the statistical power of discovery, or equivalently, the signal sizes $\lambda=nw^2$.

In principal, Relation \eqref{eq:signal-size-odds-ratio} can be optimized with respect to the fraction of cases $\phi_1$ in order to find optimal designs, if $\theta_1$ is known and held constant.
In practice, this is not the case.
While the fraction of cases can be controlled, the distributions of genotypes \emph{in the study} are often unknown prior to data collection, and can change with the case-to-control ratio.

Fortunately, the conditional distributions of genotypes in the healthy control groups are often estimated by existing studies, and are made available by consortia such as the NHGRI-EBI GWAS catalog \citep{macarthur2016new}.
% Assume (after appropriate relabelling, hence without loss of generality) that the first variant is associated with an increased risk of disease, and is henceforth referred to as the risk variant.
We denote the conditional frequency of the first genetic variant in the control group as $(f, 1-f)$, where
\begin{equation} \label{eq:RAF}
    f := \mu_{21} / \phi_2 = \mu_{21}/ (1-\phi_1).
\end{equation}
The multinomial probability is fully parametrized by the new trio: $(f, \phi_1, R)$.
\begin{center}
    \begin{tabular}{cccc}
    \hline
    & \multicolumn{2}{c}{Genotype} \\
    \cline{2-3}
    Probabilities & Variant 1 & Variant 2 & Total by phenotype \\
    \hline
    Cases & $\frac{\phi_1fR}{fR+1-f}$ & $\frac{\phi_1(1-f)}{fR+1-f}$ & $\phi_1$ \\
    Controls & $f(1-\phi_1)$ & $(1-f)(1-\phi_1)$ & $1-\phi_1$ \\
    \hline
    \end{tabular}
\end{center}
Proposition \ref{prop:signal-size-odds-ratio} may also be re-stated in terms of the new parametrization.

% Note that all these quantities refer to what is in the study, and differ from their counterparts in the general population.

\begin{corollary} \label{cor:signal-size-odds-ratio-conditional-frequency}
In the 2-by-2 multinomial distribution with marginals $(\phi_1, \phi_2 = 1-\phi_1)$, and conditional distribution of the variants in the control group $(f, 1-f)$,
Relation \eqref{eq:signal-size-odds-ratio} holds with $\theta_1 = {\phi_1fR}/{(fR+1-f)} + f(1-\phi_1)$ and $\theta_2 = 1-\theta_1$.
\end{corollary} 

The choice of $\phi_1$ now has a practical solution.

\begin{corollary} \label{cor:optimal-design}
In the context of Corollary \ref{cor:signal-size-odds-ratio-conditional-frequency},
the optimal design $(\phi^*_1, \phi^*_2)$ that maximizes the signal size per sample $w^2$ is prescribed by
\begin{equation} \label{eq:optimal-design}
    \phi_1^* = \frac{fR+1-f}{fR+1-f+\sqrt{R}}, \quad\text{and}\quad 
    \phi_2^* = 1-\phi_1^*.
\end{equation}
% when the denominator in \eqref{eq:optimal-design} is non-zero; otherwise, $\phi_1^*=\phi_2^*=1/2$.
\end{corollary} 


\begin{proof}
	Using the parametrization in \eqref{eq:reparametrize-2-by-2-table-1}, we solve for $\delta$ in \eqref{eq:odds-ratio-delta} to obtain
	\begin{align}
		\delta &= \frac{\phi_1 fR}{fR+1-f} - \left(\frac{\phi_1 fR}{fR+1-f} + f(1-\phi_1)\right)\phi_1 \nonumber \\
		&= \frac{f(1-f)\phi_1(1-\phi_1)(R-1)}{fR+1-f}. \label{eq:reparametrize-2-by-2-table-2}
	\end{align}
	Substituting \eqref{eq:reparametrize-2-by-2-table-2} into the expression \eqref{eq:signal-size-chisq-delta}, after some simplification, yields
	\begin{equation} \label{eq:reparametrize-2-by-2-table-3}
	w^2 = \frac{f(1-f)\phi_1(1-\phi_1)(R-1)^2}{\left[\phi_1 R + (1-\phi_1)D\right]\left[\phi_1 + (1-\phi_1)D\right]},
	\end{equation}
	where $D = fR+1-f > 0$.
	Therefore, the derivative of \eqref{eq:reparametrize-2-by-2-table-3} with respect to $\phi_1$ is
	\begin{equation} \label{eq:signal-size-first-derivative}
	\frac{\mathrm{d}w^2}{\mathrm{d}\phi_1} = 
	\frac{f(1-f)(R-1)^2}{\left[\phi_1 R+(1-\phi_1)D\right]^2 \left[\phi_1+(1-\phi_1)D\right]^2} \left[(D^2-R)\phi_1^2 - 2D^2\phi_1 + D^2\right].
	\end{equation}
	Further, we obtain the second derivative with respect to $\phi_1$,
	\begin{equation} \label{eq:signal-size-second-derivative}
	\frac{\mathrm{d}^2w^2}{\mathrm{d}\phi_1^2} = 
	h(R,f) \left[(\phi_1-1)D^2 - \phi_1R\right],
	\end{equation}
	where $h$ is some function of $(R,f)$ taking on strictly positive values.
	
	Since $\left[(\phi_1-1)D^2 - \phi_1R\right]<0$, the second derivative \eqref{eq:signal-size-second-derivative} must be strictly negative on $[0,1]$.
	This implies that the first derivative \eqref{eq:signal-size-first-derivative} is strictly decreasing on $[0,1]$. 
	Since the first derivative \eqref{eq:signal-size-first-derivative} is strictly positive at $\phi_1=0$, and strictly negative at $\phi_1=1$, it must have a unique zero between 0 and 1, and hence, the solution to $(D^2-R)\phi_1^2 - 2D^2\phi_1 + D^2 = 0$ in the interval of $[0,1]$ must be the maximizer of \eqref{eq:reparametrize-2-by-2-table-3} --- when $D^2-R>0$, the smaller of the two roots maximizes \eqref{eq:reparametrize-2-by-2-table-3}, and when $D^2-R<0$, it is the larger of the two.
	They share the same expression ${D}/{(D+\sqrt{R})}$, which coincides with \eqref{eq:optimal-design}.
	Finally, when $D^2=R$, the only root $\phi_1^*=1/2$, which also coincides with \eqref{eq:optimal-design}, is the maximizer 
	of \eqref{eq:reparametrize-2-by-2-table-3}.  
	%\stilian{This is so, in particular, when 
	%$R=1$ and confirms our intuition that the symmetric design is optimal when the odds ratio 
	%equals one.  This is not the only case when symmetrical designs are optimal though. }{\fbox{ Am I CORRECT?}}
\end{proof}

Of particular interest in the genetics literature are genetic variants with very low allele frequencies in the control group (i.e., $f\approx 0$), known as rare variants.
In such cases, Equation \eqref{eq:optimal-design} can be approximated using the Taylor expansion,
\begin{equation} \label{eq:optimal-design-approx}
    \phi_1^* = \frac{1}{1 + \sqrt{R}} + \frac{(R-\sqrt{R})f}{1+\sqrt{R}} + O(f^2).
\end{equation}
To illustrate, for rare and adversarial factors ($f\approx0$ and $R>1$), the optimal $\phi_1^*$ is less than $1/2$.
Therefore, for studies under a fixed budget, controls should constitute the majority of the subjects, in order to maximize power.
On the other hand, for rare and protective factors ($f\approx0$ and $R<1$), the optimal $\phi_1^*$ is greater than $1/2$, and cases should be the majority.
