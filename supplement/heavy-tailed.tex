We analyze the performance of thresholding estimators under heavy-tailed models in this section, and illustrate its lack of phase transition.
Suppose we have iid errors with Pareto tails in Model \eqref{eq:model-additive}, that is, $\epsilon(i)$'s have common marginal distribution $F$ where
\begin{equation} \label{eq:pareto-tails}
    \overline{F}(x) \sim x^{-\alpha} \quad \text{and} \quad F(-x) \sim x^{-\alpha},    
\end{equation}
as $x\to\infty$. 
It is well-known (see, e.g., Theorem 1.6.2 of \citep{leadbetter2012extremes}) that the maxima of iid Pareto random variables have Frechet-type limits.
Specifically, we have
\begin{equation} \label{eq:Frechet-limit-1}
    \frac{\max_{i\in\{1,\ldots,p\}}\epsilon(i)}{u_p} \implies Y,
\end{equation}
in distribution, where $u_p = F^{\leftarrow}(1-1/p)\sim p^{1/\alpha}$, and $Y$ is a standard $\alpha$-Frechet random variable, i.e.,
\begin{equation*}
    \P[Y\le t] = \exp{\{-t^{-\alpha}\}}, \quad t>0.
\end{equation*}
By symmetry in our assumptions, the same argument applies to the minima as well.

\begin{theorem} \label{thm:heavy-tails}
Let errors in Model \eqref{eq:model-additive} be as described in Relation \eqref{eq:pareto-tails}.
Let the signal have $s = |S| = fp$ non-zero entries, with magnitude $\Delta = rp^{1/\alpha}$, where both $f\in(0,1)$ and $r\in(0,+\infty)$ may depend on $p$, so that no generality is lost.

Under these assumptions, the necessary condition for thresholding procedures $\widehat{S}$ to achieve exact support recovery ($\P[\widehat{S}=S]\to1$) is 
\begin{equation} \label{eq:heavy-tails-exact-recovery}
    \liminf_{p\to\infty} r = \infty.
\end{equation}
Condition \eqref{eq:heavy-tails-exact-recovery} is also sufficient for the oracle thresholding procedure to succeed in the exact support recovery problem.

On the other hand, the necessary and sufficient condition for all thresholding procedures to fail exact support recovery ($\P[\widehat{S}=S]\to0$) is 
$$
\limsup_{p\to\infty} r = 0.
$$
\end{theorem}

In other words, Theorem \ref{thm:heavy-tails} states that there does not exist a non-trivial phase transition for thresholding procedures when errors have (two-sided) $\alpha$-Pareto tails.

\begin{proof}[Theorem \ref{thm:heavy-tails}]
Recall the oracle thresholding procedure $\widehat{S}^* = \left\{i:x(i) \ge x_{[s]}\right\}$, and the set of all thresholding procedures, denoted 
${\cal S}$ (see Definition \ref{eq:thresholding-procedure}).
The probability of exact support recovery by any thresholding procedure $\widehat{S}\in{\cal S}$ is bounded above by that of $\widehat{S}^*$, that is,
\begin{align}
    \max_{\widehat{S}\in{\cal S}}\P[\widehat{S}=S] 
        &= \P[\widehat{S}^*=S] 
        = \P\Big[\max_{i\in S^c}x(i) \le \min_{i\in S}x(i)\Big] \nonumber \\
        &= \P\Big[\frac{\max_{i\in S^c}x(i)}{u_p} \le \frac{\min_{i\in S}x(i)}{u_p}\Big] \nonumber \\
        &= \P\Big[\frac{M_{S^c}}{u_p} \le \frac{m_S}{u_p} + r_p\Big], \label{eq:heavy-tailed-case-proof-0}
        % &= \P\Big[\underbrace{(1-f)^{1/\alpha}Y^{(1)}}_{=:F_1} + \underbrace{f^{1/\alpha}Y^{(2)}}_{=:F_2} \le r\Big], 
\end{align}
where $M_{S^c} = \max_{i\in S^c}\epsilon(i)$ and $m_S = \min_{i\in S}\epsilon(i)$.
For any $\alpha > 0$, the following elementary relations hold,
\begin{equation*}
    0 < L \le (1-f)^{1/\alpha} + f^{1/\alpha} \le U < \infty, \quad \text{for all} \; f\in(0,1),
\end{equation*}
where $L = \min\left\{1, 2(1/2)^{1/\alpha}\right\}$ and $U = \max\left\{1, 2(1/2)^{1/\alpha}\right\}$.
Therefore we have,
\begin{equation} \label{eq:heavy-tailed-case-proof-1}
    U\max\left\{\frac{M_{S^c}}{u_p}, -\frac{m_S}{u_p}\right\} < r_p
    \implies
    (1-f)^{1/\alpha}\frac{M_{S^c}}{u_p} - f^{1/\alpha}\frac{m_S}{u_p} < r_p,
\end{equation}
and 
\begin{equation} \label{eq:heavy-tailed-case-proof-2}
    L\min\left\{\frac{M_{S^c}}{u_p}, -\frac{m_S}{u_p}\right\} < r_p
    \impliedby
    (1-f)^{1/\alpha}\frac{M_{S^c}}{u_p} - f^{1/\alpha}\frac{m_S}{u_p} < r_p.
\end{equation}
Putting together \eqref{eq:heavy-tailed-case-proof-0}, \eqref{eq:heavy-tailed-case-proof-1}, and \eqref{eq:heavy-tailed-case-proof-2}, we have
\begin{equation} \label{eq:heavy-tailed-case-proof-3}
    \P\Big[\max\left\{\frac{M_{S^c}}{u_p}, -\frac{m_S}{u_p}\right\} < r_p/U\Big]
    \le \P[\widehat{S}^*=S]
    \le \P\Big[\min\left\{\frac{M_{S^c}}{u_p}, -\frac{m_S}{u_p}\right\} < r_p/L\Big].
\end{equation}
We know from the weak convergence result \eqref{eq:Frechet-limit-1} that for any $\epsilon>0$ there is a constant $N$ such that for all $p>N$ we have 
\begin{equation} \label{eq:heavy-tailed-case-proof-3.5}
    \P\Big[\max\left\{\frac{M_{S^c}}{u_p}, -\frac{m_S}{u_p}\right\} < r_p/U\Big]
    \ge \P\Big[\max\left\{Y^{(1)}, Y^{(2)}\right\} < {r_p}/{U}\Big] - \epsilon,
\end{equation}
where $Y^{(1)}$ and $Y^{(2)}$ are independent $\alpha$-Frechet random variables with scale coefficients $(1-f)^{1/\alpha}$ and $f^{1/\alpha}$ respectively.
That is,
\begin{equation*}
    \P[Y^{(1)}\le t] = \exp{\{-(1-f)/t^\alpha\}}, 
    \quad \text{and} \quad
    \P[Y^{(2)}\le t] = \exp{\{-f/t^\alpha\}}.
\end{equation*}
Since the distributional limit in \eqref{eq:heavy-tailed-case-proof-3.5} has a density (with respect to the Lebesgue measure), we know that density is bounded above by a finite constant, say, $K$.
For the same choice of $\epsilon$ as before, we can find a further constant $N'$ such that for all $p>\max\{N, N'\}$ we have 
$$
    \liminf r_p < \epsilon/K + r_p,
$$
so that the right hand side of \eqref{eq:heavy-tailed-case-proof-3.5} is bounded by
\begin{equation} \label{eq:heavy-tailed-case-proof-3.6}
    \P\Big[\max\left\{Y^{(1)}, Y^{(2)}\right\} < {r_p}/{U}\Big] - \epsilon
    \ge \P\Big[\max\left\{Y^{(1)}, Y^{(2)}\right\} < \frac{\liminf r_p}{U}\Big] - 2\epsilon.
\end{equation}
By the arbitrariness in the choice of $\epsilon$, we conclude from \eqref{eq:heavy-tailed-case-proof-3.5} and \eqref{eq:heavy-tailed-case-proof-3.6} that
\begin{equation} \label{eq:heavy-tailed-case-proof-4}
    \liminf \P\Big[\max\left\{\frac{M_{S^c}}{u_p}, -\frac{m_S}{u_p}\right\} < r_p/U\Big]
    \ge \P\Big[\max\left\{Y^{(1)}, Y^{(2)}\right\} < \frac{\liminf r_p}{U}\Big].
\end{equation}
Combining Relations \eqref{eq:heavy-tailed-case-proof-3} and \eqref{eq:heavy-tailed-case-proof-4}, we know that if $\liminf r_p = \infty$, we must have 
$$
    \liminf\P\left[\widehat{S}^*=S\right]
    \ge \P\Big[\max\left\{Y^{(1)}, Y^{(2)}\right\} < \frac{\liminf r_p}{U}\Big] = 1.
$$
Conversely, if $\liminf\P\left[\widehat{S}^*=S\right]<1$, we must have $\liminf r_p < \infty$.

Similarly, we can obtain the upper bound of exact support recovery probability for the optimal thresholding procedure,
\begin{equation} \label{eq:heavy-tailed-case-proof-5}
    \limsup \P\Big[\min\left\{\frac{M_{S^c}}{u_p}, -\frac{m_S}{u_p}\right\} < r_p/L\Big]
    \le \P\Big[\min\left\{Y^{(1)}, Y^{(2)}\right\} < \frac{\limsup r_p}{L}\Big].
\end{equation}
The conclusions of the second part of Theorem \ref{thm:heavy-tails} follow from \eqref{eq:heavy-tailed-case-proof-3} and \eqref{eq:heavy-tailed-case-proof-5}.
\end{proof}

The probability of exact recovery can be approximated if the parameters $r$ and $f$ converge.
The next result follows from a small modification of the arguments in the proof of Theorem \ref{thm:heavy-tails}.

\begin{corollary}
Under the assumptions in Theorem \ref{thm:heavy-tails}, if 
$\lim r = r^*$, and $\lim f = f^*$, for some constant $r^*\ge0$ and $f^*\in[0,1]$, then 
$$
    \lim \P[\widehat{S}^*=S] 
    = \P\Big[(1-f^*)^{1/\alpha}Z_1 + (f^*)^{1/\alpha}Z_2 < {r^*}\Big].
$$
where $Z_1$ and $Z_2$ are independent standard $\alpha$-Frechet random variables, i.e., $\P[Z_i \le x] = \exp{\{-x^{-\alpha}\}}$, $x>0$.
\end{corollary}

\begin{remark}
Of course one might wonder if it would be meaningful to derive a ``phase transition'' under a different parametrization of the signal sizes, say 
\begin{equation} \label{eq:Pareto-parametrization-with-boundary}
    \Delta = p^{r/\alpha}.
\end{equation}
In this case, Theorem \ref{thm:heavy-tails} suggests that a ``phase transition'' takes place at $r=1$.
However, this non-multiplicative parametrization of the signal sizes would make power analysis (like in Example \ref{exmp:gap-when-signal-sparse}) dimension-dependent. 

To illustrate, in the case of Gaussian errors with variance 1, if we were interested in small signals of size $\sqrt{2r\log{p}}$, where $r<1$ is below the boundary \eqref{eq:strong-classification-boundary}, then we only need $n > 2/r$ samples to guarantee discovery of their support.
In the Pareto case with parametrization \eqref{eq:Pareto-parametrization-with-boundary}, however, if we were interested in small signals of size $p^{r/\alpha}$, where $r<1$, then the ``boundary'' says that we will need $n > p^{2(1-r)/\alpha}$ samples, which is exponential in the dimension $p$ and quickly diverges.
Recall that the ``boundary'' is really an asymptotic result in $p$. 
Such an approximation in finite dimensions becomes invalid.
\end{remark}

