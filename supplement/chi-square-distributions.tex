We shall recall, and establish, some auxiliary facts about chi-square distributions. 
These facts will be used in the proofs of Theorem \ref{thm:chi-squared-exact-boundary} and Theorem \ref{thm:chi-squared-approx-boundary}.

\begin{lemma}[Rapid variation of chi-square distribution tails] \label{lemma:rapid-variation-chisq}
The central chi-square distribution with $\nu$ degrees of freedom has rapidly varying tails.
That is, 
\begin{equation} \label{eq:rapid-variation-chisq}
    \lim_{x\to\infty}\frac{\P[\chi_\nu^2(0)>tx]}{\P[\chi_\nu^2(0)>x]} = 
    \begin{cases}
    0, & t > 1 \\
    1, & t = 1 \\
    \infty, & 0 < t < 1
\end{cases},
\end{equation}
where we overloaded the notation $\chi_\nu^2(0)$ to represent a random variable with the chi-square distribution.
\end{lemma}

\begin{proof}[Lemma \ref{lemma:rapid-variation-chisq}]
When $\nu=1$, the chi-square distribution reduces to a squared Normal, and \eqref{eq:rapid-variation-chisq} follows from the rapid variation of the standard Normal distribution.
For $\nu\ge2$, we recall the following bound on tail probabilities (see, e.g., \citep{inglot2010inequalities}),
$$
\frac{1}{2}\mathcal{E}_\nu(x) \le \P[\chi_\nu^2(0)>x] \le \frac{x}{(x-\nu+2)\sqrt{\pi}} \mathcal{E}_\nu(x), \quad \nu\ge2,\;x>\nu-2,
$$
where $\mathcal{E}_\nu(x) = \exp\left\{-\frac{1}{2}[(x-\nu-(\nu-2)\log(x/\nu) + \log\nu]\right\}$.
Therefore, we have 
$$
\frac{(x-\nu+2)\sqrt{\pi}}{2x}\frac{\mathcal{E}_\nu(tx)}{\mathcal{E}_\nu(x)} 
\le \frac{\P[\chi_\nu^2(0)>tx]}{\P[\chi_\nu^2(0)>x]}
\le \frac{2tx}{(tx-\nu+2)\sqrt{\pi}}\frac{\mathcal{E}_\nu(tx)}{\mathcal{E}_\nu(x)},
$$
where ${\mathcal{E}_\nu(tx)}/{\mathcal{E}_\nu(x)} = \exp\{-\frac{1}{2}[(t-1)x-(\nu-2)\log{t}]\}$ converges to $0$ or $\infty$ depending on whether $t>1$ or $0<t<1$.
The case where $t=1$ is trivial.
\end{proof}

Lemma \ref{lemma:rapid-variation-chisq} and Proposition \ref{prop:rapid-varying-tails} yield the following Corollary.

\begin{corollary} \label{cor:relative-stability}
Maxima of independent observations from central chi-square distributions with $\nu$ degrees of freedom are relatively stable. 
Specifically, let $\epsilon_p = \left(\epsilon_p(i)\right)_{i=1}^p$ be independently and identically distributed (iid) $\chi_\nu^2(0)$ random variables. 
Then the triangular array ${\cal E} = \{\epsilon_p, p\in\N\}$ has relatively stable (RS) maxima in the sense of \eqref{eq:RS-condition}.
\end{corollary}


\begin{lemma}[Stochastic monotonicity] \label{lemma:stochastic-monotonicity}
The non-central chi-square distribution is stochastically monotone in its non-centrality parameter.
Specifically, for two non-central chi-square distributions both with $\nu$ degrees of freedom, and non-centrality parameters $\lambda_1 \le \lambda_2$, we have $\chi^2_\nu(\lambda_1) \stackrel{\mathrm{d}}{\le} \chi^2_\nu(\lambda_2)$. 
That is,
\begin{equation} \label{eq:stochastic-monotonicity}
    \P[\chi^2_\nu(\lambda_1) \le t] \ge \P[\chi^2_\nu(\lambda_2) \le t], \quad \text{for any}\quad t\ge0.
\end{equation}
where we overloaded the notation $\chi_\nu^2(\lambda)$ to represent a random variable with the chi-square distribution with non-centrality parameter $\lambda$ and degree-of-freedom parameter $\nu$.
\end{lemma}

\begin{proof}[Lemma \ref{lemma:stochastic-monotonicity}]
Recall that non-central chi-square distributions can be written as sums of $\nu-1$ standard normal random variables and a non-central normal random variable with mean $\sqrt{\lambda}$ and variance 1,
\begin{equation*}
    \chi_\nu^2(\lambda) 
    \stackrel{\mathrm{d}}{=} Z_1^2 + \ldots + Z_{\nu-1}^2 + (Z_\nu + \sqrt{\lambda})^2.
\end{equation*}
Therefore, it suffices to show that $\P[(Z+\sqrt{\lambda})^2 \le t]$ is non-increasing in $\lambda$ for any $t\ge0$, where $Z$ is a standard normal random variable.
We rewrite this expression in terms of standard normal probability function $\Phi$,
\begin{align}
    \P[(Z+\sqrt{\lambda})^2 \le t] 
    &= \P[-\sqrt{\lambda} - \sqrt{t} \le Z \le -\sqrt{\lambda} + \sqrt{t}] \nonumber \\
    &= \Phi(-\sqrt{\lambda} + \sqrt{t}) - \Phi(-\sqrt{\lambda} - \sqrt{t}). \label{eq:stochastic-monotonicity-proof-1}
\end{align}
The derivative of the last expression (with respect to $\lambda$) is 
\begin{equation} \label{eq:stochastic-monotonicity-proof-2}
    \frac{1}{2\sqrt{\lambda}} \left(\phi(\sqrt{\lambda} + \sqrt{t}) - \phi(\sqrt{\lambda} - \sqrt{t})\right) 
    = \frac{1}{2\sqrt{\lambda}} \left(\phi(\sqrt{\lambda} + \sqrt{t}) - \phi(\sqrt{t} - \sqrt{\lambda})\right),
\end{equation}
where $\phi$ is the density of the standard normal distribution.
Notice that we have used the symmetry of $\phi$ around 0 in the last expression.

Since $0 \le \max\{\sqrt{\lambda} - \sqrt{t}, \sqrt{t} - \sqrt{\lambda}\} < \sqrt{t} + \sqrt{\lambda}$ when $t>0$, by monotonicity of the normal density on $(0,\infty)$, we conclude that the derivative \eqref{eq:stochastic-monotonicity-proof-2} is indeed negative.
Therefore, \eqref{eq:stochastic-monotonicity-proof-1} is decreasing in $\lambda$, and \eqref{eq:stochastic-monotonicity} follows for $t>0$.
For $t = 0$, equality holds in \eqref{eq:stochastic-monotonicity} with both probabilities being 0.
\end{proof}


Finally, we derive asymptotic expressions for  chi-square quantiles.

\begin{lemma}[Chi-square quantiles] \label{lemma:chisq-quantiles}
Let $F$ be the central chi-square distributions with $\nu$ degrees of freedom, and let $u(y)$ be the $(1-y)$-th generalized quantile of $F$, i.e.,
\begin{equation} \label{eq:quantiles-generic}
    u(y) = F^\leftarrow(1 - y).
\end{equation}
Then 
\begin{equation}
    u(y) \sim 2\log(1/y), \quad \text{as }y\to0. 
\end{equation}
\end{lemma}

\begin{proof}[Lemma \ref{lemma:chisq-quantiles}]
The case where $\nu=1$ follows from the well-known formula for Normal quantiles (see, e.g., Proposition 1.1 in \cite{gao2018fundamental})
$$
F^\leftarrow(1 - y) = \Phi^\leftarrow(1-y/2)\sim\sqrt{2\log{(2/y)}}\sim\sqrt{2\log{(1/y)}}.
$$
The case where $\nu\ge2$ follows from the following estimates of high quantiles of chi-square distributions (see, e.g., \citep{inglot2010inequalities}),
$$
    \nu +  2\log(1/y) -5/2 \le u(y) \le \nu +  2\log(1/y) + 2\sqrt{\nu\log(1/y)}, \quad \text{for all }y\le0.17,
$$
where both the lower and upper bound are asymptotic to $2\log(1/y)$.
\end{proof}

