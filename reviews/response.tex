\documentclass[11pt]{article}
\usepackage{fullpage,amsfonts,amsmath}

\title{Response to the reviews of the manuscript: ``Concentration of Maxima and Fundamental Limits in High-Dimensional Testing and Inference'' by Zheng Gao
and Stilian Stoev submitted to {\em SpringerBriefs Series in Probability and Statistics} }

\begin{document}
 \maketitle
 
 \section*{Summary of the revision}
 
 We sincerely thank the managing Editor and the five anonymous referees for their very helpful and constructive suggestions.  We really 
 appreciate the reviewers' time, effort, and thoughtful suggestions, which stimulated us to substantially revise and improve the manuscript.
 We begin with a high-level description of our edits and in the following section we respond to specific points raised by the reviewers.
 
 \begin{itemize}
  \item {\bf (Brief description)} The revised manuscript has now a brief introductory chapter and 6 substantive chapters, 
  which amount to a total of 124 pages. As with the original submission,  Chapter 2 reviews the statistical risks, estimation and testing procedures,
  as well as probability models for the errors used in the rest of the manuscript.  It contains also a high-level summary of our results 
  in the context of the existing literature and some auxiliary facts from probability and extreme value theory.  For convenience of the reader, we 
  have added statements of Slepyan's Lemma and the Sudakov-Fernique inequality (without proofs).
  
   Chapter 3 provides a panorama of the phase-transition results in the signal detection and support estimation problems for the additive
   iid Gaussian error model.  Its purpose is to emphasize ideas and provide an accessible account, including references to the state-of-the-art 
   in the literature.  The chapter contains also some new results on the (sub)optimality of some popular statistics, which appear to have been 
   overlooked in the literature (Theorem 3.1).  One main idea  behind the exposition is to systematically present the different types of phase transition 
   results as a function of the chosen statistical risk.  
   
   Chapters 4 through 7 contain the core of our original contributions.  
   
   Chapter 4 establishes phase-transitions results for the exact support recovery problem in the additive error model.  The main themes of this 
   chapter are {\em error--dependence} and {\em concentration of maxima}. We establish that the notion of uniform relative stability -- 
   a type of concentration of maxima phenomenon -- is key to characterizing the phase-transition in exact support recovery for the general class 
   of thresholding estimators. 
   
   Chapter 5 begins with a study of the finite-sample (Bayes) optimality and sub-optimality of the threshold-based support estimators.  
   Criteria for the latter are established and certain theoretical results indicate that {\em likelihood thresholding} (as opposed to data thresholding)
   is in general optimal, when the thresholding estimators are sub-optimal.  In the regime when the thresholding estimators are optimal, i.e., 
   for log-concave error densities, our results entail 
   a universal statistical limit in the exact support recovery problem, valid for all estimators (Theorem 5.3).   On the other hand, for the general class of
   thresholding estimators, we obtain minimax characterizations of the phase-transition in exact support recovery, for every fixed error-dependence 
   structure that satisfies mild uniform relative stability condition (Corollary 5.3).
   
   Chapter 6 is nearly unchanged. It is more probabilistic in nature and it provides a complete characterization of the uniform relatively stable (URS) Gaussian 
   triangular arrays.  This URS condition is the key property, in the lower bound on exact support recovery under dependence in Chpter 4. 
   These probabilistic results as well as their methods of proof may of independent interest and are accessible to a wider audience of graduate students in statistics. 
   
   Chapter 7 focuses on an application to genome-wide association studies (GWAS) in statistical genetics.  It is shown that all phase-transition results in the 
    additive error models of Chapter 3, have their close counterparts for the chi-square models arising naturally in GWAS. To connect these
    asymptotic results to the practical notions in statistical genetics, we establish the connection between the odds-ratio (effect size) in multinomial 
    $2\times 2$ models
    and the signal-size parameter of the corresponding chi-square association test.  This allows us to quantify the statistical power and
     optimal design questions as well as ultimately explain the role of phase-transitions in the fundamental statistical limits of GWAS.
    
  
 
  \item {\bf (Organization)} Following a suggestion of a reviewer ({\bf R2}),  we have included most proofs 
  in the text right after the corresponding claims (theorems, propositions, lemmas).  In the interest of space and to conform with the format of
  the SpringerBriefs series, we no longer include ``Exercise'' sections. 
  
  Last but not least, to better balance the exposition, we decided to divide the previous
  Chapter 4 into two new chapters (now Ch.\ 4, entitled ``Exact Support Recovery Under Dependence'', 
  and Ch.\ 5 ``Bayes and Minimax Optimality'').  In addition to making all chapters more manageable and of about equal length, it exposes more
  clearly our contributions.  The results contained in Chapters 4 and 5 as well as the more probabilistic Chapter 6 and more applied Chapter 7 are 
  original and new contributions to the literature.
  
 \end{itemize}
 
 
 \section*{Responses to specific points raised by the reviewers}
\end{document}