
We first recall some basic properties of the Gaussian distribution in Section \ref{sec:Gaussian-distributions}.
Section \ref{suppsec:BH-monotonicity} states and proves an interesting property of the \ac{BH} procedure which may be of independent interest.
Results on the signal detection problem (Theorem \ref{thm:detection-optimality}) are proved in Section \ref{subsec:proof-additive-error-detection-boundaries}, and the phase transition results on the support recovery problems (Theorems \ref{thm:Gaussian-error-exact-boundary} through \ref{thm:Gaussian-error-approx-exact-boundary}) are shown in Sections \ref{subsec:proof-additive-error-approx-boundaries} and \ref{subsec:proof-additive-error-mix-boundaries}.


\section{Auxiliary facts of Gaussian distributions}
\label{sec:Gaussian-distributions}

We recall three facts of Gaussian distributions that will be used in the proofs later.

We first state the relative stability of iid standard Gaussian random variables,
Since the standard Gaussian distribution falls in the class of asymptotically generalized Gaussians (AGG; see Definition \ref{def:AGG}), by Example \ref{exmp:AGG}, we know that the triangular array ${\cal E} = \{\left(\epsilon_p(i)\right)_{i=1}^p, p\in\N\}$ has relatively stable (RS) maxima in the sense of \eqref{eq:RS-condition}, i.e.,
\begin{equation} \label{eq:relative-stability-Gaussian-maxima}
    \frac{1}{u_{p}} \max_{i=1,\ldots,p} \epsilon_p(i) \xrightarrow{\P} 1,\quad \text{as }\;p\to\infty,
\end{equation}
where $u_p$ is the $(1/p)$-th upper quantile as defined in \eqref{eq:AGG-quantiles}.
Similarly, since the array ${\cal E}$ has distributions symmetric around 0, it also has relatively stable minima
\begin{equation} \label{eq:relative-stability-Gaussian-minima}
    \frac{1}{u_{p}} \min_{i=1,\ldots,p} \epsilon_p(i) \xrightarrow{\P} -1,\quad \text{as }\;p\to\infty.
\end{equation}

The second fact is on the well-known bounds for the Mill's ratio of Gaussian tails.
Let $\Phi$ denote the CDF of the standard Gaussian distribution and $\phi$ its density.
One can show that for all $x>0$ we have
\begin{equation} \label{eq:Mills-ratio}
    \frac{x}{1+x^2}\phi(x) \le \overline{\Phi}(x) = 1-\Phi(x) \le \frac{1}{x}\phi(x),
\end{equation}
using e.g., integration by parts.

The third fact is the stochastic monotonicity of the Gaussian location family. 
In fact, for all location families $\{F_\delta(x)\}_\delta$ where $F_\delta(x) = F(x-\delta)$, we have,
\begin{equation} \label{eq:stochastic-monotonicity-Gaussian}
    F_{\delta_1}(t) \ge F_{\delta_2}(t), \quad \text{for all}\quad t\in\mathbb{R}\quad\text{and all}\quad \delta_1 \le \delta_2.
\end{equation}
Relation \eqref{eq:stochastic-monotonicity-Gaussian} holds, of course, when $F$ is the standard Gaussian distribution. 


\section{Monotonicity of the Benjamini-Hochberg procedure}
\label{suppsec:BH-monotonicity}

As promised in the previous section, we make a connection between power of the \ac{BH} procedure and the stochastic ordering of distributions under the alternative.
This natural result seems new. 

\begin{lemma}[Monotonicity of the BH procedure] \label{lemma:monotonicity-BH-procedure}
Consider $p$ independent observations $x(i)$, $i\in\{1,\ldots,p\}$, where the $(p-s)$ coordinates in the null part have common distribution $F_0$, and the remaining $s$ signals have alternative distributions $F^{i}_j$, $i\in S$, respectively.
Compare the two alternatives $j\in\{1,2\}$, where the distributions in Alternative 2 are stochastically larger than those in Alternative 1, i.e.,
\begin{equation*}
    F^{i}_2(t) \le F^{i}_1(t), \quad \text{for all} \;\; t\in\R, \; \text{and for all} \;\; i\in S.
\end{equation*}
If the BH procedure is applied at the same nominal level of FDR, then the FNR of the \ac{BH} procedure under Alternative 2 is bounded above by the FNR under Alternative 1.
Further, the threshold of the \ac{BH} procedure under Alternative 2 is stochastically smaller than that under Alternative 1.
\end{lemma}

Loosely put, the power of the BH procedure is monotone increasing with respect to the stochastic ordering of the alternatives, yet (the distribution of) the \ac{BH} threshold is monotone decreasing in the distributions of the alternatives.

\begin{proof}[Lemma \ref{lemma:monotonicity-BH-procedure}]
We first re-express the BH procedure in a different form.
Recall that on observing $x(i)$, $i\in\{1,\ldots,p\}$, the BH procedure is the thresholding procedure with threshold set at $x_{[i^*]}$, where $i^* := \max\{i\,|\,\overline{F_0}(x_{[i]})\le \alpha i/p\}$, and $x_{[1]}\ge\ldots\ge x_{[p]}$ are the order statistics.

Let $\widehat{G}$ denote the left-continuous empirical survival function
\begin{equation} \label{eq:empirical-tail-distribution}
    \widehat{G}(t) = \frac{1}{p}\sum_{i=1}^p\mathbbm{1}\{x(i) \ge t\}.
\end{equation}
By the definition, we know that $\widehat{G}(x_{[i]}) = i/p$.
Therefore, by the definition of $i^*$, we have
\begin{equation*} 
    \overline{F_0}(x_{[i]}) > \alpha\widehat{G}(x_{[i]}) = \alpha i/p \quad \text{for all }i>i^*.
\end{equation*}
Since $\widehat{G}$ is constant on $(x_{[i^*+1]}, x_{[i^*]}]$, the fact that 
$\overline{F_0}(x_{[i^*]}) \le \alpha\widehat{G}(x_{[i^*]})$ and $\overline{F_0}(x_{[i^*+1]}) > \alpha\widehat{G}(x_{[i^*+1]})$ implies that $\alpha\widehat{G}$ and $\overline{F_0}$ must ``intersect'' on the interval by continuity of $F_0$.
We denote this ``intersection'' as
\begin{equation} \label{eq:approx-boundary-proof-tau}
    \tau = \inf\{t\,|\,\overline{F_0}(t)\le\alpha\widehat{G}(t)\}. 
    %= \min\{t\,|\,\overline{F_0}(t)=\alpha\widehat{G}(t)\}.
\end{equation}
Note that $\tau$ cannot be equal to $x_{[i^*+1]}$ since $\overline{F}_0$ is c\`adl\`ag.
Since there is no observation in $[\tau, x_{[i^*]})$, we can write the BH procedure as the thresholding procedure with threshold set at $\tau$.

Now, denote the observations under Alternatives 1 and 2 as $x_1(i)$ and $x_2(i)$.
Since $x_2(i)$ stochastically dominates $x_1(i)$ for all $i\in\{1,\ldots,p\}$, there exists a coupling $(\widetilde{x}_1, \widetilde{x}_2)$ of $x_1$ and $x_2$ such that 
% $\widetilde{x}_1(i) = \widetilde{x}_2(i)$ for $i\in S^c$, and 
$\widetilde{x}_1(i) \le \widetilde{x}_2(i)$ almost surely for all $i$.
We will replace $\widetilde{x}_1$ and $\widetilde{x}_2$ with $x_1$ and $x_2$ in what follows.
Since we will compare the FNR's, i.e., expectations with respect to the marginals of ${x}$'s in the last step, this replacement does not affect the conclusions.
To simplify notation, we still write $x_1$ and $x_2$ in place of $\widetilde{x}_1$ and $\widetilde{x}_2$.

Let $\widehat{G}_k$ be the left-continuous empirical survival function under Alternative $k$, i.e.,
\begin{equation} \label{eq:empirical-survival}
    \widehat{G}_k(t) = \frac{1}{p}\sum_{i=1}^p\mathbbm{1}\{x_k(i) \ge t\}, \quad k\in\{1,2\}.
\end{equation}
We define the BH thresholds $\tau_1$ and $\tau_2$ by replacing $\widehat{G}$ in \eqref{eq:approx-boundary-proof-tau} with $\widehat{G}_1$ and $\widehat{G}_2$, respectively.
Denote the set estimates of signal support $\widehat{S}_k = \{i\,|\,x_k(i)\ge\tau_k\}$ by the BH procedure.
We claim that 
\begin{equation} \label{eq:monotonicity-BH-procedure-thresholds}
    \tau_2 \le \tau_1 \quad \text{with probability } 1.
\end{equation}

Indeed, by definition of the empirical survival function \eqref{eq:empirical-survival} and the fact that $x_1(i) \le x_2(i)$ almost surely for all $i$,  we have $\widehat{G}_1(t) \le \widehat{G}_2(t)$ for all $t$.
Hence, $\overline{F_0}(t)\le\alpha\widehat{G}_1(t)$ implies $\overline{F_0}(t)\le\alpha\widehat{G}_2(t)$, and Relation \eqref{eq:monotonicity-BH-procedure-thresholds} follows from the definition of $\tau$ in \eqref{eq:approx-boundary-proof-tau}.
The claim of stochastic ordering of the \ac{BH} thresholds in Lemma \ref{lemma:monotonicity-BH-procedure} follows from \eqref{eq:monotonicity-BH-procedure-thresholds}.

Finally, when $\tau_2 \le \tau_1$, we have $\tau_2 \le \tau_1 \le x_1(i) \le x_2(i)$ with probability 1 for all $i\in\widehat{S}_1$.
Therefore, it follows that $\widehat{S}_1 \subseteq \widehat{S}_2$ and hence $|S\setminus\widehat{S}_2| \le |S\setminus\widehat{S}_1|$ almost surely. 
The first conclusion in Lemma \ref{lemma:monotonicity-BH-procedure} follows from the last inequality.
\end{proof}



\section{Proof of Theorem \ref{thm:detection-optimality}}
\label{subsec:proof-additive-error-detection-boundaries}







\section{Proof of Theorem \ref{thm:Gaussian-error-approx-boundary}}
\label{subsec:proof-additive-error-approx-boundaries}

We first show the necessary condition. 
That is, when $\overline{r} < \beta$, no thresholding procedure is able to achieve approximate support recovery.
The arguments are similar to that in Theorem 1 of \cite{arias2017distribution}, although we allow for unequal signal sizes. 

\begin{proof}[Proof of necessary condition in Theorem \ref{thm:Gaussian-error-approx-boundary}]
Denote the distributions of $\mathrm{N}(0,1)$, $\mathrm{N}(\underline{\Delta}, 1)$, and $\mathrm{N}((\overline{\Delta}, 1)$ as $F_0$, $F_{\underline{a}}$, and $F_{\overline{a}}$ respectively.

% We first show the necessary condition, i.e., when $\overline{r}<\beta$, approximate support recovery cannot be achieved with any thresholding procedure.
% In particular, we show that the liminf of the sum of FDP and NDP is at least 1.

Recall that thresholding procedures are of the form
$$
\widehat{S}_p = \left\{i\,|\,x(i) > t_p(x)\right\}.
$$
Denote $\widehat{S} := \left\{i\,|\,x(i) > t_p(x)\right\}$, and $\widehat{S}(u) := \left\{i\,|\,x(i) > u\right\}$.
For any threshold $u\ge t_p$ we must have $\widehat{S}(u)\subseteq\widehat{S}$, and hence
\begin{equation} \label{eq:approx-boundary-proof-FDP-Gaussian}
    \text{FDP} := \frac{|\widehat{S}\setminus{S}|}{|\widehat{S}|} \ge \frac{|\widehat{S}\setminus{S}|}{|\widehat{S}\cup{S}|} = \frac{|\widehat{S}\setminus{S}|}{|\widehat{S}\setminus{S}| + |S|} \ge
    \frac{|\widehat{S}(u)\setminus{S}|}{|\widehat{S}(u)\setminus{S}| + |S|}.
\end{equation}
On the other hand, for any threshold $u\le t_p$ we must have $\widehat{S}(u)\supseteq\widehat{S}$, and hence
\begin{equation} \label{eq:approx-boundary-proof-NDP-Gaussian}
    \text{NDP} := \frac{|{S}\setminus\widehat{S}|}{|{S}|} \ge 
    \frac{|{S}\setminus\widehat{S}(u)|}{|{S}|}.
\end{equation}
Since either $u\ge t_p$ or  $u\le t_p$ must take place, putting \eqref{eq:approx-boundary-proof-FDP-Gaussian} and \eqref{eq:approx-boundary-proof-NDP-Gaussian} together, we have
\begin{equation} \label{eq:approx-boundary-proof-converse-1-Gaussian}
    \text{FDP} + \text{NDP} 
    \ge \frac{|\widehat{S}(u)\setminus{S}|}{|\widehat{S}(u)\setminus{S}|+|{S}|} \wedge \frac{|{S}\setminus\widehat{S}(u)|}{|{S}|},
\end{equation}
for any $u$.
Therefore it suffices to show that for a suitable choice of $u$, the RHS of \eqref{eq:approx-boundary-proof-converse-1-Gaussian} converges to 1 in probability; the desired conclusion on FDR and FNR follows by the dominated convergence theorem.

Let $t^* = \sqrt{2q\log{p}}$ for some fixed $q$, we obtain an estimate of the tail probability by Mill's ratio \eqref{eq:Mills-ratio}, 
\begin{equation}
    \overline{F_0}(t^*) 
    \sim \frac{1}{t^*}\phi(t^*)
    = \frac{1}{2\sqrt{\pi q\log{p}}} p^{-q}, \label{eq:approx-boundary-proof-null-tail-prob-Gaussian}
\end{equation}
where $a_p\sim b_p$ is taken to mean $a_p/b_p\to 1$.
Observe that $|\widehat{S}(t^*)\setminus{S}|$ has distribution $\text{Binom}(p-s, \overline{F_0}(t^*))$ where $s=|S|$, denote $X = X_p := {|\widehat{S}(t^*)\setminus{S}|}/{|S|}$, and we have 
$$
\mu := \E\left[X\right] = \frac{(p-s)\overline{F_0}(t^*)}{s},
\quad \text{and} \quad
\var\left(X\right) = \frac{(p-s)\overline{F_0}(t^*){F_0}(t^*)}{s^2} \le \mu/s.
$$
Therefore for any $M>0$, we have, by Chebyshev's inequality,
\begin{equation}
    \P\left[X < M\right] 
    \le \P\left[\left|X-\mu\right| > \mu - M\right]
    \le \frac{\mu/s}{(\mu-M)^2}
    = \frac{1/(\mu s)}{(1-M/\mu)^2}. \label{eq:approx-boundary-proof-converse-2-Gaussian}
\end{equation}
Now, from the expression of $\overline{F_0}(t^*)$ in \eqref{eq:approx-boundary-proof-null-tail-prob-Gaussian}, we obtain
$$
\mu = (p^\beta - 1)\overline{F_0}(t^*) \sim \frac{1}{2\sqrt{\pi q\log{p}}} p^{\beta-q}.
$$
Since $\overline{r}<\beta$, we can pick $q$ such that $\overline{r}<q<\beta$. 
In turn, we have $\mu \to\infty$, as $p\to\infty$.
Therefore the last expression in \eqref{eq:approx-boundary-proof-converse-2-Gaussian} converges to 0, and we conclude that $X\to\infty$ in probability, and hence
\begin{equation} \label{eq:approx-boundary-proof-converse-3-Gaussian}
\frac{|\widehat{S}(t^*)\setminus{S}|}{|\widehat{S}(t^*)\setminus{S}|+|{S}|} 
= \frac{X}{X+1} \to 1 \quad \text{in probability}.
\end{equation}

On the other hand, we show that with the same choice of $u = t^*$, we have,
\begin{equation} \label{eq:approx-boundary-proof-converse-4-Gaussian}
    \frac{|{S}\setminus\widehat{S}(t^*)|}{|{S}|}\to 1 \quad \text{in probability}.
\end{equation}
By the stochastic monotonicity of Gaussian location family \eqref{eq:stochastic-monotonicity-Gaussian}, we have the following lower bound for the probability of missed detection for each signal $\mu(i)$, $i\in S$, 
\begin{equation} \label{eq:approx-boundary-proof-converse-5-Gaussian}
    \P[\mathrm{N}(\mu(i), 1) \le t^*] \ge F_{\overline{a}}(t^*).
\end{equation}
Since $|{S}\setminus\widehat{S}(t^*)|$ can be written as the sum of $s$ independent Bernoulli random variables,
\begin{equation*}
    |{S}\setminus\widehat{S}(t^*)| = \sum_{i\in S} \mathbbm{1}_{(-\infty, t^*]}(x(i)),
\end{equation*}
using with \eqref{eq:approx-boundary-proof-converse-5-Gaussian}, we conclude that $|{S}\setminus\widehat{S}(t^*)| \stackrel{\mathrm{d}}{\ge} \text{Binom}(s, {F_{\overline{a}}}(t^*))$.
Finally, we know that ${F_{\overline{a}}}(t^*)$ converges to 1 by our choice of diverging $t^*$, and the necessary condition is shown.
\end{proof}

We now turn to the sufficient condition. 
That is, when $\underline{r} > \beta$, the Benjamini-Hochberg procedure with slowly vanishing FDR levels achieves asymptotic approximate support recovery.

\begin{proof}[Proof of the sufficient condition in Theorem \ref{thm:Gaussian-error-approx-boundary}]
The FDR vanishes by our choice of $\alpha$ and the FDR-controlling property of the BH procedure \citep{benjamini1995controlling}.
It only remains to show that FNR also vanishes.

To do so we compare the FNR under the alternative specified in Theorem \ref{thm:Gaussian-error-approx-boundary} to one with all of the signal sizes equal to $\underline{\Delta}$.
By Lemma \ref{lemma:monotonicity-BH-procedure}, it suffices to show that the FNR under the BH procedure in this setting vanishes.
Let $x(i)$ be vectors of independent observations with $p-s$ nulls having standard Gaussian distributions, and $s$ signals having $\mathrm{N}(\underline{\Delta}, 1)$ distributions.

Denote the null and the alternative distributions as $F_0$ and $F_{a}$ respectively.
Let $\widehat{G}$ denote the empirical survival function as in \eqref{eq:empirical-tail-distribution}.
Define the empirical survival functions for the null part and signal part
\begin{equation} \label{eq:empirical-survival-null-signal-Gaussian}
    \widehat{W}_\text{null}(t) = \frac{1}{p-s}\sum_{i\not\in S}\mathbbm{1}\{x(i) \ge t\},
    \quad
    \widehat{W}_\text{signal}(t) = \frac{1}{s}\sum_{i\in S}\mathbbm{1}\{x(i) \ge t\},
\end{equation}
where $s=|S|$, so that
$$
\widehat{G}(t) = \frac{p-s}{p}\widehat{W}_\text{null}(t) + \frac{s}{p}\widehat{W}_\text{signal}(t).
$$

We need the following result to describe the deviations of the empirical distributions.
\begin{lemma}[Theorem 1 of \citet{eicker1979asymptotic}] \label{lemma:empirical-process}
Let $Z_1,\ldots,Z_k$ be iid with continuous survival function $Q$.
Let $\widehat{Q}_k$ denote their empirical survival function and define 
$\xi_k = \sqrt{2\log{\log{(k)}}/k}$ for $k \ge 3$. 
Then
$$
\frac{1}{\xi_k}\sup_z\frac{|\widehat{Q}_k(z) - Q(z)|}{\sqrt{Q(z)(1 - Q(z))}} \to 1,
$$
in probability as $k \to \infty$.
In particular,
$$
\widehat{Q}_k(z) = Q(z) + O_\P\left(\xi_k\sqrt{Q(z)(1 - Q(z))}\right),
$$
uniformly in z.
\end{lemma}

Apply Lemma \ref{lemma:empirical-process} to the two summands in $\widehat{G}$, we obtain
$\widehat{G}(t) = G(t) + \widehat{R}(t)$,
where 
\begin{equation} \label{eq:empirical-process-mean-Gaussian}
    G(t) = \frac{p-s}{p}\overline{F_0}(t) + \frac{s}{p}\overline{F_a}(t),
\end{equation}
and 
\begin{equation} \label{eq:empirical-process-residual-Gaussian}
    \widehat{R}(t) = O_\P\left(\xi_p\sqrt{\overline{F_0}(t)F_0(t)} + \frac{s}{p}\xi_s\sqrt{\overline{F_a}(t)F_a(t)}\right),
\end{equation}
uniformly in $t$.

Recall (see proof of Lemma \ref{lemma:monotonicity-BH-procedure}) that the BH procedure is the thresholding procedure with threshold set at 
\begin{equation} \label{eq:approx-boundary-proof-tau-Gaussian}
    \tau = \inf\{t\,|\,\overline{F_0}(t)\le\alpha\widehat{G}(t)\}. 
    %= \min\{t\,|\,\overline{F_0}(t)=\alpha\widehat{G}(t)\}.
\end{equation}
The NDP may also be re-written as 
$$
\text{NDP} = \frac{|{S}\setminus\widehat{S}|}{|{S}|} = \frac{1}{s}\sum_{i\in S}\mathbbm{1}\{x(i) < \tau\} = 1 - \widehat{W}_\text{signal}(\tau),
$$
so that it suffices to show that 
\begin{equation} \label{eq:approx-boundary-proof-sufficient-1-Gaussian}
    \widehat{W}_\text{signal}(\tau)\to 1
\end{equation} in probability.
Applying Lemma \ref{lemma:empirical-process} to $\widehat{W}_\text{signal}$, we know that 
$$
\widehat{W}_\text{signal}(\tau) = \overline{F_a}(\tau) + O_\P\left(\xi_s\sqrt{\overline{F_a}(\tau)F_a(\tau)}\right) = \overline{F_a}(\tau) + o_\P(1).
$$
So it suffices to show that $F_a(\tau)\to 0$ in probability.
Now let $t^* = \sqrt{2q\log(p)}$ for some $q$ such that $\beta<q<\underline{r}$.
We have 
\begin{equation} \label{eq:approx-boundary-proof-sufficient-2-Gaussian}
    F_a(t^*) 
    = \Phi(t^* - \underline{\Delta}) 
    = \Phi(\sqrt{2(q - \underline{r})\log{p}}) \to 0. 
\end{equation}
Hence in order to show \eqref{eq:approx-boundary-proof-sufficient-1-Gaussian}, it suffices to show 
\begin{equation} \label{eq:approx-boundary-proof-sufficient-3-Gaussian}
    \P\left[\tau \le t^*\right] \to 1.
\end{equation}
By \eqref{eq:empirical-process-mean-Gaussian}, the mean of the empirical process $\widehat{G}$ evaluated at $t^*$ is
\begin{equation} \label{eq:approx-boundary-proof-sufficient-4-Gaussian}
    G(t^*) = \frac{p-s}{p}\overline{F_0}(t^*) + \frac{s}{p}\overline{F_a}(t^*).
\end{equation}
The first term, using Relation \eqref{eq:approx-boundary-proof-null-tail-prob-Gaussian}, is asymptotic to $p^{-q}L(p)$, where $L(p)$ is the logarithmic term in $p$.
The second term, since $\overline{F_a}(t^*)\to 1$ by Relation \eqref{eq:approx-boundary-proof-sufficient-2-Gaussian}, is asymptotic to $p^{-\beta}$.
Therefore, $G(t^*) \sim p^{-q}L(p) + p^{-\beta} \sim p^{-\beta}$, since 
$p^{\beta-q}L(p)\to0$ where $q>\beta$.

The fluctuation of the empirical process at $t^*$, by Relation \eqref{eq:empirical-process-residual-Gaussian}, is 
\begin{align*}
    \widehat{R}(t^*) 
    &= O_\P\left(\xi_p\sqrt{\overline{F_0}(t^*)F_0(t^*)} + \frac{s}{p}\xi_s\sqrt{\overline{F_a}(t^*)F_a(t^*)}\right)\\
    &= O_\P\left(\xi_p\sqrt{\overline{F_0}(t^*)}\right) + o_\P\left(p^{-\beta}\right).
\end{align*}
By \eqref{eq:approx-boundary-proof-null-tail-prob-Gaussian} and the expression for $\xi_p$, the first term is $O_\P\left(p^{-(q+1)/2}L(p)\right)$ where $L(p)$ is a poly-logarithmic term in $p$.
Since $\beta<\min\{q,1\}$, we have $\beta<(q+1)/2$, and hence $\widehat{R}(t^*) = o_\P(p^{-\beta})$.

Putting the mean and the fluctuation of $\widehat{G}(t^*)$ together, we obtain
$$
\widehat{G}(t^*) = G(t^*) + \widehat{R}(t^*) \sim_\P G(t^*) \sim p^{-\beta},
$$
and therefore, together with \eqref{eq:approx-boundary-proof-null-tail-prob-Gaussian}, we have
$$
\overline{F_0}(t^*)/\widehat{G}(t^*) = p^{\beta-q}L(p)(1+o_{\P}(1)),
$$
which is eventually smaller than the FDR level $\alpha$ by the assumption \eqref{eq:slowly-vanishing-error} and the fact that $\beta<q$.
That is, 
$$
\P\left[\overline{F}_0(t^*) / \widehat{G}(t^*) < \alpha\right] \to 1.
$$
By definition of $\tau$ (recall \eqref{eq:approx-boundary-proof-tau-Gaussian}), this implies that $\tau \le t^*$ with probability tending to 1, and \eqref{eq:approx-boundary-proof-sufficient-3-Gaussian} is shown.
The proof for the sufficient condition is complete.
\end{proof}










\section{Proof of Theorems \ref{thm:Gaussian-error-exact-approx-boundary} and \ref{thm:Gaussian-error-approx-exact-boundary}}
\label{subsec:proof-additive-error-mix-boundaries}


Proof of Theorem \ref{thm:Gaussian-error-exact-approx-boundary} uses ideas from the proof of Theorem \ref{thm:Gaussian-error-approx-boundary} and is substantially shorter.


\begin{proof}[Proof of Theorem \ref{thm:Gaussian-error-exact-approx-boundary}]
We first show the sufficient condition. 
Vanishing FWER is guaranteed by the properties of the procedures, and we only need to show that FNR also goes to zero. 
Similar to the proof of Theorem \ref{thm:Gaussian-error-approx-boundary}, it suffices to show that
\begin{equation} \label{eq:additive-error-exact-approx-boundary-proof-sufficient-1}
    \text{NDP} = 1 - \widehat{W}_\text{signal}(t_p) \to 0,
\end{equation}
where $t_p$ is the threshold of Bonferroni's procedure.

Since $\alpha$ vanishes slowly (see Definition \ref{eq:slowly-vanishing-error}), for any $\delta>0$, we have $p^{-\delta}=o(\alpha)$.
Therefore, we have $-\log\alpha\le\delta\log{p}$ for large $p$, and
\begin{equation*} 
    1 \le \limsup_{p\to\infty}\frac{2\log{p} - 2\log{\alpha}}{2\log{p}} \le 1+\delta,
\end{equation*}
for any $\delta>0$.
Therefore, by the expression for normal quantiles, we know that 
$$
t_p=F^\leftarrow(1-\alpha/p)\sim(2\log{p}-2\log{\alpha})^{1/2} \sim(2\log{p})^{1/2}.
$$

Since $\underline{r}>\widetilde{g}(\beta)=1$, we can pick $q$ such that $1<q<\underline{r}$.
Let $t^* = \sqrt{2q\log{p}}$, we know that $t_p<t_p^*$ for large $p$.
Therefore for large $p$, we have
$$
\widehat{W}_\text{signal}(t_p) \ge \widehat{W}_\text{signal}(t^*) \ge \overline{F_a}(t^*) + o_\P(1),
$$
where $\overline{F_a}$ is the survival function of $\mathrm{N}(\sqrt{2\underline{r}\log{p}}, 1)$; the last inequality follows from the stochastic monotonicity of the Gaussian location family \eqref{eq:stochastic-monotonicity-Gaussian}, and Lemma \ref{lemma:empirical-process}.
Indeed, by our choice of $q<\underline{r}$, we obtain
$$
F_a(t^*) = \Phi\left(\sqrt{2(q-\underline{r})\log{p}}\right)\to0,
$$
and \eqref{eq:additive-error-exact-approx-boundary-proof-sufficient-1} is shown. 
This completes the proof of the sufficient condition.

The proof of the necessary condition follows similar structure as in the proof of Theorem \ref{thm:Gaussian-error-approx-boundary}, and uses the lower bound
\begin{equation} \label{eq:additive-error-exact-approx-boundary-proof-necessary-1}
    \mathrm{FWER}(\mathcal{R}) + \mathrm{FNR}(\mathcal{R}) \ge \P\left[\max_{i\in S^c}x(i)>u\right] \wedge \E\left[\frac{|S\setminus \widehat{S}(u)|}{|S|}\right],
\end{equation}
which holds for any arbitrary thresholding procedure $\mathcal{R}$ and arbitrary real $u\in\R$.

By the assumption that $\overline{r}<\widetilde{g}(\beta)=1$, we can pick $q$ such that $\overline{r}<q<1$ and let $u = t^*=\sqrt{2q\log{p}}$ in \eqref{eq:additive-error-exact-approx-boundary-proof-necessary-1}.
By relative stability of iid Gaussian random variables \eqref{eq:relative-stability-Gaussian-maxima}, we have
\begin{equation} \label{eq:additive-error-exact-approx-boundary-proof-necessary-2}
    \P\left[\frac{\max_{i\in S^c} x(i)}{\sqrt{2\log{p}}} > \frac{t^*}{\sqrt{2\log{p}}}\right] \to 1.
\end{equation}
since the first fraction in \eqref{eq:additive-error-exact-approx-boundary-proof-necessary-2} converges to 1, while the second converges to $q<1$.
Therefore, the first term on the right-hand side of \eqref{eq:additive-error-exact-approx-boundary-proof-necessary-1} converges to 1.

On the other hand, by the stochastic monotonicity of Gaussian location family \eqref{eq:stochastic-monotonicity-Gaussian}, the probability of missed detection for each signal is lower bounded by $\P[Z+\mu(i) \le t^*] \ge F_{\overline{a}}(t^*)$, where $Z$ is a standard Gaussian r.v., and $F_{\overline{a}}$ is the cdf of $\mathrm{N}(\sqrt{2\overline{r}\log{p}}, 1)$.
Therefore, $|{S}\setminus\widehat{S}(t^*)| \stackrel{\mathrm{d}}{\ge} \text{Binom}(s, {F_{\overline{a}}}(t^*))$, and it suffices to show that ${F_{\overline{a}}}(t^*)$ converges to 1.
Indeed,
\begin{equation*}
    {F_{\overline{a}}}(t^*) = \Phi(\sqrt{2(q-\overline{r})\log{p}}) \to 1,
\end{equation*}
by our choice of $q>\overline{r}$.
Hence both quantities in the minimum on the right-hand side of \eqref{eq:additive-error-exact-approx-boundary-proof-necessary-1} converge to 1 in the limit, and the necessary condition is shown.
\end{proof}






\begin{proof}[Proof of Theorem \ref{thm:Gaussian-error-approx-exact-boundary}]
We first show the sufficient condition.
Since FDR control is guaranteed by the BH procedure, we only need to show that the FWNR also vanishes, that is,
\begin{equation} \label{eq:approx-exact-boundary-proof-sufficient-1-Gaussian}
    \P\left[\min_{i\in S}x(i) \ge \tau\right] \to 1,
\end{equation}
where $\tau$ is the threshold for the BH procedure.

By the assumption that $\underline{r}>\widetilde{h}(\beta)=(\sqrt{\beta}+\sqrt{1-\beta})^2$, we have $\sqrt{\underline{r}}-\sqrt{1-\beta}>\sqrt{\beta}$, so we can pick $q>0$, such that 
\begin{equation} \label{eq:approx-exact-boundary-proof-sufficient-2-Gaussian}
\sqrt{\underline{r}}-\sqrt{1-\beta}>\sqrt{q}>\sqrt{\beta}.
\end{equation}
We only need to show that with a specific choice of $t^*=\sqrt{2q\log{p}}$ where
\begin{equation} \label{eq:additive-error-approx-exact-boundary-proof-sufficient-1}
\sqrt{\underline{r}}-\sqrt{1-\beta}>\sqrt{q}>\sqrt{\beta},
\end{equation}
we have both
\begin{equation} \label{additive-error-eq:approx-exact-boundary-proof-sufficient-2}
\P\left[\tau\le t^*\right]\to 1,
\end{equation}
and 
\begin{equation} \label{eq:additive-error-approx-exact-boundary-proof-sufficient-3}
    \P\left[\min_{i\in S}x(i) \ge t^* \right] \to 1,
\end{equation}
so that 
\begin{equation*} 
    \P\left[\min_{i\in S}x(i) \ge \tau\right] \ge 
    % \P\left[\min_{i\in S}x(i) \ge t^* \ge \tau\right] \ge
    \P\left[\min_{i\in S}x(i) \ge t^*,\; t^* \ge \tau\right] \to 1.
\end{equation*}

Relation \eqref{additive-error-eq:approx-exact-boundary-proof-sufficient-2} follows in exactly the same way \eqref{eq:approx-boundary-proof-sufficient-3-Gaussian} did on page  \pageref{eq:approx-boundary-proof-sufficient-3-Gaussian}.

Dividing the left-hand-side in Relation \eqref{eq:additive-error-approx-exact-boundary-proof-sufficient-3} by $\sqrt{2\log{p}}$, we have,
\begin{align*}
    \frac{\min_{i\in S}x(i)}{\sqrt{2\log{p}}} 
    &= \frac{\min_{i\in S}\mu(i)+\epsilon(i)}{\sqrt{2\log{p}}} 
    \stackrel{\mathrm{d}}{\ge} \frac{\sqrt{2\underline{r}\log{p}} + \min_{i\in S}\epsilon(i)}{\sqrt{2\log{p}}} \\
    &\to -\sqrt{1-\beta} + \sqrt{\underline{r}},
\end{align*}
where the last convergence follows from the relative stability of iid Gaussians minima \eqref{eq:relative-stability-Gaussian-minima}. 
On the other hand, ${t^*}/{\sqrt{2\log{p}}}=\sqrt{q}<\sqrt{\underline{r}}-\sqrt{1-\beta}$ by our choice of ${q}$, and Relation \eqref{eq:additive-error-approx-exact-boundary-proof-sufficient-3} follows.


The necessary condition follows from the lower bound
\begin{equation} \label{eq:additive-error-approx-exact-boundary-proof-necessary-1}
    \mathrm{FDR}(\mathcal{R}) + \mathrm{FWNR}(\mathcal{R}) \ge \E\left[\frac{|\widehat{S}(u)\setminus S|}{|\widehat{S}(u)\setminus S| + |S|}\right] \wedge 
    \P\left[\min_{i\in S}x(i)<u\right],
\end{equation}
which holds for any thresholding procedure $\mathcal{R}$ and for arbitrary $u\in\R$.
In particular, we show that both terms in the minimum in \eqref{eq:additive-error-approx-exact-boundary-proof-necessary-1} converge to 1 when we set $u=t^*=\sqrt{2q\log{p}}$ where 
\begin{equation}
\sqrt{\overline{r}}-\sqrt{1-\beta}<\sqrt{q}<\sqrt{\beta}.
\end{equation}

On the one hand, we have,
$$
\frac{\min_{i\in S}x(i)}{\sqrt{2\log{p}}} 
\stackrel{\mathrm{d}}{\le} \frac{\min_{i\in S}\epsilon(i)+\sqrt{2\overline{r}\log{p}}}{\sqrt{2\log{p}}} 
\to \sqrt{\overline{r}}-\sqrt{1-\beta},
$$
by relative stability of iid Gaussians \eqref{eq:relative-stability-Gaussian-minima}. On the other hand, ${t^*}/{\sqrt{2\log{p}}}=\sqrt{q}>\sqrt{\underline{r}}-\sqrt{1-\beta}$ by our choice of ${q}$;
this shows that the second term on the right-hand side of \eqref{eq:additive-error-approx-exact-boundary-proof-necessary-1} converges to 1.

Observe that $|\widehat{S}(t^*)\setminus{S}|$ has distribution $\text{Binom}(p-s, \overline{\Phi}(t^*))$, and define $X = X_p := {|\widehat{S}(t^*)\setminus{S}|}/{|S|}$, we obtain,
% On the other hand, define $ = \E[|S\setminus\widehat{S}(t^*)|/|S|]$,
\begin{align*}
    \mu &:= \E[X] = (p^\beta-1)\overline{\Phi}(t^*) 
    \sim (p^\beta-1)\frac{\phi(t^*)}{t^*} \\
    &\sim \frac{1}{\sqrt{2\pi}}\left(2q\log{p}\right)^{-1/2}p^{\beta-q}\to\infty,
\end{align*}
where the divergence follows from our choice of $q<\beta$.
Using again Relations \eqref{eq:approx-boundary-proof-converse-2-Gaussian} and \eqref{eq:approx-boundary-proof-converse-3-Gaussian}, we conclude that the first term on the right-hand side of \eqref{eq:additive-error-approx-exact-boundary-proof-necessary-1} also converges to 1.
This completes the proof of the necessary condition.
\end{proof}
