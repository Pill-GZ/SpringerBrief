
We review some properties of the chi-square distributions in Section \ref{sec:chi-square-distributions}, before presenting the proofs of the main theorems on phase transitions in Sections \ref{subsec:proof-chi-squared-exact-boundary}, \ref{subsec:proof-chi-squared-approx-boundary}, and \ref{subsec:proof-chi-squared-mix-boundaries}.
Results relating signal sizes and effect sizes in association tests will be justified in Section \ref{subsec:proof-signal-size-odds-ratio}.


\subsection{Auxiliary facts of chi-square distributions}
\label{sec:chi-square-distributions}
We shall recall, and establish, some auxiliary facts about chi-square distributions. 
These facts will be used in the proofs of Theorem \ref{thm:chi-squared-exact-boundary} and Theorem \ref{thm:chi-squared-approx-boundary}.

\begin{lemma}[Rapid variation of chi-square distribution tails] \label{lemma:rapid-variation-chisq}
The central chi-square distribution with $\nu$ degrees of freedom has rapidly varying tails.
That is, 
\begin{equation} \label{eq:rapid-variation-chisq}
    \lim_{x\to\infty}\frac{\P[\chi_\nu^2(0)>tx]}{\P[\chi_\nu^2(0)>x]} = 
    \begin{cases}
    0, & t > 1 \\
    1, & t = 1 \\
    \infty, & 0 < t < 1
\end{cases},
\end{equation}
where we overloaded the notation $\chi_\nu^2(0)$ to represent a random variable with the chi-square distribution.
\end{lemma}

\begin{proof}[Proof of Lemma \ref{lemma:rapid-variation-chisq}]
When $\nu=1$, the chi-square distribution reduces to a squared Normal, and \eqref{eq:rapid-variation-chisq} follows from the rapid variation of the standard Normal distribution.
For $\nu\ge2$, we recall the following bound on tail probabilities (see, e.g., \citep{inglot2010inequalities}),
$$
\frac{1}{2}\mathcal{E}_\nu(x) \le \P[\chi_\nu^2(0)>x] \le \frac{x}{(x-\nu+2)\sqrt{\pi}} \mathcal{E}_\nu(x), \quad \nu\ge2,\;x>\nu-2,
$$
where $\mathcal{E}_\nu(x) = \exp\left\{-\frac{1}{2}[(x-\nu-(\nu-2)\log(x/\nu) + \log\nu]\right\}$.
Therefore, we have 
$$
\frac{(x-\nu+2)\sqrt{\pi}}{2x}\frac{\mathcal{E}_\nu(tx)}{\mathcal{E}_\nu(x)} 
\le \frac{\P[\chi_\nu^2(0)>tx]}{\P[\chi_\nu^2(0)>x]}
\le \frac{2tx}{(tx-\nu+2)\sqrt{\pi}}\frac{\mathcal{E}_\nu(tx)}{\mathcal{E}_\nu(x)},
$$
where ${\mathcal{E}_\nu(tx)}/{\mathcal{E}_\nu(x)} = \exp\{-\frac{1}{2}[(t-1)x-(\nu-2)\log{t}]\}$ converges to $0$ or $\infty$ depending on whether $t>1$ or $0<t<1$.
The case where $t=1$ is trivial.
\end{proof}

Lemma \ref{lemma:rapid-variation-chisq} and Proposition \ref{prop:rapid-varying-tails} yield the following Corollary.

\begin{corollary} \label{cor:relative-stability}
Maxima of independent observations from central chi-square distributions with $\nu$ degrees of freedom are relatively stable. 
Specifically, let $\epsilon_p = \left(\epsilon_p(i)\right)_{i=1}^p$ be independently and identically distributed (iid) $\chi_\nu^2(0)$ random variables. 
Then the triangular array ${\cal E} = \{\epsilon_p, p\in\N\}$ has relatively stable (RS) maxima in the sense of \eqref{eq:RS-condition}.
\end{corollary}


\begin{lemma}[Stochastic monotonicity] \label{lemma:stochastic-monotonicity}
The non-central chi-square distribution is stochastically monotone in its non-centrality parameter.
Specifically, for two non-central chi-square distributions both with $\nu$ degrees of freedom, and non-centrality parameters $\lambda_1 \le \lambda_2$, we have $\chi^2_\nu(\lambda_1) \stackrel{\mathrm{d}}{\le} \chi^2_\nu(\lambda_2)$. 
That is,
\begin{equation} \label{eq:stochastic-monotonicity}
    \P[\chi^2_\nu(\lambda_1) \le t] \ge \P[\chi^2_\nu(\lambda_2) \le t], \quad \text{for any}\quad t\ge0.
\end{equation}
where we overloaded the notation $\chi_\nu^2(\lambda)$ to represent a random variable with the chi-square distribution with non-centrality parameter $\lambda$ and degree-of-freedom parameter $\nu$.
\end{lemma}

\begin{proof}[Proof of Lemma \ref{lemma:stochastic-monotonicity}]
Recall that non-central chi-square distributions can be written as sums of $\nu-1$ standard normal random variables and a non-central normal random variable with mean $\sqrt{\lambda}$ and variance 1,
\begin{equation*}
    \chi_\nu^2(\lambda) 
    \stackrel{\mathrm{d}}{=} Z_1^2 + \ldots + Z_{\nu-1}^2 + (Z_\nu + \sqrt{\lambda})^2.
\end{equation*}
Therefore, it suffices to show that $\P[(Z+\sqrt{\lambda})^2 \le t]$ is non-increasing in $\lambda$ for any $t\ge0$, where $Z$ is a standard normal random variable.
We rewrite this expression in terms of standard normal probability function $\Phi$,
\begin{align}
    \P[(Z+\sqrt{\lambda})^2 \le t] 
    &= \P[-\sqrt{\lambda} - \sqrt{t} \le Z \le -\sqrt{\lambda} + \sqrt{t}] \nonumber \\
    &= \Phi(-\sqrt{\lambda} + \sqrt{t}) - \Phi(-\sqrt{\lambda} - \sqrt{t}). \label{eq:stochastic-monotonicity-proof-1}
\end{align}
The derivative of the last expression (with respect to $\lambda$) is 
\begin{equation} \label{eq:stochastic-monotonicity-proof-2}
    \frac{1}{2\sqrt{\lambda}} \left(\phi(\sqrt{\lambda} + \sqrt{t}) - \phi(\sqrt{\lambda} - \sqrt{t})\right) 
    = \frac{1}{2\sqrt{\lambda}} \left(\phi(\sqrt{\lambda} + \sqrt{t}) - \phi(\sqrt{t} - \sqrt{\lambda})\right),
\end{equation}
where $\phi$ is the density of the standard normal distribution.
Notice that we have used the symmetry of $\phi$ around 0 in the last expression.

Since $0 \le \max\{\sqrt{\lambda} - \sqrt{t}, \sqrt{t} - \sqrt{\lambda}\} < \sqrt{t} + \sqrt{\lambda}$ when $t>0$, by monotonicity of the normal density on $(0,\infty)$, we conclude that the derivative \eqref{eq:stochastic-monotonicity-proof-2} is indeed negative.
Therefore, \eqref{eq:stochastic-monotonicity-proof-1} is decreasing in $\lambda$, and \eqref{eq:stochastic-monotonicity} follows for $t>0$.
For $t = 0$, equality holds in \eqref{eq:stochastic-monotonicity} with both probabilities being 0.
\end{proof}


Finally, we derive asymptotic expressions for  chi-square quantiles.

\begin{lemma}[Chi-square quantiles] \label{lemma:chisq-quantiles}
Let $F$ be the central chi-square distributions with $\nu$ degrees of freedom, and let $u(y)$ be the $(1-y)$-th generalized quantile of $F$, i.e.,
\begin{equation} \label{eq:quantiles-generic}
    u(y) = F^\leftarrow(1 - y).
\end{equation}
Then 
\begin{equation}
    u(y) \sim 2\log(1/y), \quad \text{as }y\to0. 
\end{equation}
\end{lemma}

\begin{proof}[Proof of Lemma \ref{lemma:chisq-quantiles}]
The case where $\nu=1$ follows from the well-known formula for Normal quantiles (see, e.g., Proposition 1.1 in \cite{gao2018fundamental})
$$
F^\leftarrow(1 - y) = \Phi^\leftarrow(1-y/2)\sim\sqrt{2\log{(2/y)}}\sim\sqrt{2\log{(1/y)}}.
$$
The case where $\nu\ge2$ follows from the following estimates of high quantiles of chi-square distributions (see, e.g., \citep{inglot2010inequalities}),
$$
    \nu +  2\log(1/y) -5/2 \le u(y) \le \nu +  2\log(1/y) + 2\sqrt{\nu\log(1/y)}, \quad \text{for all }y\le0.17,
$$
where both the lower and upper bound are asymptotic to $2\log(1/y)$.
\end{proof}




\subsection{Proof of Theorem \ref{thm:chi-squared-exact-boundary}}
\label{subsec:proof-chi-squared-exact-boundary}

\begin{proof}[Proof of Theorem \ref{thm:chi-squared-exact-boundary}]
We first prove the sufficient condition.
The Bonferroni procedure sets the threshold at $t_p = F^\leftarrow(1-\alpha/p)$, which, by Lemma \ref{lemma:chisq-quantiles}, is asymptotic to $2\log{p} - 2\log{\alpha}$.
By the assumption on $\alpha$ in \eqref{eq:slowly-vanishing-error}, for any $\delta>0$, we have $p^{-\delta}=o(\alpha)$.
Therefore, we have $-\log\alpha\le\delta\log{p}$ for large $p$, and
\begin{equation*} 
    1 \le \limsup_{p\to\infty}\frac{2\log{p} - 2\log{\alpha}}{2\log{p}} \le 1+\delta,
\end{equation*}
for any $\delta>0$.
Hence, $t_p\sim 2\log{p}$.

The condition $\underline{r} > {{g}}(\beta)$ implies, after some algebraic manipulation,
$\sqrt{\underline{r}} -\sqrt{1-\beta} > 1$.
Therefore, we can pick $q>1$ such that 
\begin{equation} \label{eq:choice-of-q}
    \sqrt{\underline{r}} -\sqrt{1-\beta} > \sqrt{q} > 1.
\end{equation}
Setting the $t^* = t^*_p = 2q\log{p}$, we have $t_p < t^*_p$ for large $p$.

On the one hand, $\text{FWER} = 1 - \P[\widehat{S}_p \subseteq S_p]$ vanishes under the Bonferroni procedure with $\alpha\to0$.
On the other hand, for large $p$, the probability of no missed detection is bounded from below by
\begin{equation} \label{eq:chi-square-sufficient-1}
    \P[\widehat{S}_p \supseteq S_p] 
    = \P[\min_{i\in S} x(i) \ge t_p] 
    \ge \P[\min_{i\in S} x(i) \ge t^*] 
    \ge 1 - p^{1-\beta}\P[\chi_\nu^2(\underline{\Delta}) < t^*],
\end{equation}
where we have used the fact that signal sizes are bounded below by $\underline{\Delta}$, and the stochastic monotonicity of chi-square distributions (Lemma \ref{lemma:stochastic-monotonicity}) in the last inequality.
Writing
$$
\chi_\nu^2(\underline{\Delta}) \stackrel{\mathrm{d}}{=} Z_1^2 + \ldots + Z_{\nu-1}^2 + (Z_\nu + \sqrt{\underline{\Delta}})^2
$$
where $Z_i$'s are iid standard normal variables, we have
\begin{align}
    \P[\chi_\nu^2({\underline{\Delta}}) < t^*]
    &\le \P[(Z_\nu+\sqrt{\underline{\Delta}})^2 < t^*] 
    = \P[|Z_\nu+\sqrt{\underline{\Delta}}| < \sqrt{t^*}]  \nonumber \\
    &\le \P\left[Z_\nu < - \sqrt{\underline{\Delta}} +  \sqrt{t^*}\right] \nonumber \\
    &= \P\left[Z_\nu < \sqrt{2\log{p}}\left(\sqrt{q} - \sqrt{\underline{r}}\right)\right]. \label{eq:chi-square-sufficient-2}
\end{align}
By our choice of $q$ in \eqref{eq:choice-of-q}, the last probability in \eqref{eq:chi-square-sufficient-2} can be bounded from above by 
\begin{align*}
    \P\Big[Z_\nu < -\sqrt{2(1-\beta)\log{p}}\Big]
    &\sim \frac{\phi\left(-\sqrt{2(1-\beta)\log{p}}\right)}{\sqrt{2(1-\beta)\log{p}}} \\
    &= \frac{1}{\sqrt{2(1-\beta)\log{p}}}p^{-(1-\beta)},
\end{align*}
where the first line uses Mill's ratio for Gaussian distributions (see Section \ref{sec:Gaussian-distributions} and Relation \eqref{eq:Mills-ratio}).
This, combined with \eqref{eq:chi-square-sufficient-1}, completes the proof of the sufficient condition for the Bonferroni's procedure.

Under the assumption of independence, Sid\'ak's, Holm's, and Hochberg's procedures are strictly more powerful than Bonferroni's procedure, while controlling FWER at the nominal levels.
Therefore, the risks of exact support recovery for these procedures also vanishes.
This completes the proof for the first part of Theorem \ref{thm:chi-squared-exact-boundary}.

We now show the necessary condition. 
We first normalize the maxima by the chi-square quantiles $u_p = F^{\leftarrow}(1-1/p)$, where $F$ is the distribution of a (central) chi-square random variable,
\begin{equation} \label{eq:chi-square-necessary-0}
 \P[\widehat{S}_p = S_p] \le \P\left[M_{S^c} <  t_p \le m_{S} \right]
  % &= \P\left[\frac{\max_{i\in S^c}x(i)}{u_p} < \frac{\min_{i\in S}x(i)}{u_p}\right] \nonumber \\
  % &\le  \P\left[\frac{\max_{i\in S^c}\chi_\nu^2(\lambda(i))}{u_p} < \frac{\min_{i\in S}\chi_\nu^2(\lambda(i))}{u_p}\right] \nonumber \\
  \le \P\left[ \frac{M_{S^c}}{u_p} < \frac{m_S}{u_p} \right],
\end{equation}
where $M_{S^c} = \max_{i\in S^c}x(i)$ and $m_{S} = \min_{i\in S}x(i)$.
By the relative stability of chi-square random variables (Corollary \ref{cor:relative-stability}), we know that ${M_{S^c}}/{u_{|S^c|}}\to1$ in probability. 
Further, using the expression for $u_p$ (Lemma \ref{lemma:chisq-quantiles}), we obtain
$$
\frac{u_{p-p^{1-\beta}}}{u_{p}} \sim \frac{2\log{(p-p^{1-\beta})}}{2\log{p}} = \frac{\log{p}+\log{(1-p^{-\beta})}}{\log{p}} \sim 1.
$$
Therefore, the left-hand-side of the last probability in \eqref{eq:chi-square-necessary-0} converges to 1,
\begin{equation} \label{eq:chi-square-necessary-1}
    \frac{M_{S^c}}{u_{p}} = \frac{M_{S^c}}{u_{p-p^{1-\beta}}} \frac{u_{p-p^{1-\beta}}}{u_{p}} \stackrel{\P}{\longrightarrow} 1.
\end{equation}

Meanwhile, for any $i\in S$, by Lemma \ref{lemma:stochastic-monotonicity} and the fact that signal sizes are bounded above by $\overline{\Delta}$, we have,
\begin{equation*}
    {\chi_\nu^2(\lambda(i))} \stackrel{\mathrm{d}}{\le}
    {\chi_\nu^2(\overline{\Delta})} \stackrel{\mathrm{d}}{=} 
    {Z_1^2 + \ldots + Z_{\nu-1}^2 + \left(Z_\nu + \sqrt{\overline{\Delta}}\right)^2}.
\end{equation*}
Dividing through by $u_p$, and taking minimum over $S$, we obtain
\begin{equation} \label{eq:chi-square-necessary-3}
    \frac{m_S}{u_p} 
    = \min_{i\in S} \frac{\chi_\nu^2(\lambda(i))}{u_p} 
    % \stackrel{\mathrm{d}}{\le} \min \left\{\frac{\chi_\nu^2(\overline{\Delta})}{u_p}, s \text{ iid copies} \right\} \\
    \stackrel{\mathrm{d}}{\le} 
    \min_{i\in S}\left\{\frac{Z_1^2(i) + \ldots + Z_{\nu-1}^2(i)}{u_p} + \frac{(Z_\nu(i) + \sqrt{\overline{\Delta}})^2}{u_p}\right\}.
\end{equation}
Let $i^\dagger = i^\dagger_p$ be the index minimizing the second term in \eqref{eq:chi-square-necessary-3}, i.e.,
\begin{equation}
    i^\dagger := \argmin_{i\in S} \frac{(Z_\nu(i) + \sqrt{\overline{\Delta}})^2}{u_p}
    = \argmin_{i\in S} f_p\left(Z_\nu(i)\right),
    % \frac{(Z_\nu(i) + \sqrt{\overline{\Delta}})^2}{2\log{p}},
\end{equation}
where $f_p(x):=(x+\sqrt{\overline{\Delta}})^2/(2\log{p})$. 
We shall first show that 
\begin{equation} \label{eq:chi-square-necessary-4}
    %\min_{i\in S} \frac{(Z_\nu(i) + \sqrt{\overline{\Delta}})^2}{2\log{p}} 
    \P[ f_p(Z_\nu(i^\dagger)) < 1 -\delta ] \to 1,
\end{equation}
for some small $\delta>0$.
On the one hand, we know (by solving a quadratic inequality) that
\begin{equation} \label{eq:chi-square-necessary-5}
    f_p(x)<1-\delta \iff \frac{x}{\sqrt{2\log{p}}} \in (-(\sqrt{\overline{r}}+\sqrt{1-\delta}), -(\sqrt{\overline{r}}-\sqrt{1-\delta})).
\end{equation}
On the other hand, we know (by relative stability of iid Gaussians \cite{gao2018fundamental}) that 
\begin{equation} \label{eq:chi-square-necessary-6}
    % f_p(\min_{i\in S}z_\nu(i)) =
    \frac{\min_{i\in S} Z_\nu(i)}{\sqrt{2\log{p}}}
    \to -\sqrt{1-\beta} \quad\text{in probability}.
\end{equation}
Further, by the assumption on the signal sizes $\overline{r} < (1+\sqrt{1-\beta})^2$, we have,
\begin{equation*}
    -(\sqrt{\overline{r}}+1) < -1 <- \sqrt{1-\beta} < - (\sqrt{\overline{r}}-1).
\end{equation*}
Therefore we can picking a small $\delta>0$ such that 
\begin{equation} \label{eq:chi-square-necessary-7}
    -(\sqrt{\overline{r}}+1) < -(\sqrt{\overline{r}}+\sqrt{1-\delta})
    < - \sqrt{1-\beta}
    < - (\sqrt{\overline{r}}-\sqrt{1-\delta})
    < - (\sqrt{\overline{r}}-1).
\end{equation}
Combining \eqref{eq:chi-square-necessary-5}, \eqref{eq:chi-square-necessary-6}, and \eqref{eq:chi-square-necessary-7}, we obtain
\begin{align*}
    \P\left[\min_{i\in S} f_p(Z_\nu(i)) < 1-\delta\right]
    &= \P\left[ f_p(Z_\nu(i^\dagger)) < 1-\delta \right] \\
    &\ge \P\left[ f_p\left(\min_{i\in S}Z_\nu(i)\right) < 1-\delta \right] \to 1,
\end{align*}
and we arrive at \eqref{eq:chi-square-necessary-4}.
As a corollary, since $u_p\sim2\log{p}$, it follows that
\begin{equation} \label{eq:chi-square-necessary-8}
    \P\left[\min_{i\in S}\frac{(Z_\nu(i) + \sqrt{\overline{\Delta}})^2}{u_p} < 1-\delta\right]\to1.
\end{equation}

Finally, by independence between $Z_1^2(i)+\ldots+Z_{\nu-1}^2(i)$ and $(Z_\nu^2(i)+\sqrt{\overline{\Delta}})^2$, and the fact that $i^\dagger$ is a function of only the latter, we have
$$
Z_1^2(i^\dagger)+\ldots+Z_{\nu-1}^2(i^\dagger) 
\stackrel{\mathrm{d}}{=} Z_1^2(i)+\ldots+Z_{\nu-1}^2(i) 
\quad \text{for all} \;\; i\in S.
$$
Therefore, $Z_1^2(i^\dagger)+\ldots+Z_{\nu-1}^2(i^\dagger) = O_\P(1)$, and 
\begin{equation} \label{eq:chi-square-necessary-9}
    \frac{Z_1^2(i^\dagger)+\ldots+Z_{\nu-1}^2(i^\dagger)}{u_p} \to 0 \quad \text{in probability}. 
\end{equation}
Together, \eqref{eq:chi-square-necessary-8} and \eqref{eq:chi-square-necessary-9} imply that
\begin{align}
    \P\left[\frac{m_S}{u_p}<1-\delta\right]
    &\ge \P\left[\min_{i\in S}\left\{\frac{Z_1^2(i) + \ldots + Z_{\nu-1}^2(i)}{u_p} + \frac{(Z_\nu(i) + \sqrt{\overline{\Delta}})^2}{u_p}\right\} < 1-\delta\right] \nonumber \\
    &\ge \P\left[\frac{Z_1^2(i^\dagger) + \ldots + Z_{\nu-1}^2(i^\dagger)}{u_p} + \frac{(Z_\nu(i^\dagger) + \sqrt{\overline{\Delta}})^2}{u_p} < 1-\delta\right] \to 1. \label{eq:chi-square-necessary-10}
\end{align}
In view of \eqref{eq:chi-square-necessary-0}, \eqref{eq:chi-square-necessary-1}, and \eqref{eq:chi-square-necessary-10}, we conclude that exact recovery cannot succeed with any positive probability.
The proof of the necessary condition is complete.
\end{proof}

\subsection{Proof of Theorem \ref{thm:chi-squared-approx-boundary}}
\label{subsec:proof-chi-squared-approx-boundary}

We first show the necessary condition. 
That is, when $\overline{r} < \beta$, no thresholding procedure is able to achieve approximate support recovery.

The proof follows the ideas in \cite{arias2017distribution}, and is very similar to the proof of Theorem \ref{thm:Gaussian-error-approx-boundary}. 
One could in principle obtain the proofs in this section by referencing arguments that have appeared in Section \ref{sec:proofs}.
We choose to present the the proof here in full to make this section self-contained.

\begin{proof}[Proof of necessary condition in Theorem \ref{thm:chi-squared-approx-boundary}]
Denote the distributions of $\chi^2_\nu(0)$, $\chi^2_\nu(\underline{\Delta})$ and $\chi^2_\nu(\overline{\Delta})$ as $F_0$, $F_{\underline{a}}$, and $F_{\overline{a}}$ respectively.

% We first show the necessary condition, i.e., when $\overline{r}<\beta$, approximate support recovery cannot be achieved with any thresholding procedure.
% In particular, we show that the liminf of the sum of FDP and NDP is at least 1.

Recall that thresholding procedures are of the form
$$
\widehat{S}_p = \left\{i\,|\,x(i) > t_p(x)\right\}.
$$
Denote $\widehat{S} := \left\{i\,|\,x(i) > t_p(x)\right\}$, and $\widehat{S}(u) := \left\{i\,|\,x(i) > u\right\}$.
For any threshold $u\ge t_p$ we must have $\widehat{S}(u)\subseteq\widehat{S}$, and hence
\begin{equation} \label{eq:approx-boundary-proof-FDP}
    \text{FDP} := \frac{|\widehat{S}\setminus{S}|}{|\widehat{S}|} \ge \frac{|\widehat{S}\setminus{S}|}{|\widehat{S}\cup{S}|} = \frac{|\widehat{S}\setminus{S}|}{|\widehat{S}\setminus{S}| + |S|} \ge
    \frac{|\widehat{S}(u)\setminus{S}|}{|\widehat{S}(u)\setminus{S}| + |S|}.
\end{equation}
On the other hand, for any threshold $u\le t_p$ we must have $\widehat{S}(u)\supseteq\widehat{S}$, and hence
\begin{equation} \label{eq:approx-boundary-proof-NDP}
    \text{NDP} := \frac{|{S}\setminus\widehat{S}|}{|{S}|} \ge 
    \frac{|{S}\setminus\widehat{S}(u)|}{|{S}|}.
\end{equation}
Since either $u\ge t_p$ or  $u\le t_p$ must take place, putting \eqref{eq:approx-boundary-proof-FDP} and \eqref{eq:approx-boundary-proof-NDP} together, we have
\begin{equation} \label{eq:approx-boundary-proof-converse-1}
    \text{FDP} + \text{NDP} 
    \ge \frac{|\widehat{S}(u)\setminus{S}|}{|\widehat{S}(u)\setminus{S}|+|{S}|} \wedge \frac{|{S}\setminus\widehat{S}(u)|}{|{S}|},
\end{equation}
for any $u$.
Therefore it suffices to show that for a suitable choice of $u$, the RHS of \eqref{eq:approx-boundary-proof-converse-1} converges to 1 in probability; the desired conclusion on FDR and FNR follows by the dominated convergence theorem.

Let $t^* = 2q\log{p}$ for some fixed $q$, we obtain an estimate of the tail probability
\begin{align}
    \overline{F_0}(t^*) 
    &= \P[\chi_\nu^2(0) > t^*] 
    = \frac{2^{-\nu/2}}{\Gamma(\nu/2)} \int_{2q\log{p}}^\infty x^{\nu/2-1}e^{-x/2} \mathrm{d}x \nonumber \\
    &\sim \frac{2^{-\nu/2}}{\Gamma(\nu/2)} 2\left(2q\log{p}\right)^{\nu/2-1}p^{-q}. \label{eq:approx-boundary-proof-null-tail-prob}
\end{align}
where $a_p\sim b_p$ is taken to mean $a_p/b_p\to 1$; this tail estimate was also obtained in \cite{donoho2004higher}.
Observe that $|\widehat{S}(t^*)\setminus{S}|$ has distribution $\text{Binom}(p-s, \overline{F_0}(t^*))$ where $s=|S|$, denote $X = X_p := {|\widehat{S}(t^*)\setminus{S}|}/{|S|}$, and we have 
$$
\mu := \E\left[X\right] = \frac{(p-s)\overline{F_0}(t^*)}{s},
\quad \text{and} \quad
\var\left(X\right) = \frac{(p-s)\overline{F_0}(t^*){F_0}(t^*)}{s^2} \le \mu/s.
$$
Therefore for any $M>0$, we have, by Chebyshev's inequality,
\begin{equation}
    \P\left[X < M\right] 
    \le \P\left[\left|X-\mu\right| > \mu - M\right]
    \le \frac{\mu/s}{(\mu-M)^2}
    = \frac{1/(\mu s)}{(1-M/\mu)^2}. \label{eq:approx-boundary-proof-converse-2}
\end{equation}
Now, from the expression of $\overline{F_0}(t^*)$ in \eqref{eq:approx-boundary-proof-null-tail-prob}, we obtain
$$
\mu = (p^\beta - 1)\overline{F_0}(t^*) \sim \frac{2^{1-\nu/2}}{\Gamma(\nu/2)} \left(2q\log{p}\right)^{\nu/2-1}p^{\beta-q}.
$$
Since $\overline{r}<\beta$, we can pick $q$ such that $\overline{r}<q<\beta$. 
In turn, we have $\mu \to\infty$, as $p\to\infty$.
Therefore the last expression in \eqref{eq:approx-boundary-proof-converse-2} converges to 0, and we conclude that $X\to\infty$ in probability, and hence
\begin{equation} \label{eq:approx-boundary-proof-converse-3}
\frac{|\widehat{S}(t^*)\setminus{S}|}{|\widehat{S}(t^*)\setminus{S}|+|{S}|} 
= \frac{X}{X+1} \to 1 \quad \text{in probability}.
\end{equation}

On the other hand, we show that with the same choice of $u = t^*$,
\begin{equation} \label{eq:approx-boundary-proof-converse-4}
    \frac{|{S}\setminus\widehat{S}(t^*)|}{|{S}|}\to 1 \quad \text{in probability}.
\end{equation}
By the stochastic monotonicity of chi-square distributions (Lemma \ref{lemma:stochastic-monotonicity}), the probability of missed detection for each signal is lower bounded by $\P[\chi^2_\nu(\lambda_i) \le t^*] \ge F_{\overline{a}}(t^*)$.
Therefore, $|{S}\setminus\widehat{S}(t^*)| \stackrel{\mathrm{d}}{\ge} \text{Binom}(s, {F_{\overline{a}}}(t^*))$, and it suffices to show that ${F_{\overline{a}}}(t^*)$ converges to 1.
This is indeed the case, since
\begin{align*}
    {F_{\overline{a}}}(t^*) 
    &= \P[Z_1^2 + \ldots + Z_\nu^2 + 2\sqrt{2\overline{r}\log{p}} Z_\nu + 2\overline{r}\log{p} \le 2q\log{p}] \\
    &\ge \P[Z_1^2 + \ldots + Z_\nu^2 \le (q-\overline{r})\log{p}, \; 2\sqrt{2\overline{r}\log{p}} Z_\nu \le (q-\overline{r})\log{p}],
\end{align*}
and both events in the last line have probability going to 1 as $p\to\infty$.
The necessary condition is shown.
\end{proof}

We now turn to the sufficient condition. 
That is, when $\underline{r} > \beta$, the Benjamini-Hochberg procedure with slowly vanishing FDR levels achieves asymptotic approximate support recovery.
The structure for the proof of sufficient condition follows that of Theorem 2 in \cite{arias2017distribution}. 

\begin{proof}[Proof of sufficient condition in Theorem \ref{thm:chi-squared-approx-boundary}]
The FDR vanishes by our choice of $\alpha$ and the FDR-controlling property of the BH procedure.
It only remains to show that FNR also vanishes.

To do so we compare the FNR under the alternative specified in Theorem \ref{thm:chi-squared-approx-boundary} to one with all of the signal sizes equal to $\underline{\Delta}$.
Let $x(i)$ be vectors of independent observations with $p-s$ nulls having $\chi^2_\nu(0)$ distributions, and $s$ signals having $\chi^2_\nu(\underline{\Delta})$ distributions.
By Lemma \ref{lemma:monotonicity-BH-procedure}, it suffices to show that the FNR under the BH procedure in this setting vanishes.

Let $\widehat{G}$ denote the empirical survival function as in \eqref{eq:empirical-tail-distribution}.
Define the empirical survival functions for the null part and signal part
\begin{equation} \label{eq:empirical-survival-null-signal}
    \widehat{W}_\text{null}(t) = \frac{1}{p-s}\sum_{i\not\in S}\mathbbm{1}\{x(i) \ge t\},
    \quad
    \widehat{W}_\text{signal}(t) = \frac{1}{s}\sum_{i\in S}\mathbbm{1}\{x(i) \ge t\},
\end{equation}
where $s=|S|$, so that
$$
\widehat{G}(t) = \frac{p-s}{p}\widehat{W}_\text{null}(t) + \frac{s}{p}\widehat{W}_\text{signal}(t).
$$


Apply Lemma \ref{lemma:empirical-process} to the two summands in $\widehat{G}$, we obtain
$\widehat{G}(t) = G(t) + \widehat{R}(t)$.
where 
\begin{equation} \label{eq:empirical-process-mean}
    G(t) = \frac{p-s}{p}\overline{F_0}(t) + \frac{s}{p}\overline{F_a}(t),
\end{equation}
where $\overline{F_0}$ and $\overline{F_{a}}$ are the survival functions of $\chi_\nu^2(0)$ and $\chi_\nu^2(\underline{\Delta})$ respectively, and 
\begin{equation} \label{eq:empirical-process-residual}
    \widehat{R}(t) = O_\P\left(\xi_p\sqrt{\overline{F_0}(t)F_0(t)} + \frac{s}{p}\xi_s\sqrt{\overline{F_a}(t)F_a(t)}\right),
\end{equation}
uniformly in $t$.

Recall (see proof of Lemma \ref{lemma:monotonicity-BH-procedure}) that the BH procedure is the thresholding procedure with threshold set at $\tau$ (defined in \eqref{eq:approx-boundary-proof-tau}).
% \begin{equation} \label{eq:approx-boundary-proof-tau-recall}
%     \tau = \inf\{t\,|\,\overline{F_0}(t)\le\alpha\widehat{G}(t)\}. 
%     %= \min\{t\,|\,\overline{F_0}(t)=\alpha\widehat{G}(t)\}.
% \end{equation}
The NDP may also be re-written as 
$$
\text{NDP} = \frac{|{S}\setminus\widehat{S}|}{|{S}|} = \frac{1}{s}\sum_{i\in S}\mathbbm{1}\{x(i) < \tau\} = 1 - \widehat{W}_\text{signal}(\tau),
$$
so that it suffices to show that 
\begin{equation} \label{eq:approx-boundary-proof-sufficient-1}
    \widehat{W}_\text{signal}(\tau)\to 1
\end{equation} in probability.
Applying Lemma \ref{lemma:empirical-process} to $\widehat{W}_\text{signal}$, we know that 
$$
\widehat{W}_\text{signal}(\tau) = \overline{F_a}(\tau) + O_\P\left(\xi_s\sqrt{\overline{F_a}(\tau)F_a(\tau)}\right) = \overline{F_a}(\tau) + o_\P(1).
$$
So it suffices to show that $F_a(\tau)\to 0$ in probability.
Now let $t^* = 2q\log(p)$ for some $q$ such that $\beta<q<\underline{r}$.
We have 
\begin{align}
    F_a(t^*) 
    &= \P[\chi^2_\nu(\underline{\Delta}) \le t^*]
    \le \P\left[2\sqrt{\underline{\Delta}}Z_\nu \le t^* - \underline{\Delta}\right] \nonumber \\
    &= \P\left[Z_\nu \le \frac{t^*}{2\sqrt{\underline{\Delta}}} - \frac{\sqrt{\underline{\Delta}}}{2}\right] 
    = \P\left[Z_\nu \le \frac{q-\underline{r}}{2\sqrt{\underline{r}}}\sqrt{2\log{p}}\right] \to 0. \label{eq:approx-boundary-proof-sufficient-2}
\end{align} 
Hence in order to show \eqref{eq:approx-boundary-proof-sufficient-1}, it suffices to show 
\begin{equation} \label{eq:approx-boundary-proof-sufficient-3}
    \P\left[\tau \le t^*\right] \to 1.
\end{equation}
By \eqref{eq:empirical-process-mean}, the mean of the empirical process $\widehat{G}$ evaluated at $t^*$ is
\begin{equation} \label{eq:approx-boundary-proof-sufficient-4}
    G(t^*) = \frac{p-s}{p}\overline{F_0}(t^*) + \frac{s}{p}\overline{F_a}(t^*).
\end{equation}
The first term, using Relation \eqref{eq:approx-boundary-proof-null-tail-prob}, is asymptotic to $p^{-q}L(p)$, where $L(p)$ is the logarithmic term in $p$.
The second term, since $\overline{F_a}(t^*)\to 1$ by Relation \eqref{eq:approx-boundary-proof-sufficient-2}, is asymptotic to $p^{-\beta}$.
Therefore, $G(t^*) \sim p^{-q}L(p) + p^{-\beta} \sim p^{-\beta}$, since 
$p^{\beta-q}L(p)\to0$ where $q>\beta$.

The fluctuation of the empirical process at $t^*$, by Relation \eqref{eq:empirical-process-residual}, is 
\begin{align*}
    \widehat{R}(t^*) 
    &= O_\P\left(\xi_p\sqrt{\overline{F_0}(t^*)F_0(t^*)} + \frac{s}{p}\xi_s\sqrt{\overline{F_a}(t^*)F_a(t^*)}\right)\\
    &= O_\P\left(\xi_p\sqrt{\overline{F_0}(t^*)}\right) + o_\P\left(p^{-\beta}\right).
\end{align*}
By \eqref{eq:approx-boundary-proof-null-tail-prob} and the expression for $\xi_p$, the first term is $O_\P\left(p^{-(q+1)/2}L(p)\right)$ where $L(p)$ is a poly-logarithmic term in $p$.
Since $\beta<\min\{q,1\}$, we have $\beta<(q+1)/2$, and hence $\widehat{R}(t^*) = o_\P(p^{-\beta})$.

Putting the mean and the fluctuation of $\widehat{G}(t^*)$ together, we obtain
$$
\widehat{G}(t^*) = G(t^*) + \widehat{R}(t^*) \sim_\P G(t^*) \sim p^{-\beta},
$$
and therefore, together with \eqref{eq:approx-boundary-proof-null-tail-prob}, we have
$$
\overline{F_0}(t^*)/\widehat{G}(t^*) = p^{\beta-q}L(p)(1+o_{\P}(1)),
$$
which is eventually smaller than the FDR level $\alpha$ by the assumption \eqref{eq:slowly-vanishing-error} and the fact that $\beta<q$.
That is, 
$$
\P\left[\overline{F}_0(t^*) / \widehat{G}(t^*) < \alpha\right] \to 1.
$$
By definition of $\tau$ (recall \eqref{eq:approx-boundary-proof-tau}), this implies that $\tau \le t^*$ with probability tending to 1, and \eqref{eq:approx-boundary-proof-sufficient-3} is shown.
The proof for the sufficient condition is complete.
\end{proof}

\subsection{Proof of Theorems \ref{thm:chi-squared-exact-approx-boundary} and \ref{thm:chi-squared-approx-exact-boundary}}
\label{subsec:proof-chi-squared-mix-boundaries}

As with the proof of Theorem \ref{thm:chi-squared-approx-boundary}, one could shorten the presentations in this section by referencing arguments Section \ref{sec:proofs}.
Again, we choose to present the the proof in full to make this section self-contained.

\begin{proof}[Proof of Theorem \ref{thm:chi-squared-exact-approx-boundary}]
% The reduction to equal signal sizes can be achieved 
We first show the sufficient condition.
Similar to the proof of Theorem \ref{thm:chi-squared-approx-boundary}, it suffices to show that
\begin{equation} \label{eq:exact-approx-boundary-proof-sufficient-1}
    \text{NDP} = 1 - \widehat{W}_\text{signal}(t_p) \to 0,
\end{equation}
where $t_p$ is the threshold of Bonferroni's procedure.

Since $\underline{r}>\widetilde{g}(\beta)=1$, we can pick $q$ such that $1<q<\underline{r}$.
Let $t^* = 2q\log{p}$, we have $t_p<t_p^*$ for large $p$ as in the proof of Theorem \ref{thm:chi-squared-exact-boundary}.
Therefore for large $p$, we have
$$
\widehat{W}_\text{signal}(t_p) \ge \widehat{W}_\text{signal}(t^*) \ge \overline{F_a}(t^*) + o_\P(1),
$$
where the last inequality follows from the stochastic monotonicity of the chi-square family (Lemma \ref{lemma:stochastic-monotonicity}), and Lemma \ref{lemma:empirical-process}.
Indeed, $F_a(t^*)\to0$ by \eqref{eq:approx-boundary-proof-sufficient-2} and our choice of $q<\underline{r}$. 
The proof of the sufficient condition is complete.

Proof of the necessary condition follows a similar structure to that of Theorem \ref{thm:chi-squared-approx-boundary}.
That is, we show that $\mathrm{FWER} + \mathrm{FNR}$ has liminf at least 1 by working with the lower bound
\begin{equation} \label{eq:exact-approx-boundary-proof-necessary-1}
    \mathrm{FWER}(\mathcal{R}) + \mathrm{FNR}(\mathcal{R}) \ge \P\left[\max_{i\in S^c}x(i)>u\right] \wedge \E\left[\frac{|S\setminus \widehat{S}(u)|}{|S|}\right],
\end{equation}
which holds for any thresholding procedure $\mathcal{R}$ and for arbitrary $u\in\R$.
By the assumption that $\overline{r}<\widetilde{g}(\beta)=1$, we can pick $q$ such that $\overline{r}<q<1$ and let $u = t^*=2q\log{p}$.
By relative stability of chi-squared random variables (Lemma \ref{lemma:rapid-variation-chisq}), we have
\begin{equation} \label{eq:exact-approx-boundary-proof-necessary-2}
    \P\left[\frac{\max_{i\in S^c} x(i)}{2\log{p}} > \frac{t^*}{2\log{p}}\right] \to 1.
\end{equation}
where the first fraction in \eqref{eq:exact-approx-boundary-proof-necessary-2} converges to 1, while the second converges to $q<1$.
On the other hand, by our choice of $q>\overline{r}$, the second term in \eqref{eq:exact-approx-boundary-proof-necessary-1} also converges to 1 as in \eqref{eq:approx-boundary-proof-converse-4}.
This completes the proof of the necessary condition.
\end{proof}


\begin{proof}[Proof of Theorem \ref{thm:chi-squared-approx-exact-boundary}]
We first show the sufficient condition.
Since FDR control is guaranteed by the BH procedure, we only need to show that the FWNR also vanishes, that is,
\begin{equation} \label{eq:approx-exact-boundary-proof-sufficient-1}
    \P\left[\min_{i\in S}x(i) \ge \tau\right] \to 1,
\end{equation}
where $\tau$ is the threshold for the BH procedure.

By the assumption that $\underline{r}>\widetilde{h}(\beta)=(\sqrt{\beta}+\sqrt{1-\beta})^2$, we have $\sqrt{\underline{r}}-\sqrt{1-\beta}>\sqrt{\beta}$, so we can pick $q>0$, such that 
\begin{equation} \label{eq:approx-exact-boundary-proof-sufficient-2}
\sqrt{\underline{r}}-\sqrt{1-\beta}>\sqrt{q}>\sqrt{\beta}.
\end{equation}
Let $t^*=2q\log{p}$, we claim that 
\begin{equation} \label{eq:approx-exact-boundary-proof-sufficient-3}
\P\left[\tau\le t^*\right]\to 1.
\end{equation}
Indeed, by our choice of $q>\beta$, \eqref{eq:approx-exact-boundary-proof-sufficient-3} follows in the same way that \eqref{eq:approx-boundary-proof-sufficient-3} did.

With this $t^*$, we have
\begin{equation} \label{eq:approx-exact-boundary-proof-sufficient-4}
    \P\left[\min_{i\in S}x(i) \ge \tau\right] \ge 
    % \P\left[\min_{i\in S}x(i) \ge t^* \ge \tau\right] \ge
    \P\left[\min_{i\in S}x(i) \ge t^*,\; t^* \ge \tau\right].
\end{equation}
However, by our choice of $\sqrt{q} < \sqrt{\underline{r}}-\sqrt{1-\beta}$, the probability of the first event on the right-hand side of \eqref{eq:approx-exact-boundary-proof-sufficient-4} also goes to 1 according to \eqref{eq:chi-square-sufficient-1} and \eqref{eq:chi-square-sufficient-2}.
Together with \eqref{eq:approx-exact-boundary-proof-sufficient-3}, this proves \eqref{eq:approx-exact-boundary-proof-sufficient-1}, and completes proof of the sufficient condition.

The necessary condition follows from the lower bound
\begin{equation} \label{eq:approx-exact-boundary-proof-necessary-1}
    \mathrm{FDR}(\mathcal{R}) + \mathrm{FWNR}(\mathcal{R}) \ge \E\left[\frac{|\widehat{S}(u)\setminus S|}{|\widehat{S}(u)\setminus S| + |S|}\right] \wedge 
    \P\left[\min_{i\in S}x(i)<u\right],
\end{equation}
which holds for any thresholding procedure $\mathcal{R}$ and for arbitrary $u\in\R$.

By the assumption that $\overline{r}<\widetilde{h}(\beta)=(\sqrt{\beta}+\sqrt{1-\beta})^2$, we can pick a constant $q>0$, such that 
\begin{equation} \label{eq:approx-exact-boundary-proof-necessary-2}
    \sqrt{\overline{r}} - \sqrt{1-\beta} < \sqrt{q} < \sqrt{\beta}.
\end{equation}
Let also $u = t^*=2q\log{p}$.
By our choice of $q < \beta$, we know from \eqref{eq:approx-boundary-proof-converse-3} that the first term on the right-hand-side of \eqref{eq:approx-exact-boundary-proof-necessary-1} converges to 1.
It remains to show that the second term in \eqref{eq:approx-exact-boundary-proof-necessary-1} also converges to 1.

For the second term in \eqref{eq:approx-exact-boundary-proof-necessary-1}, dividing through by $2\log{p}$, we obtain
\begin{equation} \label{eq:approx-exact-boundary-proof-necessary-3}
    \P\left[\min_{i\in S}x(i)<t^*\right] = \P\left[ \frac{m_{S}}{2\log{p}} < q \right].
\end{equation}
Similar to \eqref{eq:chi-square-necessary-3}, we have
\begin{equation} \label{eq:approx-exact-boundary-proof-necessary-4}
    \frac{m_{S}}{2\log{p}} 
    \stackrel{\mathrm{d}}{\le} \min_{i\in S}\frac{Z_1^2(i) + \ldots + Z_{\nu-1}^2(i)}{2\log{p}} + \frac{(Z_\nu(i) + \sqrt{\overline{\Delta}})^2}{2\log{p}}.
\end{equation}
Define $i^\dagger = i^\dagger_p$ to be the index minimizing the second term in \eqref{eq:approx-exact-boundary-proof-necessary-4}, i.e.,
\begin{equation}
    i^\dagger := \argmin_{i\in S} f_p\left(Z_\nu(i)\right),
    % \frac{(Z_\nu(i) + \sqrt{\overline{\Delta}})^2}{2\log{p}},
\end{equation}
where $f_p(x):=(x+\sqrt{\overline{\Delta}})^2/(2\log{p})$. 

Since $\sqrt{q}>\sqrt{\overline{r}}-\sqrt{1-\beta}$ and $q>0$, we have
$\frac{\sqrt{\overline{r}}-\sqrt{q}}{\sqrt{1-\beta}}<1$.
Also, since
$$
    \frac{\sqrt{\overline{r}}+\sqrt{q}}{\sqrt{1-\beta}}>0,
    \quad\text{and}\quad
    \frac{\sqrt{\overline{r}}-\sqrt{q}}{\sqrt{1-\beta}} < \frac{\sqrt{\overline{r}}+\sqrt{q}}{\sqrt{1-\beta}},
$$
we can further pick a constant $\beta_0\in(0,1]$ such that
\begin{equation} \label{eq:approx-exact-boundary-proof-necessary-5}
    \frac{\sqrt{\overline{r}}-\sqrt{q}}{\sqrt{1-\beta}} 
    < \sqrt{\beta_0} < 
    \frac{\sqrt{\overline{r}}+\sqrt{q}}{\sqrt{1-\beta}}.
\end{equation}
Let $Z_{[1]}\le Z_{[2]}\le\ldots\le Z_{[s]}$ be the order statistics of 
$\{Z_\nu(i)\}_{i\in S}$ and define $k=\lfloor s^{1-\beta_0}\rfloor$.
Applying Lemma \ref{lemma:relative-stability-order-statistics} (stated below), we obtain
% \begin{equation*} 
% Z_{[k]} \sim -\sqrt{2\beta_0\log{s}} \sim -\sqrt{2\beta_0(1-\beta)\log{p}}.
% % , \quad\text{as}\;\;p\to\infty.
% \end{equation*}
% Therefore, we have,
\begin{equation} \label{eq:approx-exact-boundary-proof-necessary-6}
    % f_p(\min_{i\in S}z_\nu(i)) =
    \frac{Z_{[k]}}{\sqrt{2\log{p}}}
    = \frac{Z_{[k]}}{\sqrt{2\log{s}}} \frac{\sqrt{2\log{s}}}{\sqrt{2\log{p}}} 
    \to -\sqrt{\beta_0(1-\beta)}
    % \left(-(\sqrt{\overline{r}}+\sqrt{q}), -(\sqrt{\overline{r}}-\sqrt{q})\right) 
    \quad\text{in probability}.
\end{equation}
Since we know (by solving a quadratic inequality) that
\begin{equation} \label{eq:approx-exact-boundary-proof-necessary-7}
    f_p(x)<q \iff \frac{x}{\sqrt{2\log{p}}} \in \left(-(\sqrt{\overline{r}}+\sqrt{q}), -(\sqrt{\overline{r}}-\sqrt{q})\right),
\end{equation}
combining \eqref{eq:approx-exact-boundary-proof-necessary-5}, \eqref{eq:approx-exact-boundary-proof-necessary-6}, and 
\eqref{eq:approx-exact-boundary-proof-necessary-7}, it follows that
\begin{equation*}
    %\min_{i\in S} \frac{(Z_\nu(i) + \sqrt{\overline{\Delta}})^2}{2\log{p}} 
    \P\left[ f_p\left(Z_\nu(i^\dagger)\right) < q \right] \ge \P\left[ f_p\left(Z_{[k]}\right) < q \right] \to 1.
\end{equation*}
Finally, using \eqref{eq:chi-square-necessary-9}, we conclude that 
$$
\P\left[\min_{i\in S}x(i)<t^*\right] = 
\P\left[\frac{m_{S}}{2\log{p}} < q\right] \ge
\P\left[o_\P(1) + f_p\left(Z_\nu(i^\dagger)\right) < q \right] \to 1.
$$
Therefore, the two terms on the right-hand-side of \eqref{eq:approx-exact-boundary-proof-necessary-1} both converge 1. 
This completes the proof of the necessary condition.
\end{proof}

It only remains to justify \eqref{eq:approx-exact-boundary-proof-necessary-6}.

\begin{lemma}[Relative stability of order statistics]
\label{lemma:relative-stability-order-statistics}
Let $Z_{[1]} \le \ldots \le Z_{[s]}$ be the order statistics of $s$ iid standard Gaussian random variables.
Let $\beta_0\in(0,1]$ and define $k=\lfloor s^{1-\beta_0}\rfloor$, then we have
\begin{equation}
    \frac{Z_{[k]}}{\sqrt{2\log{s}}} \to -\sqrt{\beta_0} \quad\text{in probability}.
\end{equation}
\end{lemma}

\begin{proof}[Proof of Lemma \ref{lemma:relative-stability-order-statistics}]
Using the Renyi representation for order statistics, we write
\begin{equation} \label{eq:relative-stability-order-statistics-proof-1}
    Z_{[i]} = \Phi^{\leftarrow}(U_{[i]}),
\end{equation}
where $U_{[i]}$ is the $i^\mathrm{th}$ (smallest) order statistic of $s$ independent uniform random variables over $(0,1)$.
Since $U_{[i]}$ has a $\mathrm{Beta}(i, s+1-i)$ distribution, with mean and standard deviation,
$$
\E[U_{[k]}] = k/(s+1) \sim s^{-\beta_0}, 
\quad \text{and} \quad
{\mathrm{sd}(U_{[k]})} = \frac{1}{s+1}\sqrt{\frac{k(s+1-k)}{s+2}} \sim s^{-\frac{1+\beta_0}{2}},
$$
we obtain by Chebyshev's inequality 
% (since $\E[U_{[k]}]/\mathrm{sd}(U_{[k]})\sim p^{(1-\beta_0)/2}$) 
$$
\P\left[s^{-\beta_0}(1-\epsilon) < U_{[k]} < s^{-\beta_0}(1+\epsilon)\right] \to 1,
$$
where $\epsilon$ is an arbitrary positive constant.
This implies, by representation \eqref{eq:relative-stability-order-statistics-proof-1},
\begin{equation} \label{eq:relative-stability-order-statistics-proof-2}
    \P\left[\Phi^{\leftarrow}\left(s^{-\beta_0}(1-\epsilon)\right) < Z_{[k]} < \Phi^{\leftarrow}\left(s^{-\beta_0}(1+\epsilon)\right)\right] \to 1.
\end{equation}
Using the expression for standard Gaussian quantiles (see, e.g., Proposition 1.1. in \cite{gao2018fundamental}), we know that
\begin{align*}
    \Phi^{\leftarrow}\left(s^{-\beta_0}(1-\epsilon)\right) 
    &\sim -\sqrt{2\log{\left(s^{\beta_0}/(1-\epsilon)\right)}} \\
    &= -\sqrt{2(\beta_0\log{s} - \log{(1-\epsilon)})} \sim -\sqrt{2\beta_0\log{s}},
\end{align*}
and similarly $\Phi^{\leftarrow}\left(s^{-\beta_0}(1+\epsilon)\right)\sim -\sqrt{2\beta_0\log{s}}$.
Since both ends of the interval in \eqref{eq:relative-stability-order-statistics-proof-2} are asymptotic to $-\sqrt{2\beta_0\log{s}}$, 
% dividing through by $\sqrt{2\log{s}}$, and 
the desired conclusion follows.
\end{proof}


\subsection{Proof of Proposition \ref{prop:signal-size-odds-ratio} and Corollary \ref{cor:signal-limits-OR}}
\label{subsec:proof-signal-size-odds-ratio}


\begin{proof}[Proof of Proposition \ref{prop:signal-size-odds-ratio} and Corollary \ref{cor:signal-limits-OR}]
We parametrize the 2-by-2 multinomial distribution with the parameter $\delta$, 
\begin{equation} \label{eq:reparametrize-2-by-2-table-1}
    \mu_{11} = \phi_1\theta_1+\delta,\quad 
    \mu_{12} = \phi_1\theta_2-\delta,\quad 
    \mu_{21} = \phi_2\theta_1-\delta,\quad 
    \mu_{22} = \phi_2\theta_2+\delta.
\end{equation}
By relabelling of categories, we may assume $0<\theta_1,\phi_1\le1/2$ without loss of generality.
Note that $\delta$ must lie within the range $[\delta_\mathrm{min}, \delta_\mathrm{max}]$, where
$$
\delta_\mathrm{min} := \max\{-\phi_1\theta_1, -\phi_2\theta_2, \phi_1\theta_2-1, \phi_2\theta_1-1\} 
= -\phi_1\theta_1,
$$
and
$$
\delta_\mathrm{max} := \min\{1-\phi_1\theta_1, 1-\phi_2\theta_2, \phi_1\theta_2, \phi_2\theta_1\}
= \min\{\phi_1\theta_2, \phi_2\theta_1\},
$$
in order for $\mu_{ij}\ge0$ for all $i,j\in \{1,2\}$.
Under this parametrization, Relation \eqref{eq:odds-ratio} then becomes
\begin{equation} \label{eq:odds-ratio-delta}
    \text{R} = \frac{\mu_{11}\mu_{22}}{\mu_{12}\mu_{21}}
    = \frac{\phi_1\theta_1\phi_2\theta_2 + \delta(\phi_1\theta_1+\phi_2\theta_2)+\delta^2}{\phi_1\theta_1\phi_2\theta_2 - \delta(\phi_1\theta_2+\phi_2\theta_1)+\delta^2},
\end{equation}
which is one-to-one and increasing in $\delta$ on $(\delta_\mathrm{min}, \delta_\mathrm{max})$.
Equation \eqref{eq:signal-size-chisq} becomes
\begin{equation} \label{eq:signal-size-chisq-delta}
w^2 = \sum_{i=1}^2 \sum_{j=1}^2 \frac{(\mu_{ij} - \phi_i\theta_j)^2}{\phi_i\theta_j}
= \delta^2\sum_i\sum_j \frac{1}{\phi_i\theta_j}
= \frac{\delta^2}{\phi_1\theta_1\phi_2\theta_2},
\end{equation}
Solving for $\delta$ in \eqref{eq:odds-ratio-delta}, and plugging into the expression for signal size \eqref{eq:signal-size-chisq-delta} yields Relation \eqref{eq:signal-size-odds-ratio}.
Corollary \ref{cor:signal-limits-OR} follows from the fact that $w^2(\delta)$ is decreasing on $[\delta_\mathrm{min},0)$, increasing on $(0,\delta_\mathrm{max}]$, with limits
$$
\lim_{d\to \delta_\mathrm{min}} w^2(\delta) = \frac{\phi_1\theta_1}{\phi_2\theta_2},
\quad
\text{and}
\quad
\lim_{d\to \delta_\mathrm{max}} w^2(\delta) = \min\left\{\frac{\phi_1\theta_2}{\phi_2\theta_1}, \frac{\phi_2\theta_1}{\phi_1\theta_2}\right\}.
$$
The other three cases ($1/2\le\theta_1,\phi_1\le1$; $0<\theta_1\le1/2\le\phi_1\le1$; and $0\le\phi_1\le1/2\le\theta_1\le1$) may be obtained similarly, or by appealing to the symmetry of the problem.
\end{proof}

\begin{proof}[Proof of Corollary \ref{cor:optimal-design}]
Using the parametrization in \eqref{eq:reparametrize-2-by-2-table-1} and in Corollary \ref{cor:signal-size-odds-ratio-conditional-frequency}, we solve for $\delta$ in \eqref{eq:odds-ratio-delta} to obtain
\begin{align}
    \delta &= \frac{\phi_1 fR}{fR+1-f} - \left(\frac{\phi_1 fR}{fR+1-f} + f(1-\phi_1)\right)\phi_1 \nonumber \\
    &= \frac{f(1-f)\phi_1(1-\phi_1)(R-1)}{fR+1-f}. \label{eq:reparametrize-2-by-2-table-2}
\end{align}
Substituting \eqref{eq:reparametrize-2-by-2-table-2} into the expression \eqref{eq:signal-size-chisq-delta}, after some simplification, yields
\begin{equation} \label{eq:reparametrize-2-by-2-table-3}
    w^2 = \frac{f(1-f)\phi_1(1-\phi_1)(R-1)^2}{\left[\phi_1 R + (1-\phi_1)D\right]\left[\phi_1 + (1-\phi_1)D\right]},
\end{equation}
where $D = fR+1-f > 0$.
Therefore, he derivative of \eqref{eq:reparametrize-2-by-2-table-3} with respect to $\phi_1$ is
\begin{equation} \label{eq:signal-size-first-derivative}
    \frac{\mathrm{d}w^2}{\mathrm{d}\phi_1} = 
    \frac{f(1-f)(R-1)^2}{\left[\phi_1 R+(1-\phi_1)D\right]^2 \left[\phi_1+(1-\phi_1)D\right]^2} \left[(D^2-R)\phi_1^2 - 2D^2\phi_1 + D^2\right].
\end{equation}
Further, we obtain the second derivative with respect to $\phi_1$,
\begin{equation} \label{eq:signal-size-second-derivative}
    \frac{\mathrm{d}^2w^2}{\mathrm{d}\phi_1^2} = 
    h(R,f) \left[(\phi_1-1)D^2 - \phi_1R\right],
\end{equation}
where $h$ is some function of $(R,f)$ taking on strictly positive values.

Since $\left[(\phi_1-1)D^2 - \phi_1R\right]<0$, the second derivative \eqref{eq:signal-size-second-derivative} must be strictly negative on $[0,1]$.
This implies that the first derivative \eqref{eq:signal-size-first-derivative} is strictly decreasing on $[0,1]$. 
Since the first derivative \eqref{eq:signal-size-first-derivative} is strictly positive at $\phi_1=0$, and strictly negative at $\phi_1=1$, it must have a unique zero between 0 and 1, and hence, the solution to $(D^2-R)\phi_1^2 - 2D^2\phi_1 + D^2 = 0$ in the interval of $[0,1]$ must be the maximizer of \eqref{eq:reparametrize-2-by-2-table-3} --- when $D^2-R>0$, the smaller of the two roots maximizes \eqref{eq:reparametrize-2-by-2-table-3}, and when $D^2-R<0$, it is the larger of the two.
They share the same expression ${D}/{(D+\sqrt{R})}$, which coincides with \eqref{eq:optimal-design}.
Finally, when $D^2=R$, the only root $\phi_1^*=1/2$, which also coincides with \eqref{eq:optimal-design}, is the maximizer of \eqref{eq:reparametrize-2-by-2-table-3}.
\end{proof}

