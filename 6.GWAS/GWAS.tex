
The process of scientific discovery, as famously explained by Richard Faynman, usually starts with guesses in one form or another.
The consequences of such guesses are then computed, and compared with experimental results.
If the predictions disagree with experiment, then our guesses are wrong. ``That is all there is to it''  \citep{feynman2017character}.

In previous chapters, we delved deep in to the theoretical underpinnings of the phase transition phenomena in high-dimensional multiple testing problems.
The results are interesting in their own right.
However, in the spirit of Faynman, we have not discovered any scientific law, but merely worked out mathematical consequences of our postulated models.
In this chapter, our goal is to relate these predictions to real experimental data from the field of genetics, where large-scale simultaneous hypotheses testing problems often arise.
From such comparisons, we will demonstrate that the phase transition laws are indeed accurate predictions of some curious phenomena in the real world.
The accuracy of our predictions will lend credibility to the application of these ``laws of large dimensions'' in actual statistical problems.

\medskip

In our case, the experimental data used as the measuring stick come from genome-wide association studies (\ac{GWAS}), introduced in Section \ref{sec:motivation-chisq}.
Recall that in \ac{GWAS}, a large number of marginal association tests are conducted simultaneously, resulting in statistics that can be approximated by
\begin{equation} \label{eq:model-chisq-Chapter6}
    %x(i) \distras{\mathrm{ind.}} \chi_\nu^2\left(\lambda(i)\right), \quad i=1,\ldots,p.
    x(i) \sim \chi_\nu^2\left(\lambda(i)\right), \quad i=1,\ldots,p,
\end{equation}
where $\chi_\nu^2\left(\lambda(i)\right)$ is a chi-square distributed random variable with $\nu$ degrees of freedom and non-centrality parameter $\lambda(i)$.

We establish our theoretical predictions in two steps.
In Section \ref{sec:chisq-boundaries} below, we shall first establish the phase transitions of the model \eqref{eq:model-chisq-Chapter6}.
In parallel to results in Chapter \ref{chap:phase-transitions}, we show that several commonly used family-wise error rate-control procedures --- including Bonferroni's procedure --- are asymptotically optimal for the {exact}, and {exact-approximate} support recovery problems (recall Definition \ref{def:exact-recovery-success-failure}) in idealized chi-square models with independent components.
Analogously, the \ac{BH} procedure is asymptotically optimal for the {approximate}, and {approximate-exact} support recovery problems.
% \ref{thm:chi-squared-exact-boundary}, \ref{thm:chi-squared-exact-approx-boundary}, \ref{thm:chi-squared-approx-boundary}, and \ref{thm:chi-squared-approx-exact-boundary}.
Under appropriate parametrizations of the signal sizes and sparsity, they establish the phase transitions of support recovery problems in the chi-square model.
Remarkably, the degree-of-freedom parameter does not affect the asymptotic boundaries in any of the four support recovery problems.

We then translate the canonical signal size and sparsity parametrizations into the vernacular of statistical geneticists in Section \ref{sec:odds-and-power}.
% present empirical evidence for the phase transition in the exact-approximate problem using real data from large-scale association studies on breast cancer obtained from the NHGRI-EBI GWAS Catalog \citep{macarthur2016new}.
% demystify the notion of signal size $\lambda$ in this context.
% which is perhaps less transparent than in additive error models.
We do so by characterizing the relationship between the signal size $\lambda$ and the marginal frequencies, odds ratio, and sample sizes for association tests on 2-by-2 contingency tables. 
These parameters are more often estimated and reported in \ac{GWAS} than the elusive signal size parameter $\lambda$.
% Specifically, the amount of signal, when rare variants are present, is weaker compared to the signal when the marginal distributions are balanced.
% In other words, reliable detection of the effects by rare variants would require more samples compared to common variants, even at the same odds ratio.
% This result, establishing the relationship between sample sizes and signal sizes, is made precise in Section \ref{sec:odds-and-power}.
As a bonus, we point out an implications of this relationship on optimal study designs for association studies in Section \ref{sec:optimal-design}.
Perhaps surprisingly, our analysis reveals that balanced designs with equal number of cases and controls are often statistically inefficient.
% Practical consequences of these results in power analysis will be illustrated with data examples in Section \ref{sec:phase-transitions-in-GWAS}. 

Armed with the results on phase transitions in the chi-square model, and a translation from the language of high-dimenstional statistics to the patois of association screening studies, we finally present in Section \ref{sec:phase-transitions-in-GWAS} the consequences of the phase transitions in \ac{GWAS}, and compare against real experimental data to evaluate the success of our predictions.

The phase transitions in the chi-square models are demonstrated with numerical simulations in Section \ref{sec:numerical}.
Proofs are collected in the online Supplement.
%in Section \ref{sec:proof-signal-size-odds-ratio}.

% Practical issues are also addressed to make for simple and effective power analysis.

\section{Support recovery problems in chi-squared models}
\label{sec:chisq-boundaries}

Similar to the analysis of additive error models in Chapter \ref{chap:phase-transitions}, we will work with triangular arrays of chi-square models \eqref{eq:model-chisq} indexed by $p$.
We adopt the same parametrization for the sparsity of the non-centrality parameter vectors $\lambda = \lambda_p$,
\begin{equation} \label{eq:signal-sparsity}
    |S_p| = \left\lfloor p^{1-\beta} \right\rfloor, \quad \beta\in(0,1]
\end{equation}
where $\beta$ parametrizes the problem sparsity.
The closer $\beta$ is to 1, the sparser the support $S_p$; conversely, when $\beta$ is close to 0, the support is dense with many non-null signals.

We parametrize the range of the non-zero and perhaps unequal signals in the chi-square model with
\begin{equation} \label{eq:signal-size}
    \underline{\Delta} = 2\underline{r}\log{p}
    \le \lambda(i) \le
    \overline{\Delta} = 2\overline{r}\log{p}, \quad \text{for all}\;\;i\in S_p,
\end{equation}
for some constants $0<\underline{r}\le\overline{r}\le+\infty$.

\subsection{The exact support recovery problem}
\label{subsec:exact-support-recovery-chisq}

The first main result characterizes the phase transition phenomenon in the exact support recovery problem under the chi-square model.

\begin{theorem} \label{thm:chi-squared-exact-boundary}
Consider the high-dimensional chi-squared model \eqref{eq:model-chisq} with signal sparsity and size as described in \eqref{eq:signal-sparsity} and \eqref{eq:signal-size}.
The function 
\begin{equation} \label{eq:exact-boundary-chisquared}
    g(\beta) = \left(1 + \sqrt{1-\beta}\right)^2
\end{equation}
characterizes the phase transition of exact support recovery problem.
Specifically, if $\underline{r} > {{g}}(\beta)$, then Bonferroni's, Sid\'ak's, Holm's, and Hochberg's procedures with slowly vanishing (see Definition \ref{def:slowly-vanishing}) nominal FWER levels all achieve asymptotically exact support recovery in the sense of \eqref{eq:support-recovery-success}. 

Conversely, if $\overline{r} < {{g}}(\beta)$, then for any thresholding procedure $\widehat{S}_p$, we have $\P[\widehat{S}_p=S_p]\to0$.
Therefore, in view of Lemma \ref{lemma:risk-exact-recovery-probability}, exact support recovery asymptotically fails for all thresholding procedures in the sense of \eqref{eq:support-recovery-failure}.
\end{theorem}

% \begin{figure}
%   \begin{center}
%     \includegraphics[width=0.7\textwidth]{./pics/phase_diagram_chisquared_ALL_boundaries.eps}
%   \end{center}
%    \caption{The phase diagram for the high-dimensional chi-square model \eqref{eq:model-chisq}, illustrating the boundaries of the exact support recovery (FWER + FWNR; top curve; Theorem \ref{thm:chi-squared-exact-boundary}), the approximate-exact support recovery (FDR + FWNR; second curve from top; Theorem \ref{thm:chi-squared-approx-exact-boundary}), the exact-approximate support recovery (FWER + FNR; horizontal line $r=1$; Theorem \ref{thm:chi-squared-exact-approx-boundary}), and the approximate support recovery problems (FDR + FNR; tilted line $r=\beta$; Theorem \ref{thm:chi-squared-approx-boundary}). The signal detection problem (type I + type II errors of the global test; lower curve) was studied in Donoho and Jin (2004). In each region of the diagram and above, the annotated statistical risk can be made to vanish, as dimension $p$ diverges. Conversely, the risks has liminf at least one. All boundaries are unaffected by the degree-of-freedom. All boundaries are identical to those in the Gaussian additive error model \eqref{eq:model-additive} under one-side alternatives; c.f., results in Section \ref{sec:additive-error-model-boundaries}.}
%    \label{fig:phase-chi-squared}
% \end{figure}


The procedures listed in Theorem \ref{thm:chi-squared-exact-boundary} were reviewed in Section \ref{sec:statistical-procedures}. 
Proof of the theorem can be found in Section \ref{subsec:proof-chi-squared-exact-boundary}. 
% The boundary \eqref{eq:exact-boundary-chisquared} is plotted in Figure \ref{fig:phase-chi-squared}.

It is evident that the exact support recovery boundary \eqref{eq:exact-boundary-chisquared} coincides with that in parallel results for the Gaussian additive error models \eqref{eq:model-additive} in Chapter \ref{chap:phase-transitions}.
Implications of these results will be discussed in Section \ref{subsec:one-vs-two-sided} below.

\begin{remark} \label{rmk:strong-classification-boundary-2}
Theorem \ref{thm:chi-squared-exact-boundary} predicts that the asymptotic boundaries are the same for all values of the parameter $\nu$.
In simulations (Section \ref{sec:numerical}), we find this asymptotic prediction to be quite accurate for $\nu\le3$ even in moderate dimensions ($p=100$). 
For $\nu>3$, the phase transitions take place somewhat above the boundary ${g}$.
The behavior is qualitatively similar for the other three phase transitions (see Theorems \ref{thm:chi-squared-exact-approx-boundary}, \ref{thm:chi-squared-approx-boundary}, and \ref{thm:chi-squared-approx-exact-boundary} below).
\end{remark}

\subsection{The exact-approximate support recovery problem}
\label{subsec:exact-approx-support-recovery-chisq}

The next theorem describes the phase transition in the exact-approximate support recovery problem.

\begin{theorem} \label{thm:chi-squared-exact-approx-boundary}
In the context of Theorem \ref{thm:chi-squared-exact-boundary}, 
the function 
\begin{equation} \label{eq:exact-approx-boundary-chisquared}
    \widetilde{g}(\beta) = 1
\end{equation}
characterizes the phase transition of exact-approximate support recovery problem.
Specifically, if $\underline{r} > \widetilde{g}(\beta)$, then the procedures listed in Theorem \ref{thm:chi-squared-exact-boundary} with slowly vanishing nominal FWER levels achieve asymptotically exact-approximate support recovery in the sense of \eqref{eq:support-recovery-success}. 

Conversely, if $\overline{r} < \widetilde{g}(\beta)$, then for any thresholding procedure $\widehat{S}_p$, the exact-approximate support recovery fails in the sense of \eqref{eq:support-recovery-failure}.
\end{theorem}

Theorem \ref{thm:chi-squared-exact-approx-boundary} is proved in Section \ref{subsec:proof-chi-squared-mix-boundaries}. 


\subsection{The approximate support recovery problem}
\label{subsec:approx-support-recovery-chisq}

Our third main result characterizes the phase transition phenomenon in the approximate support recovery problem in the chi-square model.

\begin{theorem} \label{thm:chi-squared-approx-boundary}
Consider the high-dimensional chi-squared model \eqref{eq:model-chisq} with signal sparsity and size as described in \eqref{eq:signal-sparsity} and \eqref{eq:signal-size}.
The function 
\begin{equation} \label{eq:approx-boundary-chisquared}
    h(\beta) = \beta
\end{equation}
characterizes the phase transition of approximate support recovery problem.
Specifically, if $\underline{r} > {h}(\beta)$, then the \ac{BH}procedure $\widehat{S}_p$ (defined in Section \ref{sec:statistical-procedures}) with slowly vanishing (see Definition \ref{def:slowly-vanishing}) nominal FDR levels achieves asymptotically approximate support recovery in the sense of \eqref{eq:support-recovery-success}. 

Conversely, if $\overline{r} < {h}(\beta)$, then approximate support recovery asymptotically fails in the sense of \eqref{eq:support-recovery-failure} for all thresholding procedures.
\end{theorem}

Theorem \ref{thm:chi-squared-approx-boundary} is proved in Section \ref{subsec:proof-chi-squared-mix-boundaries} below. 


\subsection{The approximate-exact support recovery problem}
\label{subsec:aprox-exact-support-recovery-chisq}

A counterpart of Theorem \ref{thm:Gaussian-error-approx-exact-boundary} also holds in the chi-square models.

\begin{theorem} \label{thm:chi-squared-approx-exact-boundary}
In the context of Theorem \ref{thm:chi-squared-approx-boundary}, the function 
\begin{equation} \label{eq:approx-exact-boundary-chisquared}
    \widetilde{h}(\beta) = \left(\sqrt{\beta} + \sqrt{1-\beta}\right)^2
\end{equation}
characterizes the phase transition of approximate-exact support recovery problem.
Specifically, if $\underline{r} > \widetilde{h}(\beta)$, then the Benjamini-Hochberg procedure with slowly vanishing nominal FDR levels achieves asymptotically approximate-exact support recovery in the sense of \eqref{eq:support-recovery-success}. 

Conversely, if $\overline{r} < \widetilde{h}(\beta)$, then for any thresholding procedure $\widehat{S}_p$, the approximate-exact support recovery fails in the sense of \eqref{eq:support-recovery-failure}.
\end{theorem}

Theorem \ref{thm:chi-squared-approx-exact-boundary} is proved in Section \ref{subsec:proof-chi-squared-exact-boundary}. 

Notice that all phase transitions boundaries are identical to those in the Gaussian additive error model \eqref{eq:model-additive} under one-side alternative.
We refer readers to Figure \ref{fig:phase-Gaussian-errors} in Section \ref{sec:additive-error-model-boundaries} for a visualization of the results in Theorems \ref{thm:chi-squared-exact-boundary} through \ref{thm:chi-squared-approx-exact-boundary}.

\medskip

The all four Theorems so far focus only on the idealized models \eqref{eq:model-chisq} where statistics are \emph{independent}.
Support recovery problems under dependent observations remain to be explored.
Recall in Chapter \ref{chap:phase-transitions} we showed that the boundary for the exact support recovery problem in the additive error model \eqref{eq:model-additive} continues to hold even under severe dependence and general distributional assumptions.
We conjecture that similar results would also hold, under classes of dependence structures that are ``not too different from independence'', in the chi-square models.
As an example, in the GWAS application, dependence among the genetic markers at different locations (known as linkage disequilibrium) decay as a function of their physical distances on the genome \citep{bush2012genome}, resulting in locally dependent test statistics.
It would be of great interest to extend the current theory to cover important dependence structures that arise in such applications.


\subsection{Comparison of one- versus two-sided alternatives in additive error models}
\label{subsec:one-vs-two-sided}


% $\mathrm{risk}^{\mathrm{EA}}$ and $\mathrm{risk}^{\mathrm{AE}}$.
As alluded to in Section \ref{sec:motivation-chisq} in the introduction, we draw explicit comparisons between the one-sided and two-sided alternatives in Gaussian additive error models \eqref{eq:model-additive}.
% The exact, and the approximate support recovery problems in the additive error model \eqref{eq:model-additive} under standard Gaussian errors have been studied in \cite{gao2018fundamental} and \cite{arias2017distribution}, respectively. 

The exact support recovery problem in the dependent Gaussian additive error model \eqref{eq:model-additive} was studied in Chapter \ref{chap:phase-transitions}, with parametrization of sparsity identical to that in \eqref{eq:signal-sparsity}, whereas the range of the non-zero (and perhaps unequal) mean shifts $\mu(i)$ was parametrized as 
\begin{equation*}
    \underline{\Delta} = \sqrt{2\underline{r}\log{p}}
    \le \mu(i) \le
    \overline{\Delta} = \sqrt{2\overline{r}\log{p}}, \quad \text{for all}\;\;i\in S_p,
\end{equation*}
for some constants $0<\underline{r}\le\overline{r}\le+\infty$.
Under this one-sided alternative, a phase transition in the $r$-$\beta$ plane was described, where the boundary was found to be identical to \eqref{eq:exact-boundary-chisquared} in Theorem \ref{thm:chi-squared-exact-boundary} for the chi-square models \eqref{eq:model-chisq-Chapter6}. 

As discussed in Section \ref{sec:motivation-chisq}, support recovery problems in the chi-square model with $\nu=1$, , corresponds to the support recovery problems in the additive model under two-sided alternatives.
This implies that the asymptotic signal size requirements are identical between the two-sided alternative and its one-sided counterpart, in order to achieve exact support recovery.
As we shall see in numerical experiments (in Section \ref{sec:numerical} below), the difference is not very pronounced even in moderate dimensions, and vanishes as $p\to\infty$, in accordance with Theorem \ref{thm:chi-squared-exact-boundary}.

\medskip

Comparisons can also be drawn in the approximate, approximate-exact, and exact approximate support recovery problems between the two types of alternatives.

Specifically, the approximate support recovery problem in the Gaussian additive error model \eqref{eq:model-additive} under one-sided alternatives exhibits a phase transition phenomenon characterized by a boundary that coincides with \eqref{eq:approx-boundary-chisquared} in Theorem \ref{thm:chi-squared-approx-boundary}.
Similar to the exact support recovery problem, this indicates vanishing difference in the difficulties of the two types alternatives in approximate support recovery problems.

Comparing Theorems \ref{thm:chi-squared-exact-approx-boundary} to \ref{thm:Gaussian-error-exact-approx-boundary} and Theorems \ref{thm:chi-squared-approx-exact-boundary} to \ref{thm:Gaussian-error-approx-exact-boundary}, we see that the phase transition boundaries under the two types of alternatives are also identical in the exact-approximate and approximate-exact support recovery problems.
The additional uncertainty in the two-sided alternatives do not call for larger signal sizes asymptotically in these problems.

\medskip

To complete the comparisons, we point out that the phase transition boundaries for the sparse signal {detection} problem in the two types of alternatives are both identical to \eqref{eq:detection-boundary-large-signals}. This was analyzed in \cite{donoho2004higher}.







\section{Odds ratios and statistical power}
\label{sec:odds-and-power}
\input{6.GWAS/6.odds-and-power.tex}


\section{Optimal study designs and rare variants}
\label{sec:optimal-design} 
For a study with a fixed budget, i.e., a fixed total number of subjects $n$, the researcher is free to choose the fraction of cases $\phi_1$ to be included in the study.
A natural question is how this budget should be allocated to maximize the statistical power of discovery, or equivalently, the signal sizes $\lambda=nw^2$.

In principal, Relation \eqref{eq:signal-size-odds-ratio} can be optimized with respect to the fraction of cases $\phi_1$ in order to find optimal designs, if $\theta_1$ is known and held constant.
In practice, this is not the case.
While the fraction of cases can be controlled, the distributions of genotypes \emph{in the study} are often unknown prior to data collection, and can change with the case-to-control ratio.

Fortunately, the conditional distributions of genotypes in the healthy control groups are often estimated by existing studies, and are made available by consortia such as the NHGRI-EBI GWAS catalog \cite{macarthur2016new}.
% Assume (after appropriate relabelling, hence without loss of generality) that the first variant is associated with an increased risk of disease, and is henceforth referred to as the risk variant.
We denote the conditional frequency of the first genetic variant in the control group as $(f, 1-f)$, where
\begin{equation} \label{eq:RAF}
    f := \mu_{21} / \phi_2.
\end{equation}
The multinomial probability is fully parametrized by the new trio: $(f, \phi_1, R)$.
\begin{center}
    \begin{tabular}{cccc}
    \hline
    & \multicolumn{2}{c}{Genotype} \\
    \cline{2-3}
    Probabilities & Variant 1 & Variant 2 & Total by phenotype \\
    \hline
    Cases & $\frac{\phi_1fR}{fR+1-f}$ & $\frac{\phi_1(1-f)}{fR+1-f}$ & $\phi_1$ \\
    Controls & $f(1-\phi_1)$ & $(1-f)(1-\phi_1)$ & $1-\phi_1$ \\
    \hline
    \end{tabular}
\end{center}
Proposition \ref{prop:signal-size-odds-ratio} may also be re-stated in terms of the new parametrization.

% Note that all these quantities refer to what is in the study, and differ from their counterparts in the general population.

\begin{corollary} \label{cor:signal-size-odds-ratio-conditional-frequency}
In the 2-by-2 multinomial distribution with marginals $(\phi_1, \phi_2 = 1-\phi_1)$, and conditional distribution of the variants in the control group $(f, 1-f)$,
Relation \eqref{eq:signal-size-odds-ratio} holds with $\theta_1 = {\phi_1fR}/{(fR+1-f)} + f(1-\phi_1)$ and $\theta_2 = 1-\theta_1$.
\end{corollary} 

The choice of $\phi_1$ now has a practical solution.

\begin{corollary} \label{cor:optimal-design}
In the context of Corollary \ref{cor:signal-size-odds-ratio-conditional-frequency},
the optimal design $(\phi^*_1, \phi^*_2)$ that maximizes the signal size per sample $w^2$ is prescribed by
\begin{equation} \label{eq:optimal-design}
    \phi_1^* = \frac{fR+1-f}{fR+1-f+\sqrt{R}}, \quad\text{and}\quad 
    \phi_2^* = 1-\phi_1^*.
\end{equation}
% when the denominator in \eqref{eq:optimal-design} is non-zero; otherwise, $\phi_1^*=\phi_2^*=1/2$.
\end{corollary} 

Corollary \ref{cor:optimal-design} is proved in Section \ref{subsec:proof-signal-size-odds-ratio}. 

Of particular interest in the genetics literature are genetic variants with very low allele frequencies in the control group (i.e., $f\approx 0$), known as rare variants.
In such cases, Equation \eqref{eq:optimal-design} can be approximated using the Taylor expansion,
\begin{equation} \label{eq:optimal-design-approx}
    \phi_1^* = \frac{1}{1 + \sqrt{R}} + \frac{(R-\sqrt{R})f}{1+\sqrt{R}} + O(f^2).
\end{equation}
To illustrate, for rare and adversarial factors ($f\approx0$ and $R>1$), the optimal $\phi_1^*$ is less than $1/2$.
Therefore, for studies under a fixed budget, controls should constitute the majority of the subjects, in order to maximize power.
On the other hand, for rare and protective factors ($f\approx0$ and $R<1$), the optimal $\phi_1^*$ is greater than $1/2$, and cases should be the majority.



\section{Phase transitions in large-scale association screening studies}
\label{sec:phase-transitions-in-GWAS}
\input{6.GWAS/6.GWAS-phase-transitions.tex}


\section{Numerical illustrations of the phase transitions in chi-square models}
\label{sec:numerical}
\input{6.GWAS/6.numerical.tex}


%\section{Proofs}
%\label{sec:proof-signal-size-odds-ratio}
%\input{6.GWAS/6.proofs-GWAS.tex}
