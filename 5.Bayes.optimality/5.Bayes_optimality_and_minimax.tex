
Finally, we offer some insights into the support recovery problem in the case when errors have heavier-than-exponential tails in Section \ref{subsec:optimal-procedure-super-exponential}.





\section{Bayes optimality in support recovery problems}
\label{subsec:Bayes-optimality}

In studying support recovery problems (e.g., \citet{arias2017distribution}), restrictions to the thresholding procedures are sometimes justified by arguing that such procedures are the ``reasonable'' choice for estimating the support set.
%  In Theorem \ref{thm:sufficient} we saw that a (FWER-controlling) thresholding procedure is indeed consistent for support recovery. 
% It is natural to ask if it will be sufficient to restrict our attention to only such procedures.
We show in this section that, perhaps surprisingly, for general error models, thresholding procedures are not always optimal, even when the observations are independent.

We shall identify the optimal procedure for support recovery problems under a Bayesian setting with general distributional assumptions (including but not limited to additive models \eqref{eq:model-additive}).
Specifically, we assume that there is an ordered set $P = (i_1,\ldots,i_s)$, $i_i\in\{1,\ldots,p\}$, and $s$ (not necessarily equal) densities $f_{1}, \ldots, f_{s}$, such that the observations indexed by $P$ have corresponding densities. That is,
\begin{equation} \label{eq:signal-distributions-ordered}
x(i_j) \sim f_j, \quad j=1,\ldots,s.
\end{equation}
Let also the rest $(p-s)$ observations have common density $f_0$, i.e., $x(i)\sim f_0$ for $i\not\in S$.
We further assume that the observations $x$ are mutually independent.

We adopt here a Bayesian framework to measure statistical risks. 
Let the ordered support $P=(i_1,\ldots,i_s)$ have prior
\begin{equation} \label{eq:uniform-ordered}
\pi((i_1,\ldots, i_s)) = {(p-s)!}/{p!},
\end{equation}
for all distinct $1\le i_1, \ldots, i_s\le p$.
Consequently, the unordered support $S=\{i_1,\ldots,i_s\}$ is distributed uniformly in the collection of all set of size $s$, with the unordered uniform distribution $\pi^{\text{u}}$. That is, for all for all $S\in\mathcal{S}:=\left\{S\subseteq\{1,\ldots,p\};|S|=s\right\}$, we have 
\begin{equation} \label{eq:uniform}
\pi^{\text{u}}
(\{i_1,\ldots, i_s\}) = \sum_{\sigma}\pi((i_{\sigma(1)},\ldots, i_{\sigma(s)})) = {(p-s)!s!}/{p!},
\end{equation}
where the sum is taken over all permuations of $\{1,2,\ldots,s\}$.

For any fixed configuration $P$, consider the loss function,
$$
\ell(\widehat{S},S) := \P[\widehat{S} \neq S] = \P_{P}[\widehat{S} \neq S],
$$
where the probability is taken over the randomness in the observations only.
The Bayes optimal procedures should minimize 
\begin{equation} \label{eq:Bayes-risk}
    \E_\pi \P[\widehat{S} \neq S],
\end{equation}
where the expectation is taken over the random configurations $P$, with a uniform distribution $\pi$ as specified in \eqref{eq:uniform-ordered}.

% For a sequence of estimators $\widehat{S} = \widehat{S}_p$, the asymptotically optimal procedures are defined to be the ones that minimize the asymptotic Bayes risk,
% \begin{equation} \label{eq:Bayes-risk}
%     R(\widehat{S}) := \limsup_{p\to\infty} \E_{P,{\cal E}} [\ell(\widehat{S},S)] = \limsup_{p\to\infty} \P_{P,{\cal E}}[\widehat{S} \neq S].
% \end{equation}

If, however, the sparsity $s = |S|$ of the problem is known, then a natural estimator for $S$ would be based on the set of top $s$ order statistics.
\begin{definition}[Oracle data thresholding]
We call $\widehat{S}^* = \{i\,|\, x(i)\ge x_{[s]}\}$ the oracle data thresholding procedure, where $x_{[1]} \ge \ldots \ge x_{[p]}$ are the order statistics of $x$.
\end{definition}



\subsection{When is oracle thresholding Bayes optimal?} % under sub-exponential errors}
\label{subsec:optimal-procedure-sub-exponential}

An elegant answer of this question can be given in terms of the \emph{monotone likelihood ratio} (MLR) property.

\begin{definition}[Monotone Likelihood Ratio]
A family of positive densities on $\R$, $\{f_\delta, \delta \in U\}$, is said to have the MLR property if, for all $\delta_0, \delta_1\in U\subseteq\R$ such that $\delta_0 < \delta_1$, the likelihood ratio $\left(f_{\delta_1}(x)/f_{\delta_0}(x)\right)$ is an increasing function of $x$.
\end{definition}

The following result provides a general criterion for the finite-sample optimality of the oracle thresholding 
procedure $\widehat{S}^*$.

\begin{proposition} \label{prop:optimal-oracle-procedures}
Let the observations $x(i)$, $i=1,\ldots,p$ be as prescribed as in \eqref{eq:signal-distributions-ordered} through \eqref{eq:uniform-ordered}.
If each of $\{f_0, f_{1}\},\ldots,\{f_0,f_{s}\}$ form an MLR family, then the oracle data thresholding procedure $\widehat{S}^* = \{i\,|\, x(i)\ge x_{[s]}\}$ is finite-sample optimal in terms of Bayes risk $\E_\pi \P[\widehat{S} \neq S]$. That is,
\begin{equation} \label{eq:finite-sample-Bayes-optimal}
    \widehat{S}^* \in \argmin_{\widehat{S}} \E_\pi \P[\widehat{S} \neq S].
\end{equation}
for all $s$ and $p$.
\end{proposition} 
\begin{proof}%[Proof of Proposition \ref{prop:optimal-oracle-procedures}]
The problem of support recovery can be equivalently stated as a classification problem, where the discrete parameter space is $\mathcal{S} = \{S\subseteq\{1,\ldots,p\}:|S|=s\}$, and the observation $x \in\R^p$ has likelihood $f(x|S)$ indexed by the support set $S$.

By the optimality of the Bayes classifier (see, e.g., \citep{domingos1997optimality}), a set estimator that maximizes the probability of support recovery is such that
$$
\widehat{S} \in \argmax_{S\in \mathcal{S}} f(x|S) \pi(S).
$$
Since we know from \eqref{eq:uniform} that $\pi(S)$ is uniform, the problem in our context reduces to showing that $f(x|\widehat{S}^*) = f(x|\widehat{S})$, where $f(x|S)$ is the conditional distribution of data given the unordered support $S$,
$$
f(x|S) 
= \sum_{P\in\sigma(S)} f(x|P) \pi^{\text{ord}}(P|S) 
= \frac{1}{s!} \left(\sum_{P\in\sigma{(S)}} \prod_{i=1}^s {f_{i}(x(P(i)))}\right) \prod_{k\not\in S}{f_0(x(k))},
$$
where $\sigma(S)$ is the set of all permutations of the indices in the support set $S$.

Suppose $\widehat{S} \neq \widehat{S}^*$, then there must be indices $j \in \widehat{S}$ and $j' \in \widehat{S}^c$ such that $x(j) \le x(j')$.
We exchange the labels of $x(j)$ and $x(j')$, and form a new estimate $\widehat{S}\,' = \big(\widehat{S}\setminus\{j\}\big)\cup\{j'\}$.
Comparing the likelihoods under $\widehat{S}$ and $\widehat{S}\,'$, we have
\begin{align}
    f(x|\widehat{S}) - f(x|\widehat{S}\,') 
    &= \frac{1}{s!} \sum_{P\in\sigma{(\widehat{S})}} \prod_{i=1}^s {f_{i}(x(P(i)))} f_0(x(j'))\prod_{k\not\in \widehat{S}\cup\{j'\}}{f_0(x(k))} - \nonumber \\
    &\quad\quad\quad - \frac{1}{s!} \sum_{P'\in\sigma{(\widehat{S}')}} \prod_{i=1}^s {f_{i}(x(P'(i)))} f_0(x(j)) \prod_{k\not\in \widehat{S}'\cup\{j\}}{f_0(x(k))} \nonumber \\
    &= \frac{1}{s!} \left(\sum_{i=1}^s a_i  \Big(f_i(x(j)) f_0(x(j')) - f_i(x(j')) f_0(x(j))\Big) \right) \prod_{k\not\in \widehat{S}\cup\{j'\}}{f_0(x(k))}, \label{eq:MLR-optimality-proof}
    % = \log{f_\delta(x(j))} + \log{f_0(x(j'))} - \log{f_\delta(x(j'))} - \log{f_0(x(j))} 
\end{align}
where the last equality follows by first summing over all permutations fixing $P(i) = j$ and $P'(i) = j'$, and setting $a_i = \sum_{P\in\sigma{(\widehat{S}\setminus\{j\})}} \prod_{i'\neq i} {f_{i'} (x(P(i')))}$. Notice that the $a_i$'s are non-negative.

Since $x(j) \le x(j')$, and that each of $\{f_0, f_{i}\}$ is an MLR family, we have
$$
\frac{f_i(x(j))}{f_0(x(j))} - \frac{f_i(x(j'))}{f_0(x(j'))} \le 0 \implies f_i(x(j)) f_0(x(j')) - f_i(x(j')) f_0(x(j)) \le 0.
$$
Using Relation \eqref{eq:MLR-optimality-proof}, we conclude that $f(x|\widehat{S}) \le f(x|\widehat{S}\,')$.
This implies that any estimator that is not $\widehat{S}^*$ may be improved, and the optimality follows.
\end{proof}

We emphasize that under the MLR conditions in Proposition \ref{prop:optimal-oracle-procedures},
 the oracle thresholding procedures are in fact \emph{finite-sample optimal} in the 
above Bayesian context. Further, our setup allows for different alternative distributions, and relaxes the assumptions 
of \citet{butucea2018variable} when studying distributional generalizations, where the alternatives are assumed to be identically distributed.

It remains to understand when the key MLR property holds.  We elaborate on this question next.
Returning to the more concrete signal-plus-noise model \eqref{eq:model-additive}, it turns out that the error tail behavior is what determines the optimality of data thresholding procedures.
In this setting, log-concavity of the error densities is \emph{equivalent} to the MLR property (Lemma \ref{lemma:MLR-log-concavity}). This, in turn, yields the finite-sample optimality of data thresholding procedures (Proposition \ref{prop:log-concave}).

\begin{lemma} \label{lemma:MLR-log-concavity}
Let $\delta$ be the magnitude of the non-zero signals in the signal-plus-noise model \eqref{eq:model-additive} with positive error density $f_0$, and let $f_\delta(x) = f_0(x-\delta)$.
The family $\{f_\delta, \delta \in \R\}$ has the MLR property if and only if the error density $f_0$ is log-concave.
\end{lemma}

\begin{proof}%[Proof of Lemma \ref{lemma:MLR-log-concavity}]
Suppose MLR holds, we will show that $f_0(t) = \exp\{\phi(t)\}$ for some concave function $\phi$.
By the assumption of MLR, for any $x_1 < x_2$, setting $\delta_0 = 0$, and $\delta_1 = (x_2 - x_1)/2 > 0$, we have
\begin{equation*}
    \log{\frac{f_{\delta_1}(x_2)}{f_{\delta_0}(x_2)}}
    = \phi\left(\frac{(x_1+x_2)}{2}\right)- \phi(x_2) 
    \ge \phi(x_1)- \phi\left(\frac{(x_1+x_2)}{2}\right) 
    = \log{\frac{f_{\delta_1}(x_1)}{f_{\delta_0}(x_1)}}.
\end{equation*}
This implies that the log-density $\phi(t)$ is midpoint-concave, i.e., for all $x_1$ and $x_2$, we have,
\begin{equation}
    \phi\left(\frac{(x_1+x_2)}{2}\right) 
    \ge \frac{1}{2} \phi(x_1) + \frac{1}{2} \phi(x_2).
\end{equation}
For Lebesgue measurable functions, midpoint concavity is equivalent to concavity by the Sierpinki Theorem (see, e.g., Sec I.3 of \cite{donoghue2014distributions}). This proves the `only-if' part.

For the `if' part, when  $\phi(t) = \log{(f_0(t))}$ is log-concave, then for any $\delta_0 < \delta_1$, and any $x<y$, we have
\begin{equation} \label{eq:concavity-implies-MLR}
    \log{\frac{f_{\delta_1}(y)}{f_{\delta_0}(y)}} - \log{\frac{f_{\delta_1}(x)}{f_{\delta_0}(x)}}
    = \phi(y-\delta_1) - \phi(y-\delta_0) - \phi(x-\delta_1) + \phi(x-\delta_0) \ge 0,
\end{equation}
where the last inequality is a simple consequence of concavity (see Lemma \ref{lemma:four-point-concavity} below). This proves the `if' part.
\end{proof}
\begin{lemma} \label{lemma:four-point-concavity}
Let $\phi$ be any concave function on $\R$. For any $x<y\in\R$, and $\delta>0$ we have
$$
\phi(x) + \phi(y+\delta) \le \phi(y) + \phi(x+\delta).
$$
\end{lemma}

\begin{proof}%[Proof of Lemma \ref{lemma:four-point-concavity}]
Pick $\lambda = \delta/(y-x+\delta)$, by concavity of $f$ we have
\begin{equation} \label{eq:four-point-concavity-1}
    \lambda \phi(x) + (1-\lambda) \phi(y+\delta) 
    \le \phi(\lambda x + (1-\lambda)(y+\delta)) 
    = \phi(y),
\end{equation}
and
\begin{equation} \label{eq:four-point-concavity-2}
    (1-\lambda) \phi(x) + \lambda \phi(y+\delta)
    \le \phi((1-\lambda) x + \lambda(y+\delta)) 
    = \phi(x+\delta).
\end{equation}
Summing up \eqref{eq:four-point-concavity-1} and \eqref{eq:four-point-concavity-2} and we arrive at the conclusion as desired.
\end{proof}

Proposition \ref{prop:optimal-oracle-procedures} and Lemma \ref{lemma:MLR-log-concavity} yield immediately the following.

\begin{proposition} \label{prop:log-concave}
Consider the additive error model \eqref{eq:model-additive}.
Let the errors $\epsilon$ be independent with common distribution $F$.
Let the signal $\mu$ have $s$ positive entries with magnitudes $0<\delta_1\le\ldots\le\delta_s$, located on $\{1,\ldots,p\}$ as prescribed in \eqref{eq:uniform-ordered}.
If $F$ has a positive, log-concave density $f$, then the oracle thresholding procedure $\widehat{S}^* = \{i\,;\,x(i)\ge x_{[s]}\}$ is finite-sample optimal in terms of Bayes risk in the sense of \eqref{eq:finite-sample-Bayes-optimal}.
\end{proposition} 
Notice that under MLR (or equivalently, log-concavity of the errors in additive models), the oracle thresholding procedure is finite-sample optimal even in the case where the signals have different (positive) sizes.

The assumption of log-concavity of the densities is compatible with the AGG model when $\nu\ge1$, as demonstrated in the next example.

\begin{example} \label{exmp:AGG-logconcave}
The generalized Gaussian density $f(x)\propto \exp\{-|x|^\nu/\nu\}$ is log-concave for all $\nu\ge1$.
Therefore in the additive error model \eqref{eq:model-additive}, according to Proposition \ref{prop:log-concave}, the oracle thresholding procedure is Bayes optimal in the sense of \eqref{eq:finite-sample-Bayes-optimal}.
\end{example}

%(This also explains the $\gamma\ge 1$ condition in Theorem 1 of Arias-Castro and Chen)

Consider the asymptotic Bayes risk as defined in \eqref{eq:Bayes-risk}, the statement for the necessary condition of support recovery, with the help of Proposition \ref{prop:log-concave}, can be strengthened to include all procedures (in the Bayesian context), regardless of whether they are thresholding.

\begin{theorem} \label{thm:necessary-strengthened}
Consider the additive model \eqref{eq:model-additive} where the $\epsilon_p(i)$'s are independent and identically distributed with log-concave densities in the AGG class. 
Let the signals be as prescribed in Proposition \ref{prop:log-concave}.
If the signal sizes fall below the strong classification boundary \eqref{eq:strong-classification-boundary}, i.e. $\overline{r}<g(\beta)$,
then we have
\begin{equation} \label{eq:classification-impossible-Bayes}
    \liminf_{p\to\infty} \inf_{\widehat{S}_p} \E_\pi \P[\widehat{S}_p\neq S_p] = 1,
\end{equation}
where the infimum on $\widehat{S}_p$ is taken over all procedures.
\end{theorem}

\begin{proof}%[Proof of Theorem \ref{thm:necessary-strengthened}]
When errors are independent with log-concave density, the oracle thresholding procedure $\widehat{S}^*_p$, by Proposition \ref{prop:log-concave}, minimizes the Bayes risk \eqref{eq:Bayes-risk} among \emph{all} procedures. That is,
$$
\liminf_{p\to\infty} \inf_{\widehat{S}_p} \E_\pi\P[\widehat{S}_p\neq S_p]
\ge \liminf_{p\to\infty} \E_\pi\P[\widehat{S}^*_p\neq S_p].
$$
Since $\widehat{S}^*_p$ belongs to the class of all thresholding procedures, we have
\begin{align*}
    \liminf_{p\to\infty} \E_\pi\P[\widehat{S}^*_p\neq S_p] 
    &\ge \liminf_{p\to\infty} \inf_{\widehat{S}_p\in{\cal T}} \E_\pi\P[\widehat{S}_p\neq S_p] \\
    &\ge \liminf_{p\to\infty} \inf_{\widehat{S}_p\in{\cal T}} \inf_{S_p} \P[\widehat{S}_p\neq S_p] = 1,
\end{align*}
when $\overline{r}<g(\beta)$, where the last line follows from Theorem \ref{thm:necessary}.
\end{proof}

\subsection{Bayes optimality of likelihood ratio thresholding}
\label{subsec:optimal-procedure-super-exponential}

When the MLR condition in Proposition \ref{prop:optimal-oracle-procedures} is violated, the oracle thresholding 
procedures can in fact be sub-optimal (see Example \ref{exmp:suboptimal-data-thresholding}, below). 
The following result provides the general form of the finite-sample Bayes optimal procedures.
It turns out that in general, \emph{likelihood ratio thresholding} is optimal.

\begin{proposition} \label{prop:likelihood-ratio-thresholding}
Let the observations $x(i)$, $i=1,\ldots,p$ have $s$ signals as prescribed in \eqref{eq:uniform-ordered} having common density $f_a$, and let the rest $(p-s)$ locations have common density $f_0$.
Define the likelihood ratios 
$$
L(i) := {f_a(x(i))}\big/{f_0(x(i))},
$$
and let $L_{[1]} \ge L_{[2]} \ge \ldots \ge L_{[p]}$ be the order statistics of the $L(i)$'s.
Then the procedure $\widehat{S}_{\text{opt}} = \{i\,|\,L(i) \ge L_{[s]}\}$ is finite-sample optimal in terms of Bayes risk. That is,
\begin{equation}
    \widehat{S}_{\text{opt}} \in \argmin_{\widehat{S}} \E_\pi \P[\widehat{S} \neq S].
\end{equation}
for all $s$ and $p$, where the infimum on $\widehat{S}_p$ is taken over all procedures.
\end{proposition} 
\begin{proof}%[Proof of Proposition \ref{prop:likelihood-ratio-thresholding}]
The proof is entirely analogous to that of Proposition \ref{prop:optimal-oracle-procedures}.
Since we know from \eqref{eq:uniform} that $\pi(S)$ is uniform, the problem reduces to showing that $f(x|\widehat{S}_{\text{opt}}) = f(x|\widehat{S})$, where 
$$
\widehat{S} \in \argmax_{S \in \mathcal{S}} f(x|S) \pi(S).
$$
and
$f(x|S)$ is the conditional distribution of data given the unordered support $S$,
\begin{equation} \label{eq:likelihood-ratio-thresholding-proof}
    f(x|S) = \sum_P f(x|P) \pi^{\text{ord}}(P|S) = \prod_{j\in S} f_a(x(j)) \prod_{j\not\in S}{f_0(x(j))}.
\end{equation}
Suppose $\widehat{S} \neq \widehat{S}_{\text{opt}}$, then there must be indices $j \in \widehat{S}$ and $j' \in \widehat{S}^c$ such that $L(j) \le L(j')$.
If we exchange the labels of $L(j)$ and $L(j')$, that is, we form a new estimate $\widehat{S}\,' = \big(\widehat{S}\setminus\{j\}\big)\cup\{j'\}$,
comparing the log-likelihoods under $\widehat{S}$ and $\widehat{S}\,'$, we have
\begin{equation*}
    \log{f(x|\widehat{S})} - \log{f(x|\widehat{S}\,')} 
    = \log{f_a(x(j))} + \log{f_0(x(j'))} - \log{f_a(x(j'))} - \log{f_0(x(j))}.
\end{equation*}
By the definition of $L(j)$'s, and the order relations, we obtain
\begin{equation*}
    \log{f(x|\widehat{S})} - \log{f(x|\widehat{S}\,')} 
    = \log{L(j)} - \log{L(j')} \le 0
\end{equation*}
This implies that any estimator that is not $\widehat{S}_{\text{opt}}$ may be improved, and the optimality follows.
\end{proof}


The characterization of optimal likelihood ratio thresholding procedures in Proposition \ref{prop:likelihood-ratio-thresholding} may not always yield practical estimators, as the density of alternatives, and number of signals are typically unknown.
Still, some insights can be gained by virtue of Proposition \ref{prop:likelihood-ratio-thresholding}.
In particular, when MLR fails (or equivalently, when the errors in model \eqref{eq:model-additive} do not have log-concave densities), data thresholding is sub-optimal. 


\begin{example}[Sub-optimality of data thresholding] \label{exmp:suboptimal-data-thresholding}
Let the errors have iid generalized Gaussian density with $\nu=1/2$, i.e., $\log{f_0(x)}\propto -x^{1/2}$. 
Let dimension $p=2$, sparsity $s=1$ with uniform prior, and signal size $\delta=1$.
That is, $\mathbb P[\mu = (0,1)^\mathrm{T}] = \mathbb P[\mu = (1,0)^\mathrm{T}] = 1/2$.
If the observations take on values $x = (x_1, x_2)^\mathrm{T} = (1,2)^\mathrm{T}$, we see from a comparison of the likelihoods (and hence, the posteriors),
$$
\log \frac{f(x|\{1\})}{f(x|\{2\})} = 2x_1^{1/2} + 2(x_2 - 1)^{1/2} - 2x_2^{1/2} - 2(x_1 - 1)^{1/2} = 4 - 2\sqrt{2} > 0,
$$
that even though $x_1<x_2$, the set $\{1\}$ is a better estimate of support than $\{2\}$, i.e., $\mathbb P[S=\{1\}\,\big|\,x] > \mathbb P[S=\{2\}\,\big|\,x]$.
\end{example}

This simple example shows that, in the case when the errors have super-exponential tails, the optimal procedures are in general \emph{not} data thresholding.
A slightly more general conclusion can be found in Corollary \ref{cor:log-convex}.

\subsection{Sub-optimality of data thresholding procedures}

We provide a slightly more general result on the sub-optimality of data thresholding procedures.

\begin{corollary} \label{cor:log-convex}
Consider the additive error model \eqref{eq:model-additive}.
Let the errors $\epsilon$ be independent with common distribution $F$.
Let each of the $s$ signals be located on $\{1,\ldots,p\}$ uniformly at random with equal magnitude $0<\delta<\infty$.
% , independently with probability $p_1,\ldots,p_s$.
% , {\color{red} so that we have $f_a = p_1f_1 + p_2f_2 + \ldots + p_sf_s$.}
If errors $\epsilon$ are iid with density $f$ log-convex on $[K, +\infty)$, then whenever $j\in\widehat{S}_{\text{opt}}$ for some $x(j) > K+\delta$, we must have $j'\in\widehat{S}_{\text{opt}}$ for all $j'$ such that $K+\delta \le x(j') < x(j)$.
\end{corollary} 

Specifically, if there are $m$ observations exceeding $K+\delta_s$, with $m>s$, then the top $m-s$ observations will \emph{not} be included in the optimal estimator $\widehat{S}_{\text{opt}}$.
This shows that, in the case when the errors have super-exponential tails, the optimal procedures are in general \emph{not} data thresholding.

\begin{proof}[Proof of Corollary \ref{cor:log-convex}]
Since the alternative $f_{a}(t) = {f(t-\delta)}$ are log-convex on $[K+\delta, \infty)$, by Relation \eqref{eq:concavity-implies-MLR} in the proof of Lemma \ref{lemma:MLR-log-concavity} and appealing to log-convexity (rather than log-concavity), the likelihood ratio
$$
L(j) := \frac{f_a(x(j))}{f_0(x(j))} %= \sum_{i=1}^s \frac{p_if_{i}(x(j))}{f_0(x(j))},
$$
is decreasing in $x(j)$ on $[K+\delta, \infty)$.
The proof is complete by applying Proposition \ref{prop:likelihood-ratio-thresholding}.
\end{proof}

\begin{remark}[An open problem]
In the non-log-concave setting, where we know that thresholding procedures are suboptimal, likelihood thresholding procedures are promising, thanks to Proposition \ref{prop:likelihood-ratio-thresholding}. 

In the case where the signals have difference sizes, however, likelihood thresholding procedures are undefined since 
in the proof of Proposition \ref{prop:likelihood-ratio-thresholding}, the signal densities need to be identical for Relation \eqref{eq:likelihood-ratio-thresholding-proof} to hold.  In such cases, the existence of an optimal procedure is an open problem.
\end{remark}







% \begin{remark}
% In view of Theorem \ref{thm:necessary-strengthened}, Bonferroni's thresholding procedure $\widehat{S}_p^{\text{Bonf}}$ described in Theorem \ref{thm:sufficient} is an asymptotically minimax procedure over the class of error models $U(F)$.
% That is, in the class of errors with AGG($\nu$) marginals $F$ where $\nu\ge1$, we have
% \begin{equation} \label{eq:minimax}
%     \widehat{S}_p^{\text{Bonf}} \in \argmin_{\widehat{S}_p} \sup_{{\cal E}\in U(F)} R(\widehat{S}_p),
% \end{equation}
% where the minimum is taken over \emph{all} procedures $\widehat{S}$.
% It can be shown that the supremum in the Bayes risk formulation \eqref{eq:minimax}, where $S$ is random and uniformly distributed, can be replaced by a supremum over all possible (deterministic) configurations of support sets.
% This can be done by applying Theorem 1.1 in the supplement of \citep{butucea2018variable}.
% We content ourselves with a Bayesian minimax result here.
% \end{remark}

\section{Minimax optimality in exact support recovery}

We establish in this section minimax versions of our results from Chapter \ref{chap:exact-support-recovery}.
Specifically, if we restrict ourselves to \emph{the class of thresholding procedures} $\cal T$ (defined in \eqref{eq:thresholding-procedure}), then Bonferroni's procedure is minimax optimal, for \emph{any} fixed dependence structures in the URS class.
This is formalized in Corollary \ref{cor:point-wise-minimax} below.% in Section \ref{subsec:point-wise-minimax}.
We refer to this result as \emph{point-wise} minimax, to emphasize the fact that this optimality holds for every \emph{fixed} URS array.

Meanwhile, if we search over \emph{all procedures}, but expand the model space to include {\em all} dependence 
structures, then a different minimax optimality statement holds for Bonferroni's procedure.
This result, formally stated in Section \ref{subsec:minimax-over-dependence}, is a consequence of our characterization 
of the finite-sample Bayes optimality of thresholding procedures in Sections \ref{subsec:Bayes-optimality} and \ref{subsec:optimal-procedure-sub-exponential}.
% The result is of independent interest and it is the main contribution of this section.


\subsection{Point-wise minimax optimality for thresholding procedures}
\label{subsec:point-wise-minimax}

Theorems \ref{thm:sufficient} and \ref{thm:necessary} can be cast in the form of an asymptotic minimax statement. 
% (concealing the gap between sufficient and necessary conditions as pointed out in Remark \ref{rmk:gap-between-sufficient-necessary}).

\begin{corollary}[Point-wise minimax]
\label{cor:point-wise-minimax}
Let $\widehat{S}^{\text{Bonf}}$ be the sequence of Bonferroni's procedure described in Theorem \ref{thm:sufficient}. 
Let also the errors have common $\text{AGG}(\nu)$ distribution $F$ with parameter $\nu>0$, and $\Theta_p^+$ be as defined in \eqref{eq:minimax-signal-config-over}.
If $\underline{r}>g(\beta)$, then we have
\begin{equation} \label{eq:point-wise-minimax-above}
    \limsup_{p\to\infty} \sup_{\mu\in\Theta_p^+(\beta, \underline{r})} \P(\widehat{S}^{\text{Bonf}}_p \neq S_p) = 0,
\end{equation}
for arbitrary dependence structure of the error array ${\cal E} = \{\epsilon_p(i)\}_p$.
Let $\cal T$ be the class of thresholding procedures \eqref{eq:thresholding-procedure}. 
If $\underline{r}<g(\beta)$, then we have
\begin{equation} \label{eq:point-wise-minimax-below}
    \liminf_{p\to\infty} \inf_{\widehat{S}_p\in \cal T} \sup_{\mu\in\Theta_p^+(\beta, \underline{r})} \P(\widehat{S}_p \neq S_p) = 1,
\end{equation}
for any error dependence structure such that ${\cal E}\in U(F)$ and ${(-\cal E)}\in U(F)$.
\end{corollary}

\begin{proof}%[Proof of Corollary \ref{cor:point-wise-minimax}]
The first conclusion \eqref{eq:point-wise-minimax-above} is a restatement of Theorem \ref{thm:sufficient}.

For the second statement \eqref{eq:point-wise-minimax-below}, since $\underline{r}<g(\beta)$, we can pick a sequence $\mu^*\in\Theta_p^+(\beta, \underline{r})$ such that $|S_p| = \lfloor p^{1-\beta}\rfloor$, with signals having the same signal size $\mu(i)=(2r\log{p})^{1/\nu}$ for all $i\in S_p$, where $\underline{r}<{r}<g(\beta)$.
For this particular choice of $\mu^*$ we have $\mu^*\in\Theta_p^-(\beta, \overline{r})$ where $r<\overline{r}<g(\beta)$,
and according to Theorem \ref{thm:necessary}, we obtain $\lim_{p\to\infty} \inf_{\widehat{S}_p\in \cal T} \mathbb P[\widehat{S}_p \neq S_p] = 1$, for all dependence structures in the URS class.
\end{proof}

\begin{remark}
Theorem \ref{thm:necessary} is a stronger result than the traditional minimax claim in Relation \eqref{eq:point-wise-minimax-below}.
Indeed,  \eqref{eq:classification-impossible-dependent} involves an infimum (over the class $\Theta^-_p$) while \eqref{eq:point-wise-minimax-below} has a supremum (over the class $\Theta^+_p$).

On the other hand, Corollary \ref{cor:point-wise-minimax} is more informative than many minimax-type statements, since it applies ``point-wise'' to any fixed error dependence structure in the URS class.
\end{remark}

\begin{remark}
Corollary \ref{cor:point-wise-minimax} echoes Remark \ref{rmk:dependence-assumptions}: for a very large class of dependence structures, we cannot improve upon Bonferroni's procedure in exact support recovery problems (asymptotically), unless we look beyond thresholding procedures.
% There is a limited amount of literature on non-thresholding procedures that utilize the error dependence structures to improve power.
% A notable effort was made by \citet{jin2014optimality}, where the performance (in terms of Hamming loss) of a so-called graphlet screening procedure was studied. 
\end{remark}

% This is an enhancement (in the asymptotic sense) of the minimax results in Section 4.1 of \cite{butucea2018variable}, where the supremum was taken over the dependence structures.


\subsection{Minimax optimality over all procedures}
\label{subsec:minimax-over-dependence}

% In the previous section we demonstrated the minimax optimality of thresholding procedures, in a point-wise sense over a large class of dependence structure.
Theorem \ref{thm:necessary-strengthened} allows us to state another minimax conclusion --- one in which we search over \emph{all procedures}, by allowing the supremum in the minimax statement to be taken over the dependence structures. 

\begin{corollary}
\label{cor:minimax-over-dependence}
Let $D(F)$ be the collection of error arrays with common marginal $F$ as defined in \eqref{eq:common-marginal-distribution} where $F$ is an $\text{AGG}(\nu)$ distribution.
Let also $\widehat{S}^{\text{Bonf}}_p$ be Bonferroni's procedure as described in Theorem \ref{thm:sufficient}.
If $\underline{r}>g(\beta)$, then we have
\begin{equation} \label{eq:minimax-over-dependence-over}
    \limsup_{p\to\infty} \sup_{\substack{\mu\in\Theta_p^+(\beta,\underline{r})\\ {\cal E}\in D(F)}} \P(\widehat{S}^{\text{Bonf}}_p \neq S_p) = 0.
\end{equation}
Further, when $\underline{r}<g(\beta)$, and $F$ has a positive log-concave density $f$, we have
\begin{equation} \label{eq:minimax-over-dependence-under}
    \liminf_{p\to\infty} \inf_{\widehat{S}_p} \sup_{\substack{\mu\in\Theta_p^+(\beta,\underline{r})\\ {\cal E}\in D(F)}} \P(\widehat{S}_p \neq S_p) = 
    1,
\end{equation}
where the infimum on $\widehat{S}_p$ is taken over all procedures.
\end{corollary}
\begin{proof}%[Proof of Corollary \ref{cor:minimax-over-dependence}]
Relation \eqref{eq:minimax-over-dependence-over} is a re-statement of Remark \ref{rmk:sufficient-strengthened}.

For any distribution $\pi$ (with a slight abuse of notation) over the parameter space $\Theta_p^+\times D(F)$, we have
\begin{equation} \label{eq:minimax-over-dependence-under-proof}
    \liminf_{p\to\infty} \inf_{\widehat{S}_p} \sup_{\substack{\mu\in\Theta_p^+(\beta, \underline{r})\\ {\cal E}\in D(F)}} \P(\widehat{S}_p \neq S_p) 
\ge \liminf_{p\to\infty} \inf_{\widehat{S}_p} \E_{\pi} \P(\widehat{S}_p \neq S_p),
\end{equation}
since the supremum is bounded from below by expectations.
In particular, define $\pi$ to be the uniform distribution over the configurations $\Theta_p^*\times I(f)$, where
\begin{align*}
    \Theta_p^* &=
    \{\mu\in\mathbb{R}^d:\;|S_p|=\lfloor p^{1-\beta}\rfloor,\;\mu(i)=0\;\text{for all }i\not\in S, \;\text{and}\\
    &\quad\quad\mu(i) = (\nu{r}\log{p})^{1/\nu}\;\text{for all }i\in S,\;\text{where}\;\underline{r}<r<g(\beta)\},
\end{align*}
and 
% \begin{align*}
%     I(f)&=\{{\cal E}=(\epsilon_p(i))_p:\;\epsilon_p(i)\;\text{are independently and identically distributed}\\
%     &\quad\quad\text{with density}\; f(x)\propto \exp\{-|x|^\nu/\nu\}\}.
% \end{align*}
\begin{equation*}
    I(f)=\{{\cal E}=(\epsilon_p(i))_p:\;\epsilon_p(i)\;\text{\ac{iid} with density}\; f(x)\propto \exp\{-|x|^\nu/\nu\}\}.
\end{equation*}

Since the density $f$ of $F$ is log-concave, the distribution of the signal configurations satisfies the conditions in Theorem \ref{thm:necessary-strengthened}.
Thus, the desired conclusion \eqref{eq:minimax-over-dependence-under} follows from Theorem \ref{thm:necessary-strengthened} and \eqref{eq:minimax-over-dependence-under-proof}.
\end{proof}

\begin{remark}
Since the class $\text{AGG}(\nu)$, $\nu\ge1$ contains distributions with log-concave densities (Example \ref{exmp:AGG-logconcave}), the minimax statement \eqref{eq:minimax-over-dependence-under} continues to hold if the supremum is taken over the entire class $F\in\text{AGG}(\nu)$, $\nu\ge1$.
We opted for a more informative formulation which emphasizes the log-concavity condition on the density of $F$.
\end{remark}

\begin{remark}
Corollary \ref{cor:minimax-over-dependence} is no stronger than Corollary \ref{cor:point-wise-minimax}. 
In Corollary \ref{cor:point-wise-minimax} we search over only the class of thresholding procedures, but offer a tight, point-wise lower bound on the asymptotic risk over the class of URS dependence structures.
On the other hand, Corollary \ref{cor:minimax-over-dependence} provides a uniform lower bound for the asymptotic risk over all dependence structures, which may not be tight except in the case of independent errors.
% , and when the errors have nice, log-concave densities.
\end{remark}




\section{A discussion on optimality}

  \stilian{Here we briefly discuss the optimality and sub-optimality of the 
 thresholding procedures.}{Needs work}
 
%\begin{remark} \label{rmk:tail-assumptions}

Consider the model \eqref{eq:model-additive} with independent errors, Proposition \ref{prop:likelihood-ratio-thresholding}, and indeed, Example \ref{exmp:suboptimal-data-thresholding} demonstrate that thresholding procedures are in fact \emph{sub-optimal} for $\text{AGG}(\nu)$ models with $\nu<1$.
Therefore, the optimality of thresholding procedures (specifically, Bonferroni's procedure) only applies to $\text{AGG}(\nu)$ models with $\nu\ge1$. 
% This suggests that the condition $\nu\ge1$ in Corollary \ref{cor:minimax-over-dependence} is tight.

If we restrict the space of methods to only thresholding procedures, then results in Section \ref{subsec:point-wise-minimax} state that the phase transition phenomenon --- the 0-1 law in the sense of Corollary \ref{cor:point-wise-minimax} --- is universal in all error models with rapidly varying tails.
This includes $\text{AGG}(\nu)$ models {\it for all} $\nu>0$.
In contrast, models with heavy (regularly varying) tailed errors do not exhibit this phenomenon (see Theorem \ref{thm:heavy-tails}).
We summarize the properties of thresholding procedures in Table \ref{table:role-of-thresholding}. 

%\end{remark}

\begin{table}[ht]
    \centering
    \caption{Properties of thresholding procedures under different error distributions when errors are independent. Properties of the error distributions are listed in brackets.}
    \medskip
    \begin{tabular}{p{45mm}p{40mm}p{40mm}} \toprule
        Thresholding procedure & Bayes optimality &  Phase transition \\ 
        (Error distributions) &  (Log-concave density) & (Rapidly-varying tails) \\ \midrule
        AGG($\nu$), $\nu\ge1$ & Yes (Yes) & Yes (Yes) \\ \cmidrule{1-3}
        AGG($\nu$), $0<\nu<1$ & No (No) & Yes (Yes) \\ \cmidrule{1-3}
        Power laws & No (No) & No (No) \\ \bottomrule
    \end{tabular}
    \label{table:role-of-thresholding}
\end{table}


% It should be remarked that Corollary \ref{cor:log-convex} assumes the existence and knowledge of a maximum signal size $\delta_s$. 
% When maximum signal sizes are unknown a priori, a large observation may be attributed to either a truly large signal, or to the error.
% Common practice has been to attribute large observations to signals, and not errors.

% As an example, consider the linear regression
% \begin{equation} \label{eq:regression}
%  Y = X\mu + \xi,
% \end{equation}
% where $\mu$ is a vector of regression coefficients of interest to be inferred from observations of $X$ and $Y$.
% If the design matrix $X$ is of full column rank, then the OLS estimator of $\mu$ can be formed 
% \begin{equation*}
%     \widehat{\mu} = \left(X'X\right)^{-1}X'Y = \mu + \epsilon,
% \end{equation*}
% where $\epsilon := (X'X)^{-1}X'\xi$.
% Hence we recover the generic problem \eqref{eq:model}. 
% The support recovery problem is therefore equivalent to the fundamental model selection problem.
% Often an investigator calculates the t-scores of each coefficient as 
% \begin{equation} \label{eq:linear-model-selection}
%     \widehat{\mu}(i) \Big/ \widehat{\mathrm{s.e.}}(\widehat{\mu}(i)),
% \end{equation}
% where $\widehat{\mathrm{s.e.}}(\widehat{\mu}(i))$ is the estimated standard error of $\widehat{\mu}(i)$.
% The investigator then chooses indices with large t-scores to enter the model.
% If the errors in model \eqref{eq:regression} are iid Gaussian, the expression \eqref{eq:linear-model-selection} is t-distributed and have power-law tails; the discussion above suggests that this commonplace procedure may be sub-optimal for bounded signals.




