
\preface


%\fbox{From Springer}
%{\color{blue}
%% Please write your preface here
%Use the template \emph{preface.tex} together with the Springer document class SVMono (monograph-type books) or SVMult (edited books) to %style your preface in the Springer layout.

%A preface\index{preface} is a book's preliminary statement, usually written by the \textit{author or editor} of a work, which states its origin, scope,
 %purpose, plan, and intended audience, and which sometimes includes afterthoughts and acknowledgments of assistance. 
%When written by a person other than the author, it is called a foreword. The preface or foreword is distinct from the introduction, which deals with 
%the subject of the work.
%
%Customarily \textit{acknowledgments} are included as last part of the preface.}
 


This text presents a collection of new results and recent developments on the phase transition phenomena in sparse signal problems.
The main theme is the study of the fundamental limits in high-dimensional testing and inference. Since the seminal works
of \cite{ingster1998minimax} and \cite{donoho2004higher}, the subject has received a lot of attention in the literature with important contributions
from \cite{ji2012ups,genovese2012comparison,jin2014optimality,arias2017distribution,butucea2018variable}.  These works (among others) 
 have discovered the presence of fundamental limits in the so-called  {\em needle in haystack} problems, where a sparse signal is observed in 
 high-dimensional additive noise.  In this setting, two archetypal  problems arise -- the {\em signal detection} and {\em support support recovery}.  
 Signal detection is a hypothesis testing problem, which amounts
to testing for the presence of non-zero signal in some (any) of the dimensions.  Support recovery, on the other hand, is an inference problem, which aims to estimate the signal support -- the locations of the non-zero signal components.  The fundamental limits of these problems are studied in the so-called high-dimensional asymptotics regime where the sample size $n$ is fixed and the dimension $p$ of the underlying signal 
grows to infinity.   

From a probabilistic perspective, the above phase transition phenomena are asymptotic zero-one type laws, as $p\to\infty$.  
Namely, consider a sparse signal with  {\em support size} on the order of $p^{1-\beta}$ for some parameter $\beta\in (0,1)$.  
Parameterize the non-zero {\em signal amplitude} by $\Delta(p^r)$, 
for some $r>0$ and a suitable monotone non-decreasing function $\Delta(\cdot)$.  Then, for a broad range of error distributions and
statistical problems, one encounters a sharp transition between the regimes where the problem is solvable and unsolvable depending on 
the signal magnitude $r$ and signal sparsity $\beta$.  More precisely, there exists a boundary function $f(\beta)$ such that if 
the signal magnitudes are {\em above} the boundary, $r> f(\beta)$, then the problem can be solved with vanishing loss as $p\to\infty$, with a suitable statistical procedure.  On the other hand, if the signal is below that same boundary, i.e., $r<f(\beta)$, all statistical procedures fail to  provide a solution with a vanishing loss, as $p\to\infty$. Of course, depending on whether one considers the  detection (testing) or 
support recovery (inference) problems, different loss functions quantify success and failure. The choice of the loss functions is often guided by the 
applications, resulting in a rich picture of phase-transitions (see e.g.\ Figure \ref{fig:phase-Gaussian-errors}).

     
{\bf The contributions of this work.} The fundamental limits of the classic detection problem hinge 
of the analysis of the discrepancy between the {\em null} and {\em alternative} hypotheses e.g., via Hellinger distance.  Thus, perhaps 
for  technical reasons, much of the analysis in the existing literature has been done under the assumption that the additive errors are 
independent and/or Gaussian, or using loss functions unaffected by the dependence such as the Hamming loss.  
In this work, we demonstrate that the support recovery problems, especially {\em exact support recovery}, are best understood from 
the novel perspective of the {\em concentration of maxima} phenomenon in extreme value theory.  It turns out that under a very broad 
range of light-tailed error distributions and under a {\em very} broad range of error dependence structures, the maxima of the errors, 
when rescaled (but not centered!) converge in probability to a positive constant. This concentration property leads to a complete solution 
of the exact  support recovery problem for the broad family of thresholding procedures.  Most if not all existing support estimation 
procedures are types thresholding procedures (see Section \ref{sec:statistical-procedures}).   
That is, the signal support estimate comprises of all components exceeding a suitable (potentially data-dependent)  threshold.  We show, by exploiting concentration of maxima, that thresholding procedures obey a phase-transition, where if the signal is above a 
certain boundary, asymptotically exact recovery is possible while below the boundary all thresholding procedures fail, as $p\to\infty$.  
Remarkably, light-tailed maxima concentrate under very broad and strong dependence.  This is exemplified by our characterization of
the concentration of maxima phenomenon for Gaussian triangular arrays.  For example, in the special case of stationary Gaussian time
series, vanishing auto-covariance is necessary and sufficient for the maxima to concentrate in the same way as independent standard 
normal random variables.   These probabilistic contributions are of independent interest and extend classic work of \cite{berman1964limit}. 
Concentration of maxima is a type of super-concentration phenomenon studied also in \cite{chatterjee2014superconcentration} 
and \cite{tanguy:2015}. The robustness of the concentration of maxima phenomenon to dependence can perhaps explain the universality 
of phase transitions in support recovery problems. 

The use of concentration of maxima phenomenon highlights one core idea in our work, which allows for a first of its kind comprehensive 
treatment of thresholding procedures under very broad error-dependence conditions.  The text involves also a full spectrum of related results 
such as minimax-optimality and finite-sample Bayes optimality in support estimation.  Using different type of loss functions and type I error 
controls, we obtain a rich picture of the exact and approximate support recovery problems in high dimensions.  
Many of these phase transition results have not appeared in previously published literature.  

High-dimensional support recovery problems arise in many modern applications ranging from cybersecurity, theoretical computer science,
to statistical genetics.  Genome-wide association studies (GWAS) in genetics is a particularly natural application, where the asymptotic 
phase-transition results help explain and quantify a previously observed empirical phenomenon of the so-called steep part of the power 
curve.  In the last chapter of this work, we detail this application and highlight future theoretical and practical consequences of our work.


{\bf Target audience.} The original research presented in this text originates from the doctoral dissertation of the first author in the 
Statistics department at the University of Michigan, Ann Arbor.  The main goal of this text is to provide a comprehensive treatment of the
exact and approximate support recovery problems by utilizing existing and newly developed probabilistic tools on concentration of maxima.
The text also provides a self-contained introduction to the state-of-the-art in the dynamic area of phase-transitions in high-dimensional 
testing and inference.  It is accessible to doctoral students in Statistics with background in measure-theoretic probability and  
statistics as well as to researchers in applied fields working with high-dimensional data sets.  The text can be used as a reference and a
supplement to a special topics course on high-dimensional inference.

{\bf Acknowledgements.} The authors gratefully acknowledge the support of their families and all colleagues from the Statistics Department 
at the University of Michigan, Ann Arbor.   Special thanks (in alphabetical order) go to Xuming He, Tailen Hsing, Michalis Kallitsis, Liza Levina, 
Rodderick Little, Ya'acov Ritov, Kerby Shedden, Jinqi Shen, Jonathan Terhorst, Gongjun Xu, \fbox{add more}
The authors were partially supported by the NSF program {\em Algorithms for Threat Detection}.


\comment{

  \fbox{ ADD ATD}

application of t
complete pi phase transitions in support estimation under several important loss functions arising
in applications.  This leads to a rich picture of phase transitions, 

 Our contributions involve a The support 

  for the support to be recovered (as $p\to\infty$) the 

 effectively implies that for the support to be possible to estimate, the set of 
positive signal entries must separate as $p\to\infty$
from the rest of the signal, when the signal magnitude is above a certa is  with with 

one has that 




  and makes the connection to the forces behind these statistcal results --- a probablistic phenomena known as concentration of maxima.
We begin with idealized models where high-dimensional signals are observed with independent additive Gaussian noise.
Under suitable parametrization of the signal sparsity and signal sizes, 
we characterize several new phase transition phenomena in high-dimensional additive error models, and derive the signal sizes necessary and sufficient for statistical procedures to simultaneously control false discovery (in terms of Type I error, {FDR}, or {FWER}) and missed detection (in terms of Type II error, {FNR}, or {FWNR}) in large dimensions.
% Several new phase transition phenomena are characterized.
Specifically, we show that, depending on the signal sparsity and signal sizes, the sum of the statistical risks for false discovery  and missed detection either can be controlled at very low levels in high dimensions, or must be at least one in the limit, for all thresholding procedures.

The phase transition in the case of family-wise type I and type II error control is further generalized to cover non-Gaussian and dependent observations.
We show that the sharp zero-one law continues to hold under a very broad class of error dependence structures, and for a very broad class of error distributions with light, rapidly varying, tails.
The key to this important generalization is a certain {concentration of maxima} phenomenon known as relative stability. 
We demystify the dependence conditions of the phase transition by providing a complete characterization of the relative stability concept for Gaussian triangular arrays in terms of their correlation structures.
% The proof uses classic Sudakov-Fernique and Slepian lemma arguments along with a curious application of Ramsey's coloring theorem. 

Finally, motivated by marginal screenings of categorical variables, we study high-dimensional multiple testing problems where test statistics have approximate chi-square distributions.
Phase transitions are established for support recovery problems in high-dimensional chi-square models.
Remarkably, the degree-of-freedom parameters in the chi-square distributions do not affect the boundaries in all phase transitions.
% Several well-known procedures are shown to attain these boundaries.
We also elucidate on the notion of signal sizes in association tests by characterizing its relationship with marginal frequencies, odds ratio, and sample sizes in $2\times2$ contingency tables. 
This allows us to illustrate an interesting manifestation of the phase transition phenomena in % \ac{GWAS},
genome-wide association studies (GWAS), 
and explain some long-standing empirical observations on the discoverability of signals in GWAS.
As an auxiliary result, we show that, perhaps surprisingly, given total sample sizes, balanced designs in such association studies rarely deliver optimal power. }


\vspace{\baselineskip}
\begin{flushright}\noindent
Chicago \hfill {\it Zheng Gao }\\
Ann Arbor, August 2020\hfill {\it Stilian Stoev}\\
\end{flushright}


