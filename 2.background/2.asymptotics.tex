
We look to derive useful asymptotic approximations for high dimensional problems, and analyze the afore-mentioned procedures in the regime where the dimensionality of the observations diverge.
Throughout this text, we consider triangular arrays of observations as described in Models \eqref{eq:model-additive} and \eqref{eq:model-chisq}, and study the performance of procedures in the signal detection and support recovery tasks when 
\begin{equation*}
  p\to\infty.
\end{equation*}
The criteria for success and failure in support recovery problems under this high-dimensional asymptotic regime are defined as follows.
\begin{definition} \label{def:exact-recovery-success-failure}
We say a sequence of procedures $\mathcal{R} = \mathcal{R}_p$ succeeds asymptotically in the detection problem (and respectively, exact, exact-approximate, approximate-exact, and approximate support recovery problem) if 
\begin{equation} \label{eq:support-recovery-success}
    \mathrm{risk}^{\mathrm{P}}(\mathcal{R}) \to 0, \quad \text{as}\quad p\to\infty,
\end{equation}
where $\mathrm{P}=\mathrm{D}$ (respectively, $\mathrm{E}$, $\mathrm{EA}$, $\mathrm{AE}$, $\mathrm{A}$).

Conversely, we say the exact support recovery fails asymptotically in the detection problem (and respectively, exact, exact-approximate, approximate-exact, and approximate support recovery problem) if 
\begin{equation} \label{eq:support-recovery-failure}
    \liminf\mathrm{risk}^{\mathrm{P}}(\mathcal{R}) \ge 1, \quad \text{as}\quad p\to\infty,
\end{equation}
where $\mathrm{P}=\mathrm{D}$ (respectively, $\mathrm{E}$, $\mathrm{EA}$, $\mathrm{AE}$, $\mathrm{A}$).
\end{definition}
The choice of the constant 1 in the definition \eqref{eq:support-recovery-failure} allows us to declare failure for trivial testing procedures. 
For example, trivial deterministic procedures that always reject, and ones that always fail to reject, both have statistical risks 1 in either the detection or the support recovery problem.
Similarly, a trivial randomized procedure that reject the nulls uniformly at random also has risk of 1, and is declared as a failure in both problems.

% Similarly, we define the criteria for asymptotic success and failure for approximate support recovery as follows.
% \begin{definition} \label{def:approx-recovery-success-failure}
% We say a sequence of procedures $\mathcal{R} = \mathcal{R}_p$ succeeds asymptotically in the approximate support recovery problem if 
% \begin{equation} \label{eq:approx-recovery-success}
%     \mathrm{risk}^{\mathrm{A}}(\mathcal{R}) \to 0, \quad \text{as}\quad p\to\infty.
% \end{equation}
% We say the approximate support recovery fails asymptotically if 
% \begin{equation} \label{eq:approx-recovery-failure}
%     \liminf\mathrm{risk}^{\mathrm{A}}(\mathcal{R}) \ge 1, \quad \text{as}\quad p\to\infty.
% \end{equation}
% \end{definition}

% The performance of procedures in terms of the criteria in Definition \ref{def:exact-recovery-success-failure} 
% % and \ref{def:approx-recovery-success-failure} 
% will be analyzed in Sections \ref{subsec:exact-support-recovery-boundary} and \ref{subsec:approx-support-recovery-boundary}.

{\bf Signal detection.}
The asymptotic behavior of the statistical risk of signal detection problems \eqref{eq:risk-detection} in high dimensions was first studied in \cite{ingster1998minimax}, where a so-called phase transition phenomena was discovered for sparse additive models \eqref{eq:model-additive} with independent and Gaussian components. 
That is, depending on the signal size and sparsity of the signal $\mu$, the detection risk either vanishes, or has a liminf of 1 when we apply the theoretically optimal \ac{LR} test, as dimensionality of the problem diverges.
The \ac{LR} test, unfortunately, relies on the knowledge of signal sparsity and signal sizes which are often unknown.
The sparsity-and-signal-size-agnostic statistic \ac{HC} was identified to attain such optimal performance limits in sparse Gaussian models in \cite{donoho2004higher}. 
A modified goodness-of-fit test statistic in \citet{zhang2002powerful}, and two statistics based on thresholded-$L_1$ and $L_2$ norms in \cite{zhong2013tests} were also shown to be asymptotically optimal in the detection problem.
Recent studies have also focused on the behavior of detection risk \eqref{eq:risk-detection} in dense and scale mixture models \cite{cai2011optimal}, under general distributional assumptions \citep{cai2014optimal, arias2017distribution1}, as well as when errors are dependent \citep{hall2010innovated}.
A comprehensive review focusing on the role of \ac{HC} in detection problems can be found in \cite{donoho2015special}.
Notwithstanding the extensive literature on the detection problem, performances of simple statistics such as $L_q$ norms \eqref{eq:Lq-norm} and sums \eqref{eq:sum-statistic}, to the best of our knowledge, have only been sparingly documented. 
We gather relevant results in Chapter \ref{chap:phase-transitions}, and make several new reports on the performance of statistics commonly used in practice.

\medskip

{\bf Exact support recovery.}
There is a wealth of literature on the so-called sparsistency (i.e., $\P[\widehat{S} = S]\to 1$ as $p\to\infty$) problem in the regression context. 
Sparsistency problems were pursued, among many others, by \citet{zhao2006model} and \citet{wasserman2009high} in the high-dimensional regression setting (where the number of samples $n\ll p$), and by \citet*{meinshausen2006high} in graphical models.
Although there have been numerous studies on the sufficient conditions for sparsistency, efforts on necessary conditions have been scarce.
Notable exceptions include \cite{wainwright2009information, wainwright2009sharp} and \cite{comminges2012tight} in regression problems.
We refer readers to the recent book by \cite{wainwright2019high} (and in particular, the bibliographical sections of Chapters 7 and 15) for a comprehensive review.

Elaborate asymptotic minimax optimality results under the Hamming loss were derived for methods proposed in \cite{ji2012ups} and \cite{jin2014optimality} for regression problems.
More recently, \cite{butucea2018variable} also obtained similar minimax optimality results for a specific procedure in the Gaussian additive error model \eqref{eq:model-additive} in terms of the expected Hamming loss.

Nevertheless, two important questions remained unanswered.
Namely, precise phase-transition-type results for the exact support recovery risk \eqref{eq:risk-exact} and for the support recovery probability \eqref{eq:risk-prob} 
% --- akin to that obtained in \cite{arias2017distribution} for the approximate support recovery risk \eqref{eq:risk-approximate} --- 
have not been established.
And secondly, performance of commonly used statistical procedures reviewed in Section \ref{sec:statistical-procedures}
in terms of these risk metrics have not been studied.
Resolving these two issues, we show in this text that several well-known FWER-controlling procedures --- including Bonferroni's procedure --- are optimal in the additive error model under both one-sided and two-sided alternatives.
% these results have appeared in \cite{gao2018fundamental} and \cite{gao2019five}

\medskip

{\bf Approximate support recovery.}
Performance limits of FDR-controlling procedures in the support recovery problem have been actively studied in recent years.
The asymptotic optimality of the Benjamini-Hochberg procedure
% in the Gaussian scale mixture model 
was analyzed under decision theoretic frameworks in \cite{genovese2002operating, bogdan2011asymptotic, neuvial2012false}, with main focus on location/scale models. 
In particular, these papers show that the statistical risks of the procedures come close to that of the oracle procedures under suitable asymptotic regimes.
Strategies for dealing with multiple testing under general distributional assumptions can be found in, e.g., \cite{efron2004large}, \cite{storey2007optimal}, and \cite{sun2007oracle}.
The two-sided alternative in the additive error model was featured as the primary example in \cite{sun2007oracle}.

In the additive error model \eqref{eq:model-additive} under independent Gaussian errors and one-sided alternatives \eqref{eq:global-test-one-sided},
\cite{arias2017distribution} showed that a phase transition exists for  the approximate support recovery risk \eqref{eq:risk-approximate}.
The \ac{BH} procedure \cite{benjamini1995controlling}, and the Cand\'es-Barber procedure \cite{barber2015controlling} was identified to be asymptotically optimal. % \citet{rabinovich2017optimal} further established the rate-optimality of both procedures under the same regime.
However, \cite{arias2017distribution}, as all related work so far, assumed the non-nulls to follow a common alternative distribution.
We derive a new phase transitions result that relaxes this assumption on the alternatives in Chapter \ref{chap:phase-transitions}.

\medskip

{\bf Asymmetric statistical risks.}
Although weighted sums of false discovery and non-discovery have been studied in the literature mentioned above, none has considered simultaneous family-wise error control and marginal, location-wise power requirements.
As a result, asymmetric statistical risks such as \eqref{eq:risk-exact-approx} and \eqref{eq:risk-approx-exact} have not been investigated.
As argued in Section \ref{sec:risks}, properties of these asymmetric risks are of important practical concern in applications such as GWAS.
We study the asymptotic behavior of these risks in Chapters \ref{chap:phase-transitions} and \ref{chap:GWAS} of this text.
% ; the results therein have appeared in \cite{gao2019five}.
% these results have appeared in \cite{gao2018fundamental} and \cite{gao2019five}

\medskip

{\bf Chi-square models and GWAS.}
The high-dimensional chi-square model \eqref{eq:model-chisq} seemed to have received little attention in the literature.
While the sparse signal detection problem in the chi-square model has been studied \cite{donoho2004higher}, to the best of our knowledge,  asymptotic limits of the support recovery problems have not been studied.
The chi-squared model and the motivating GWAS application are analyzed in Chapter \ref{chap:GWAS}.
The results obtained therein help us understand the sharp power decay in GWAS that have been long observed \citep{bush2012genome} and recently reviewed and visualized \citep{gao2019upass}.

% Results in Chapter \ref{chap:GWAS} now appear in \cite{gao2019five}.
% We also analyzed asymptotic equivalences of several additional common association tests, and implement power calculations for GWAS in a software tool \cite{gao2019upass}. 
% The software streamlines power analysis with a canonical disease model invariant parametrization, and therefore enables forensics of reported findings in genetic association studies.
% We introduce the software and illustrate its use in the appendix.