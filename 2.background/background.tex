
We establish the background necessary for the study of sparse signal detection and support recovery problems in this chapter.
Sections \ref{sec:risks} and \ref{sec:statistical-procedures} provide a refresher on the definitions of statistical risks and some commonly used statistical procedures.
Section \ref{sec:asymptotics} describes the asymptotic regime under which we analyze these procedures, and reviews the related literature in high-dimensional statistics.
We discuss in Section \ref{sec:risks-relations} the connections among the risk metrics, and point out some common fallacies.
The remaining sections collect the technical preparations for this text.
Section \ref{suppsec:AGG} defines an important class of error distributions which will be analyzed in detail in later chapters. 
And finally, Section \ref{subsec:RS} introduces the concepts of concentration of maxima, which plays a crucial role in the analysis of high-dimensional support recovery problems.

% prepare the definitions and technical results used in our analyses.

\section{Statistical risks}
\label{sec:risks}
  
\stilian{To understand the fundamental limits of the signal detection and support recovery problems, one needs to first
define the respective notions of statistical risk.  In this section, we review many existing as well as some new notions 
of statistical risk, motivated by applications. Formally, in both problems, we have a statistical procedure 
${\cal R}={\cal R}(x)$, which is a measurable function of the data.
In the testing context, ${\cal R}$ takes binary values encoding the presence or absence of a signal, while in the support
recovery problem the values of ${\cal R}$ are sets of indices in $\{1,\cdots,p\}$.  The risk will then be a suitable 
functional of ${\cal R}$ accounting for the Type I, Type II, and various mis-classification errors.}{We define the statistical risk metrics for signal detection and signal support recovery problems in this section.}

{\bf Signal detection.} 
Recall that in sparse signal detection problems, our goal is to come up with a procedure, $\mathcal{R}(x)$, such that the null hypothesis is rejected if the data $x$ is deemed incompatible with the null.
In the additive error models context \eqref{eq:model-additive}, we wish to tell apart two hypotheses
\begin{equation} \label{eq:global-test-additive}
    \mathcal{H}_0: \mu(i) = 0, \;i=1,\ldots,p,
    \quad\text{v.s.}\quad 
    \mathcal{H}_1: \mu(i)\neq 0, \; \text{for some }i\in\{1,\ldots,p\},
\end{equation}
based on the $p$-dimensional observation $x$.
Similarly in the chi-square model \eqref{eq:model-chisq}, we look to test
\begin{equation} \label{eq:global-test-chisq}
    \mathcal{H}_0: \lambda(i) = 0, \;i=1,\ldots,p,
    \quad\text{v.s.}\quad 
    \mathcal{H}_1: \lambda(i)\neq 0, \; \text{for some }i\in\{1,\ldots,p\}.
\end{equation} 
Since the decision is binary, we may write the outcome of the procedure in the form of an indicator function, 
$\mathcal{R}(x)\in\{0,1\}$, where ${\cal R}$ takes on value 1 if the null is to be rejected in favor of the alternative, 
and 0 if we fail to reject the null. 
The Type I and Type II errors of the procedure, i.e., the probability of wrong decisions under the null hypothesis $\mathcal{H}_0$ and alternative hypothesis $\mathcal{H}_1$, respectively, are defined as 
\begin{equation} \label{eq:type-II-error}
    \alpha(\mathcal{R}) := \P_{\mathcal{H}_0}\left(\mathcal{R}(x)=1\right)
    \quad \text{and} \quad
    \beta(\mathcal{R}) := \P_{\mathcal{H}_1}\left(\mathcal{R}(x)=0\right).
\end{equation}
The Neyman-Pearson framework of hypothesis testing then seeks tests that minimize the Type II error of the test, while controlling the Type I error of the test at low levels.
We are particularly interested in the sum of the two errors, 
\begin{equation} \label{eq:risk-detection}
    \mathrm{risk}^{\mathrm{D}}(\mathcal{R}) := \alpha(\mathcal{R}) + \beta(\mathcal{R}),
\end{equation}
which shall be referred to as the risk of signal detection (of the procedure $\mathcal{R}$).
It is trivial that a small $\mathrm{risk}^{\mathrm{D}}$ would imply both small Type I and Type II errors of the procedure.

\medskip

{\bf Signal support recovery.}
Turning to support recovery problems, our goal is to design a procedure $\mathcal R$ that produces a set estimate $\widehat{S}= {\cal R}(x)$ of the true index set of relevant variables $S$.
% The true index set, of course, depends on the modeling assumptions. 
For example, in the sparse additive error model \eqref{eq:model-additive} we aim to estimate $S=\{i:\mu(i)\neq 0\}$, while in the sparse chi-square model \eqref{eq:model-chisq} the goal is to estimate $S=\{i:\lambda(i)\neq 0\}$.
\stilian{For simplicity of notation, we shall write $\widehat{S}$ for the support estimation procedure ${\cal R}(x)$.}{\xout{Formally, one should write $\widehat{S}_{{\cal R}}(x)$ to reflect the dependence of the set estimate on the 
procedure $\mathcal{R}$ and on the test statistics $x$; for notational convenience, we suppress this dependence and simply write $\widehat{S}$.
}}% in place of $\widehat{S}(\mathcal{R}(x))$ 

For a given procedure $\mathcal{R}$, its {false discovery rate} (FDR) 
and  {false non-discovery rate} (FNR) are defined, respectively, as
% of the procedure is defined to be the expected fraction of false findings not in the true index set, among the reported discoveries \cite{benjamini1995controlling}. 
% Its counterpart, \emph{false non-discovery rate} (FNR), measuring the power of the procedure, is defined as the expected fraction of missed detection. 
% Mathematically, we define
\begin{equation} \label{eq:FDR-FNR}
    \mathrm{FDR}(\mathcal{R}) := \E\left[\frac{|\widehat{S}\setminus S|}{\max\{|\widehat{S}|,1\}}\right]
    \quad \text{and} \quad
    \mathrm{FNR}(\mathcal{R}) := \E\left[\frac{|S\setminus \widehat{S}|}{\max\{|{S}|,1\}}\right],
\end{equation}
where the maxima in the denominators resolve the possible division-by-0 problem. 
Roughly speaking, FDR measures the expected fraction of false findings, while FNR describes the proportion of Type II errors among the true signals, and reflects the average marginal power of the procedure.

A more stringent criterion for false discovery is the family-wise error rate (\ac{FWER}), defined to be the probability of 
reporting at least one finding not contained 
in the true index set. Correspondingly, a more stringent criterion for false non-discovery is the family-wise non-discovery rate (\ac{FWNR}), i.e., the probability of missing at least one signal in the true index set. That is,
\begin{equation} \label{eq:FWER-FWNR}
    \mathrm{FWER}(\mathcal{R}) := 1 - \P[\widehat{S} \subseteq S]
    \quad \text{and} \quad
    \mathrm{FWNR}(\mathcal{R}) := 1 - \P[S \subseteq \widehat{S}].
\end{equation}

We introduce five different statistical risk metrics, 
each having different asymptotic limits in the support recovery problems as we will see in 
Chapter \ref{chap:phase-transitions}.
Following \cite{arias2017distribution}, we define the risk for \emph{approximate} support recovery as
\begin{equation} \label{eq:risk-approximate}
    \mathrm{risk}^{\mathrm{A}}(\mathcal{R}) := \mathrm{FDR}(\mathcal{R}) + \mathrm{FNR}(\mathcal{R}).
\end{equation}
Analogously, we define the risk for \emph{exact} support recovery as
\begin{equation} \label{eq:risk-exact}
    \mathrm{risk}^{\mathrm{E}}(\mathcal{R}) := \mathrm{FWER}(\mathcal{R}) + \mathrm{FWNR}(\mathcal{R}).
\end{equation}
Two closely related measures of success in the exact support recovery risk are the probability of exact recovery, 
\begin{equation} \label{eq:risk-prob}
    \P[\widehat{S} = S] = 1 - \P[\widehat{S} \neq S],
\end{equation}
and the Hamming loss
\begin{equation} \label{eq:Hamming-loss}
    H(\widehat S, S) := \left|\widehat{S}\triangle S\right|
    = \sum\limits_{i=1}^p\left|\mathbbm{1}_{\widehat{S}}(i)- \mathbbm{1}_{S}(i)\right|.
\end{equation}
which counts the number of mismatches between the estimated and true support sets.

The relationship between probability of support recovery $\P[\widehat{S} = S]$, exact support recovery risk $\mathrm{risk}^{\mathrm{E}}$, and the expected Hamming loss $\mathbb{E}[H(\widehat S, S)]$ will be discussed in Section \ref{sec:risks-relations} below.

\medskip

Notice that all risk metrics introduced so far penalize false discoveries and missed signals somewhat symmetrically --- the approximate support recovery risk combines proportions of errors, the exact support recovery risk combines probabilities of errors, and the Hamming loss increments the risk by one regardless of the types of errors made.
In applications, however, attitudes towards Type I and Type II errors are often different.
In the example of GWAS, where the number of candidate locations $p$ could be in the millions, the researchers are typically interested in the marginal (location-wise) power of discovery, while exercising stringent (family-wise) false discovery control. 
 \stilian{Such types of asymmetric risks, while important in applications, have not been studied theoretically.  The}{ 
 \xout{--- a situation not reflected in the above-mentioned risk metrics.
This asymmetric consideration, and in particular the}} GWAS application, prompts us to consider risks that weigh both the family-wise error rate and the marginal power of discovery.
One such risk metric is what shall be referred to as the \emph{exact-approximate} support recovery risk
\begin{equation} \label{eq:risk-exact-approx}
    \mathrm{risk}^{\mathrm{EA}}(\mathcal{R}) := \mathrm{FWER}(\mathcal{R}) + \mathrm{FNR}(\mathcal{R}).
\end{equation}
The somewhat cumbersome name and notation are chosen to reflect
the asymmetry in dealing with the two types of errors in support recovery.
Namely, when the risk metric \eqref{eq:risk-exact-approx} vanishes, we have ``exact false discovery control, and approximate false non-discovery control'' asymptotically.

Analogously, we consider the \emph{approximate-exact} support recovery risk
\begin{equation} \label{eq:risk-approx-exact}
    \mathrm{risk}^{\mathrm{AE}}(\mathcal{R}) := \mathrm{FDR}(\mathcal{R}) + \mathrm{FWNR}(\mathcal{R}),
\end{equation}
which places more emphasis on non-discovery control over false discovery.

% These two risks differ in their stringency in controlling false discovery and false non-discovery.
Theoretical limits and performance of procedures in support recovery problems will be studied in terms of the five risk metrics \eqref{eq:risk-approximate}, \eqref{eq:risk-exact}, \eqref{eq:risk-prob}, \eqref{eq:risk-exact-approx} and \eqref{eq:risk-approx-exact}, in Chapters \ref{chap:phase-transitions}, \ref{chap:exact-support-recovery}, and \ref{chap:GWAS}.
We are particularly interested in fundamental limits of signal detection and support recovery problems in terms of these metrics, as well as the optimality of commonly used procedures in high dimensional settings. 

% This asymmetry motivates us to study, and discover, two new phase transitions, both in the additive error model \eqref{eq:model-additive} under one-sided alternatives, and in the chi-square model \eqref{eq:model-chisq}.
% The latter, as discussed in Section \ref{subsec:motivation-additive}, entails the additive error model \eqref{eq:model-additive} under two-sided alternatives.
% Further details can be found in Section \ref{subsec:risks} below, after 
% We summarize the main messages of this paper next.

% The third consideration that motivates us is the choice of practically relevant statistical risks.


\section{Statistical procedures}
\label{sec:statistical-procedures}

We review some popular procedures for signal detection and signal support recovery tasks in this section.

{\bf Signal detection.}
One of the commonly used statistics in sparse signal detection problems such as \eqref{eq:global-test-additive} and \eqref{eq:global-test-chisq} are the $L_q$ norms of the observations $x$,
\begin{equation} \label{eq:Lq-norm}
   L_q(x) = \left(\sum_{i=1}^p{|x(i)|^q}\right)^{1/q}.
\end{equation}
Typical choices of $q$ include $q=1, 2$ and $\infty$, where $L_\infty(x)$ is interpreted as the limit of $L_q(x)$ norms as $q\to\infty$, and is equivalent to $\max_{i}|x(i)|$.
Test procedures based on \eqref{eq:Lq-norm} may then be written as $T(\mathcal{R}(x)) = \mathbbm{1}_{(t,+\infty)}(L_q(x))$, where the cutoff $t$ can be chosen to control the Type I error at desired levels.

While \eqref{eq:Lq-norm} measures the deviation of the data from the origin in an omnidirectional manner, statistics that are tailored to the alternatives can be used in the hopes of power improvement if the directions of the alternatives are known.
For example, in the additive error model \eqref{eq:model-additive}, suppose we want to test for positive mean shifts, i.e., one-sided alternative
\begin{equation} \label{eq:global-test-one-sided}
    %\mathcal{H}_0: \mu(i) = 0, \;i=1,\ldots,p,
    %\quad\text{v.s.}\quad 
    \mathcal{H}_1: \mu(i)> 0, \; \text{for some }i\in\{1,\ldots,p\}.
\end{equation}
Then, one might consider monitoring the sum (or equivalently, the arithmetic average) of the observations, 
\begin{equation} \label{eq:sum-statistic}
    T(x) := \sum_{i=1}^p{x(i)},
\end{equation}
or the maximum of the observations,
\begin{equation} \label{eq:max-statistic}
    M(x) := \max_{i=1,\ldots,p}{x(i)}.
\end{equation}
Other tests based on the empirical \ac{CDF} are also available.
Assuming the same one-sided alternative, let 
\begin{equation}
    q(i) = 1 - \sup\{F_i(y)\,:\,y<x(i)\}, \quad i = 1,\ldots,p,
\end{equation}
be the p-values of the individual observations, where $F_i$ is the \ac{CDF} of the $i$-th component $x(i)$ under $\mathcal{H}_0$.
We define empirical \ac{CDF} of the p-values as
\begin{equation}
    \widehat{F}_p(t) = 
    \frac{1}{p} \sum_{i=1}^p \mathbbm{1}_{[0, t]}(q(i)).
\end{equation}
Viewed as random elements in the space of c\`adl\`ag functions with the Skorohod $J_1$ topology, the centered and scaled \ac{CDF}s converge weakly to a Brownian bridge,
\begin{equation*}
    \left\{\sqrt{p}\left(\widehat{F}_p(t) - t\right)\right\}_{t\in[0,1]} 
    % \stackrel{J_1}{\longrightarrow} 
    \implies
    \left\{\mathbb{B}(t)\right\}_{t\in[0,1]},\ \ \ \mbox{ as } p\to\infty,
\end{equation*}
under the global null $\mathcal{H}_0$ and mild continuity assumptions on the $F_i$'s \citep{skorokhod1956limit}. 
Therefore, goodness-of-fit statistics such as Kolmogorov-Smirnov distance \citep{smirnov1948table}, Cramer-von Mises-type statistics \citep{cramer1928composition, anderson1952asymptotic} that measure the departure from this limiting behavior can be used for testing $\mathcal{H}_0$ against $\mathcal{H}_1$.
Of particular interest is the higher criticism (\ac{HC}) statistic, first proposed by \cite{tukey1976lecture},
\begin{equation} \label{eq:HC-statistic}
    HC(x) = 
    \max_{0\le t\le\alpha_0}\frac{\widehat{F}_p(t) - t}{\sqrt{t(1 - t)/p}}.
\end{equation}

Each of the above statistics $L_q$, $S$, $M$, or $HC$, gives rise to a decision rule, whereby the null hypothesis is rejected if 
the statistic exceeds a suitably calibrated threshold. The choice of the threshold is typically determined based on large-sample limit 
theorems. For example, as shown in Theorem 1.1 of \cite{donoho2004higher}, under the null hypothesis
$$
 \frac{HC(x)}{\sqrt{2\log\log(p)}} \longrightarrow  1,\ \ \mbox{ in probability,}
$$ 
as $p\to\infty$.  Thus, one decision rule is to reject ${\cal H}_0$, if $HC(x)> t(p,\alpha_p)$, where 
$t(p,\alpha_p) = \sqrt{2\log\log(p)}(1+o(1))$.  As we will see, this yields an optimal signal detection procedure 
\citep[see also Theorem 1.2 in ][]{donoho2004higher}. The performance of these statistics in high-dimensional 
sparse signal detection problems will be reviewed in Section \ref{sec:asymptotics}, and analyzed in Chapter \ref{chap:phase-transitions}.

\medskip

{\bf Signal support recovery.}
In signal support recovery tasks, we shall study the performance of five procedures, all of which belong to the broad class of thresholding procedures.
\begin{definition}[Thresholding procedures]
A thresholding procedure for estimating the support 
$S:=\{i\, :\, \lambda(i)\neq0\}$ is one that takes on the form
\begin{equation} \label{eq:thresholding-procedure}
    \widehat{S} = \left\{i\,|\,x(i) \ge t(x)\right\},
\end{equation}
where the threshold $t(x)$ may depend on the data $x$.
\end{definition}

Examples of thresholding procedures include ones that aim to control FWER \eqref{eq:FWER-FWNR} --- Bonferroni's \citep{dunn1961multiple}, Sid\'ak's \citep{vsidak1967rectangular}, Holm's \citep{holm1979simple}, and Hochberg's procedure \citep{hochberg1988sharper} --- as well as procedures that target FDR \eqref{eq:FDR-FNR}, such as the Benjamini-Hochberg \cite{benjamini1995controlling} and the Barber-Cand\`es procedure \citep{barber2015controlling}.
Indeed, the class of thresholding procedures \eqref{eq:thresholding-procedure} is so general that it contains most (but not all) of the statistical procedures in the multiple testing literature.
% \cite{roquain2011type}.


% \subsection{FWER-controlling procedures}
% \label{subsec:FWER-controlling-procedures}

Under the assumption that the data $x(i)$'s under the null have a common marginal distribution $F$, we review five thresholding procedures for support recovery, starting with the well-known Bonferroni's procedure which aims at controlling family-wise error rates.
\begin{definition}[Bonferroni's procedure] \label{def:Bonf}
Bonferroni's procedure with level $\alpha$ is the thresholding procedure that uses the threshold
\begin{equation} \label{eq:Bonferroni-procedure}
    t_p = F^{\leftarrow}(1 - \alpha/p).
\end{equation}
where  $F^{\leftarrow}(u)=\inf{\left\{x:F(x)\ge u\right\}}$ is the generalized inverse function.
\end{definition}
% It is easy to see that the family-wise error rate (FWER) is controlled at level $\alpha$ by applying the union bound, regardless of the error-dependence structure (see e.g.\ Relation \eqref{eq:Bonferroni-FWER-control}, below).
The Bonferroni procedure is deterministic, i.e. non data-dependent, and only depends on the dimension of the problem and the null distribution.
A closely related procedure is Sid\'ak's procedure \citep{vsidak1967rectangular},
which is a more aggressive (and also deterministic) thresholding procedure that uses the threshold
\begin{equation} \label{eq:Sidak-procedure}
    t_p = F^{\leftarrow}((1 - \alpha)^{1/p}).
\end{equation}
% can be shown to control FWER in the case independent errors.

The third procedure, strictly more powerful than Bonferroni's, is the so-called Holm's procedure \citep{holm1979simple}.
On observing the data $x$, its coordinates can be ordered from largest to smallest
$x(i_1) \ge x(i_2)  \ge \ldots \ge x(i_p)$,
where $(i_1, \ldots, i_p)$ is a permutation of $\{1, \ldots, p\}$. 
Denote these order statistics as $x_{[1]}, x_{[2]}, \ldots, x_{[p]}$.
\begin{definition}[Holm's procedure]
Let $i^*$ be the largest index such that
$$
\overline{F}(x_{[i]}) \le \alpha / (p-i+1),\quad \text{for all }\;i\le i^*.
$$
Holm's procedure with level $\alpha$ is the thresholding procedure with threshold
\begin{equation} \label{eq:Holm-procedure}
    t_p(x) = x_{[i^*]}.
\end{equation}
\end{definition}
In contrast to the Bonferroni procedure, Holm's procedure is data-dependent.
% It can be shown that Holm's procedure also controls FWER at $\alpha$ level, regardless of dependence in the data.
A closely related, more aggressive (and also data-dependent) thresholding procedure is Hochberg's procedure \citep{hochberg1988sharper}.
%\begin{definition}[Hochberg's procedure]
%Hochberg's procedure 
It replaces the index $i^*$ in Holm's procedure with the largest index such that
$$
\overline{F}(x_{[i]}) \le \alpha / (p-i+1).
$$
Notice that both Holm's and Hochberg's procedures compare p-values to the same thresholds $\alpha / (p-i+1)$.
However, Holm's procedure only rejects the set of hypotheses whose p-values are all smaller than their respective thresholds.
On the other hand, Hochberg's procedure rejects the set of hypotheses as long as the largest of their p-values fall below its threshold, and therefore, can be more powerful than Holm's procedure. 
%where  $\overline{F}(x)=1-F(x)$ is the survival function.
%\end{definition}

It can be shown that both Bonferroni's and Holm's procedures control FWER at their nominal levels, regardless of dependence in the 
data \citep{holm1979simple}. In contrast, Sid\'ak's and Hochberg's procedures control FWER at nominal levels when data are independent \citep{vsidak1967rectangular, hochberg1988sharper}. 

Last but not least, we review the \ac{BH} procedure, which aims at controlling \ac{FDR} in 
\eqref{eq:FDR-FNR}, proposed by  \cite{benjamini1995controlling}.

Recall the order statistics of our observations are: $x_{[1]} \ge x_{[2]}  \ge \ldots \ge x_{[p]}$.
\begin{definition}[Benjamini-Hochberg's procedure] \label{def:BH}
Let $i^*$ be the largest index such that
$$
\overline{F}(x_{[i]}) \le \alpha i/p.
$$
The Benjamini-Hochberg (BH) procedure with level $\alpha$ is the thresholding procedure with threshold
\begin{equation} \label{eq:BH-procedure}
    t_p(x) = x_{[i^*]},
\end{equation}
\end{definition}
The \ac{BH} procedure is shown to control the FDR at level $\alpha$ when the $x(i)$'s are independent \citep{benjamini1995controlling}. 
 Variations of this procedure have been proposed to control the \ac{FDR} under certain models of dependent observations \citep{benjamini2001control}.


% We now turn to discuss the framework for anlayzing the asymptotic risks in high dimensions.

The performance of these procedures in high-dimensional sparse signal support recovery problems will be reviewed in Section \ref{sec:asymptotics}, and analyzed in Chapters \ref{chap:phase-transitions}, \ref{chap:exact-support-recovery}, and \ref{chap:GWAS}.


\section{Related literature and our contributions}
\label{sec:asymptotics}

We look to derive useful asymptotic approximations for high-dimensional problems, and analyze the afore-mentioned procedures in the regime where the dimensionality of the observations diverge.
Throughout this text, we consider triangular arrays of observations as described in Models \eqref{eq:model-additive} and \eqref{eq:model-chisq}, and study the performance of various procedures in the signal detection and support recovery tasks when 
\begin{equation*}
  p\to\infty.
\end{equation*}
The criteria for success and failure in support recovery problems under this high-dimensional asymptotic regime are defined as follows.
\begin{definition} \label{def:exact-recovery-success-failure}
We say a sequence of procedures $\mathcal{R} = \mathcal{R}_p$ succeeds asymptotically in the detection problem (and respectively, exact, exact-approximate, approximate-exact, and approximate support recovery problem) if 
\begin{equation} \label{eq:support-recovery-success}
    \mathrm{risk}^{\mathrm{P}}(\mathcal{R}) \to 0, \quad \text{as}\quad p\to\infty,
\end{equation}
where $\mathrm{P}=\mathrm{D}$ (respectively, $\mathrm{E}$, $\mathrm{EA}$, $\mathrm{AE}$, $\mathrm{A}$).

Conversely, we say the exact support recovery fails asymptotically in the detection problem (and respectively, exact, exact-approximate, approximate-exact, and approximate support recovery problem) if 
\begin{equation} \label{eq:support-recovery-failure}
    \liminf\mathrm{risk}^{\mathrm{P}}(\mathcal{R}) \ge 1, \quad \text{as}\quad p\to\infty,
\end{equation}
where $\mathrm{P}=\mathrm{D}$ (respectively, $\mathrm{E}$, $\mathrm{EA}$, $\mathrm{AE}$, $\mathrm{A}$).
\end{definition}
The choice of the constant 1 in the definition \eqref{eq:support-recovery-failure} allows us to declare failure for trivial testing procedures. 
For example, trivial deterministic procedures that always reject, and ones that always fail to reject, both have statistical risks 1 in either the detection or the support recovery problem.
Similarly, a trivial randomized procedure that reject the nulls uniformly at random also has risk of 1, and is declared as a failure in both problems.

% Similarly, we define the criteria for asymptotic success and failure for approximate support recovery as follows.
% \begin{definition} \label{def:approx-recovery-success-failure}
% We say a sequence of procedures $\mathcal{R} = \mathcal{R}_p$ succeeds asymptotically in the approximate support recovery problem if 
% \begin{equation} \label{eq:approx-recovery-success}
%     \mathrm{risk}^{\mathrm{A}}(\mathcal{R}) \to 0, \quad \text{as}\quad p\to\infty.
% \end{equation}
% We say the approximate support recovery fails asymptotically if 
% \begin{equation} \label{eq:approx-recovery-failure}
%     \liminf\mathrm{risk}^{\mathrm{A}}(\mathcal{R}) \ge 1, \quad \text{as}\quad p\to\infty.
% \end{equation}
% \end{definition}

% The performance of procedures in terms of the criteria in Definition \ref{def:exact-recovery-success-failure} 
% % and \ref{def:approx-recovery-success-failure} 
% will be analyzed in Sections \ref{subsec:exact-support-recovery-boundary} and \ref{subsec:approx-support-recovery-boundary}.

\medskip
{\bf Signal detection.}
\stilian{The asymptotic behavior of the statistical risk in signal detection problems \eqref{eq:risk-detection} in high dimensions was first studied by Yurii Ismailovich Ingster in the context of sparse additive models \eqref{eq:model-additive} with 
independent and Gaussian components. Specifically, \cite{ingster1998minimax} considered the behavior of the 
theoretically optimal likelihood ratio (\ac{LR}) test in the high-dimensional asymptotic regime, where the dimension $p$ grows to infinity. Then, under certain parameterization of the size and sparsity of the signal $\mu$, there are only two possibilities: Either ${\rm risk}^D(LR)$ vanishes, as $p\to\infty$, or $\liminf_{p\to\infty} {\rm risk}^D(LR) =1$.
The precise signal size and sparsity parameterizations as well as the so-called {\em detection boundary} discovered 
by Ingster are described in Chapter \ref{chap:phase-transitions}.}{\fbox{replacing?}
 where a phase transition phenomenon was discovered for 
That is, as $p\to\infty$, depending on the size and sparsity of the signal $\mu$, the detection risk either vanishes, 
or has a liminf of 1 when we apply the theoretically optimal likelihood ratio (\ac{LR}) test. }

The \ac{LR} test, unfortunately, depends  relies on the knowledge of the signal sparsity and signal sizes which are 
not available in practice. The sparsity-and-signal-size-agnostic statistic \ac{HC} in \eqref{eq:HC-statistic} was identified 
to attain such optimal performance limits in sparse Gaussian models in \cite{donoho2004higher}. 
A modified goodness-of-fit test statistic in \citet{zhang2002powerful}, and two statistics based on thresholded-$L_1$ and $L_2$ norms in \cite{zhong2013tests} were also shown to be asymptotically optimal in the detection problem.
Recent studies have also focused on the behavior of detection risk \eqref{eq:risk-detection} in dense and scale mixture models \cite{cai2011optimal}, under general distributional assumptions \citep{cai2014optimal, arias2017distribution1}, as well as when the errors are dependent \citep{hall2010innovated}.
A comprehensive review focusing on the role of \ac{HC} in detection problems can be found in \cite{donoho2015special}.
Notwithstanding the extensive literature on the detection problem, the performances of simple statistics such as 
$L_q$ norms \eqref{eq:Lq-norm} and sums \eqref{eq:sum-statistic}, to the best of our knowledge, have only been 
sparingly documented.  We gather relevant results in Chapter \ref{chap:phase-transitions}, and make several new 
contributions on the performance of several statistics commonly used in practice.

\medskip

{\bf Exact support recovery.}
There is a wealth of literature on the so-called sparsistency (i.e., $\P[\widehat{S} = S]\to 1$ as $p\to\infty$) problem in the regression context. 
Sparsistency problems were pursued, among many others, by \citet{zhao2006model} and \citet{wasserman2009high} in the high-dimensional regression setting (where the number of samples $n\ll p$), and by \citet*{meinshausen2006high} in graphical models.
Although there have been numerous studies on the sufficient conditions for sparsistency, efforts on necessary conditions have been scarce.
Notable exceptions include \cite{wainwright2009information, wainwright2009sharp} and \cite{comminges2012tight} in regression problems.
We refer the reader to the recent book by \cite{wainwright2019high} (and in particular, the bibliographical sections of Chapters 7 and 15, therein) for a comprehensive review.

Elaborate asymptotic minimax optimality results under the Hamming loss were derived for methods proposed in \cite{ji2012ups} and \cite{jin2014optimality} for regression problems.
More recently, \cite{butucea2018variable} also obtained similar minimax optimality results for a specific procedure in the Gaussian additive error model \eqref{eq:model-additive} in terms of the expected Hamming loss.

Nevertheless, two important questions remained unanswered.
Namely, precise phase-transition-type results for the exact support recovery risk 
\eqref{eq:risk-exact} and for the support recovery probability \eqref{eq:risk-prob} 
% --- akin to that obtained in \cite{arias2017distribution} for the approximate support recovery risk \eqref{eq:risk-approximate} --- 
have not been established.
And secondly, performance of commonly used statistical procedures reviewed in Section \ref{sec:statistical-procedures}
in terms of these risk metrics have not been studied.
\stilian{Some of our main contributions in this text are the solutions to these problems, presented in Chapters \ref{chap:phase-transitions} and \ref{chap:exact-support-recovery}, below. Specifically, it turns out that the simple Bonferoni thresholding procedure (among others) is asymptotically optimal for the exact support recovery problem in 
\eqref{eq:model-additive} under broad classes of error distributions. Furthermore, a phase-transition in the exact support recovery 
problem for thresholding procedures is established under broad dependence conditions on the 
errors using the concentration of maxima phenomenon (Chapter \ref{chap:exact-support-recovery}). While the 
optimality of thresholding procedures under dependence remains an open question, in Chapter \ref{sec:optimality}, 
we establish finite-sample Bayes optimality and sub-optimality results for these procedures under independence.  
The landscape of the fundamental statistical limits in support estimation is yet to be fully charted. We conjecture, 
however, that the general concentration of maxima phenomenon will lead to its complete solution under very broad error 
dependence scenarios.  }{
Resolving these two issues, we show in this text that several well-known FWER-controlling procedures --- including Bonferroni's procedure --- are optimal in the additive error model under both one-sided and two-sided alternatives.}
% these results have appeared in \cite{gao2018fundamental} and \cite{gao2019five}

\medskip

{\bf Approximate support recovery.}
The performance limits of FDR-controlling procedures in the support recovery problem have been actively studied in recent years.
The asymptotic optimality of the Benjamini-Hochberg procedure
% in the Gaussian scale mixture model 
was analyzed under decision theoretic frameworks in \cite{genovese2002operating, bogdan2011asymptotic, neuvial2012false}, with main focus on location/scale models. 
In particular, these papers show that the statistical risks of the procedures come close to that of the oracle procedures under suitable asymptotic regimes.
Strategies for dealing with multiple testing under general distributional assumptions can be found in, e.g., \cite{efron2004large}, \cite{storey2007optimal}, and \cite{sun2007oracle}.
The two-sided alternative in the additive error model was featured as the primary example in \cite{sun2007oracle}.

In the additive error model \eqref{eq:model-additive} under independent Gaussian errors and one-sided alternatives \eqref{eq:global-test-one-sided},
\cite{arias2017distribution} showed that a phase transition exists for  the approximate support recovery risk \eqref{eq:risk-approximate}.
The \ac{BH} procedure \cite{benjamini1995controlling}, and the \stilian{DISCUSS}{Is it Candes-Barber or Barber-Candes?}
Cand\`es-Barber procedure \citep{barber2015controlling} was identified to be asymptotically optimal. % \citet{rabinovich2017optimal} further established the rate-optimality of both procedures under the same regime.
However, \cite{arias2017distribution}, as all related work so far, assumed the non-nulls to follow a common alternative distribution.
We derive a new phase transition result that relaxes this assumption on the alternatives in Chapter \ref{chap:phase-transitions}.

\medskip

{\bf Asymmetric statistical risks.}
Although weighted sums of false discovery and non-discovery have been studied in the literature mentioned above, \stilian{the case of simultaneous family-wise error control and marginal, location-wise power requirements, has not been previously considered. }{\fbox{Altered, check the wording.}} As a result, asymmetric statistical risks such as \eqref{eq:risk-exact-approx} and \eqref{eq:risk-approx-exact} have not 
been investigated.
As argued in Section \ref{sec:risks}, the properties of these asymmetric risks are of important practical concern in applications such as GWAS.
We study the asymptotic behavior of these risks in Chapters \ref{chap:phase-transitions} and \ref{chap:GWAS} of this text.
% ; the results therein have appeared in \cite{gao2019five}.
% these results have appeared in \cite{gao2018fundamental} and \cite{gao2019five}

\medskip

{\bf Chi-square models and GWAS.}
The high-dimensional chi-square model \eqref{eq:model-chisq} seemed to have received little attention in the literature.
While the sparse signal detection problem in the chi-square model has been studied \cite{donoho2004higher}, to the best of our knowledge,  asymptotic limits of the support recovery problems have not been studied.
The chi-squared model and the motivating GWAS application are analyzed in Chapter \ref{chap:GWAS}.
\stilian{The results obtained therein help us explain the sharp power decay in GWAS known as the 
{\em steep part of the power curve}.  This empirical fact has long been observed by statistical geneticists \citep{bush2012genome} but so far it had not been mathematically quantified.  \cite{gao2019upass} provide 
further details on the power and design in GWAS as well as an accompanying interactive 
statistical software \citep{gaoUPASS_ShinyApp}.}{\fbox{altered the text, pls check/edit.}}

% Results in Chapter \ref{chap:GWAS} now appear in \cite{gao2019five}.
% We also analyzed asymptotic equivalences of several additional common association tests, and implement power calculations for GWAS in a software tool \cite{gao2019upass}. 
% The software streamlines power analysis with a canonical disease model invariant parametrization, and therefore enables forensics of reported findings in genetic association studies.
% We introduce the software and illustrate its use in the appendix.

\section{Relationships between the asymptotic risks}
\label{sec:risks-relations}
We now elaborate on the relationship between statistical risks, as promised in Section \ref{sec:risks}.
The first lemma concerns the asymptotic relationship between the probability of exact recovery \eqref{eq:risk-prob} and the risk of exact support recovery \eqref{eq:risk-exact}.

\begin{lemma} \label{lemma:risk-exact-recovery-probability}
For any sequence of procedures for support recovery $\mathcal{R} = \mathcal{R}_p$, we have, 
%The probability of exact recovery $\P[\widehat{S} = S]$, and risk of exact support recovery $\mathrm{risk}^{\mathrm{E}}$, defined in \eqref{eq:risk-exact}, are related as follows,
\begin{equation} \label{eq:exact-recovery-implies-risk-0}
    \P[\widehat{S} = S] \to 1 \iff \mathrm{risk}^{\mathrm{E}}(\mathcal{R})\to0,
\end{equation}
and
\begin{equation} \label{eq:failure-recovery-implies-risk-1}
    \P[\widehat{S} = S] \to 0 \implies \liminf\mathrm{risk}^{\mathrm{E}}(\mathcal{R})\ge1,
\end{equation}
as $p\to\infty$. Dependence on $p$ was suppressed for notational convenience.
\end{lemma}


\begin{proof}[Proof of Lemma \ref{lemma:risk-exact-recovery-probability}]
Notice that $\{\widehat{S}=S\}$ implies $\{\widehat{S}\subseteq S\} \cap \{\widehat{S}\supseteq S\}$, therefore we have for every fixed $p$,
\begin{equation} \label{eq:risk-exact-recovery-probability-proof-1}
    \mathrm{risk}^{\mathrm{E}} 
    = 2 - \P[\widehat{S} \subseteq S] - \P[S \subseteq \widehat{S}] \\
    \le 2 - 2\P[\widehat{S}=S].
\end{equation}
On the other hand, since $\{\widehat{S}\neq S\}$ implies $\{\widehat{S}\not\subseteq S\} \cup \{\widehat{S}\not\supseteq S\}$, we have for every fixed $p$,
\begin{equation} \label{eq:risk-exact-recovery-probability-proof-2}
    1 - \P[\widehat{S}=S]
    = \P[\widehat{S}\neq S]
    \le 2 - \P[\widehat{S} \subseteq S] - \P[S \subseteq \widehat{S}]
    = \mathrm{risk}^{\mathrm{E}}. 
\end{equation}
Relation \eqref{eq:exact-recovery-implies-risk-0} follows from \eqref{eq:risk-exact-recovery-probability-proof-1} and \eqref{eq:risk-exact-recovery-probability-proof-2}, and Relation \eqref{eq:failure-recovery-implies-risk-1} from \eqref{eq:risk-exact-recovery-probability-proof-2}.
\end{proof}

By virtue of Lemma \ref{lemma:risk-exact-recovery-probability}, it is sufficient to study the probability of exact support recovery $\P[\widehat{S}=S]$ in place of $\mathrm{risk}^{\mathrm{E}}$, if we are interested in the asymptotic properties of the risk in the sense of \eqref{eq:support-recovery-success} and \eqref{eq:support-recovery-failure}.
% (The converse, discussed in Section \ref{sec:discussions} below, is not true.)

\medskip

Keen readers must have noticed the asymmetry in Relation \eqref{eq:failure-recovery-implies-risk-1} when we discussed the relationship between the exact support recovery risk \eqref{eq:risk-exact} and the probability of exact support recovery \eqref{eq:risk-prob}.

%The converse of \eqref{eq:failure-recovery-implies-risk-1} is not true.
While a trivial procedure that never rejects and a procedure that always rejects both have $\mathrm{risk}^{\mathrm{E}}$ equal to 1, the converse is not true.
For example, it is possible that a procedure selects the true index set $S$ with probability $1/2$, but makes one false inclusion \emph{and} one false omission simultaneously the other half of the time. 
In this case the procedure will have 
$$\mathrm{risk}^{\mathrm{E}} = 1, \quad \text{and} \quad \P[\widehat{S}=S] = 1/2,$$
showing that the converse of Relation \eqref{eq:failure-recovery-implies-risk-1} is in fact false.

The same argument applies to $\mathrm{risk}^{\mathrm{A}}$:
a procedure may select the true index set $S$ with probability $1/2$, but makes enough false inclusions and omissions other half of the time, so that $\mathrm{risk}^{\mathrm{A}}$ is equal to one.
Therefore, although the class of methods with risks equal to or exceeding 1 certainly contains the trivial procedures that we mentioned, they are not necessarily ``useless'' as some researchers have claimed (c.f., \citet{arias2017distribution}, Remark 2).

\medskip

Upper and lower bounds for \ac{FDR} and \ac{FNR} can be immediately derived by replacing the numerators in \eqref{eq:FDR-FNR} with the Hamming loss,
\begin{equation} \label{eq:Hamming-loss-FDR-FNR-bound}
    \E\left[\frac{H(\widehat{S}, S)}{\max\{|\widehat{S}|, |S|, 1\}}\right] 
    \le \mathrm{FDR} + \mathrm{FNR}
    \le \E\left[\frac{H(\widehat{S}, S)}{\max\{\min\{|\widehat{S}|, |S|\}, 1\}}\right].
\end{equation}
Therefore, it is sufficient, but not necessary, that the Hamming loss vanish in order to have vanishing approximate support recovery risks \eqref{eq:risk-approximate}.

\medskip

Turning to the relationship between the probability of exact support recovery \eqref{eq:risk-prob} and Hamming loss \eqref{eq:Hamming-loss}, we point out a natural lower bound of the former using the expectation of the latter,
\begin{equation} \label{eq:Hamming-loss-lower-bound}
    \mathbb{P}[\widehat S = S] 
    \ge 1 - \mathbb{E}[H(\widehat S, S)]
    = 1 - \sum\limits_{i=1}^p\E\left|\mathbbm{1}_{\widehat{S}}(i)- \mathbbm{1}_{S}(i)\right|.
\end{equation}
A key observation in Relation \eqref{eq:Hamming-loss-lower-bound} is that the expected Hamming loss decouples into $p$ terms, and dependence of the estimates $\mathbbm{1}_{\widehat{S}}(i)$ among the $p$ locations no longer plays a role in the sum.
Therefore, studying support recovery problems via the expected Hamming loss is not very informative especially under severe dependence, as the bound \eqref{eq:Hamming-loss-lower-bound} may become {very} loose.
Vanishing Hamming loss is again sufficient, but not necessary for 
$\P[\widehat{S}=S]$ or the exact support recovery risk to fo to zero.
% the probability of exact support recovery --- and hence the exact support recovery risk --- to go to zero.



\section{The asymptotic generalized Gaussian (AGG) models}
\label{suppsec:AGG}

We introduce a fairly general class of distributions known as asymptotic generalized Gaussians \ac{AGG}. 
We also state some of their tail properties which play important roles in the analysis of phase transitions of high-dimensional testing problems.

\begin{definition} \label{def:AGG}
A distribution $F$ is called \index{asymptotic generalized Gaussian}{asymptotic generalized Gaussian} 
with parameter $\nu>0$ (denoted $\text{AGG}(\nu)$) if
\begin{enumerate}
    \item $F(x)\in(0,1)$ for all $x\in\R$, and \smallskip
    \item $\log{\overline{F}(x)} \sim -\frac{1}{\nu}x^\nu$ and $\log{F(-x)} \sim -\frac{1}{\nu}(-x)^\nu,$ \label{eq:AGG}
\end{enumerate}
where $\overline{F}(x) = 1 - F(x)$ is the survival function, and $a(x)\sim b(x)$ is taken to mean $\lim_{x\to\infty} a(x)/b(x) = 1$.
\end{definition}

The AGG models include, for example, the standard Gaussian distribution ($\nu = 2$) and the Laplace distribution ($\nu = 1$) as special cases. 
Since the requirement is only placed on the tail behavior, this class encompasses a large variety of light-tailed models. 
This class is commonly used in the literature on high-dimensional testing
 \citep{cai2007estimation, arias2017distribution}.


% On the other hand, the AGG models are themselves special cases of a more general class of tail models as we will see in Example \ref{exmp:AGG}.
% For simplicity of exposition, however, we shall focus on the $\text{AGG}(\nu)$ distributions, where the quantiles have explicit expressions.

\begin{proposition} \label{prop:quantile}
The $(1/p)$-th upper quantile of $\text{AGG}(\nu)$ is
\begin{equation} \label{eq:AGG-quantiles}
    u_{p} := F^\leftarrow(1-1/p) \sim \left(\nu\log{p}\right)^{1/\nu},\quad \text{as }\;p\to\infty,
\end{equation}
where $F^\leftarrow(q) = \inf_x\{x:F(x)\ge q\},\ q\in (0,1)$.
\end{proposition}
% The proof of Proposition \ref{prop:quantile} can be found in Section \ref{suppsec:AGG} of the supplement \citep{gao2018supplement}.
% We remark here that the family of density is log-concave for $\nu \ge 1$, therefore the oracle procedure is the thresholding procedure (See Section \ref{subsec:oracle}).


% \begin{proposition}[$(1 - 1/p)$-th quantile] \label{prop:quantile}
% Let 
% \begin{equation} \label{eq:quantiles-Appendix}
% u_p = F^{\leftarrow}(1-1/p),    
% \end{equation}
% then,
% \begin{equation}
% \lim_{p\to\infty} \frac{u_p}{\left(\nu\log{p}\right)^{1/\nu}} = 1.
% \end{equation}
% \end{proposition}
% 
\begin{proof}[Proof of Proposition \ref{prop:quantile}]
By the definition of AGG, for any $\epsilon>0$, there is a constant $C=C(\epsilon)$ such that for all $x\ge C$, we have
$$
-\frac{1}{\nu}x^\nu(1+\epsilon) \le \log{\overline{F}(x)} \le -\frac{1}{\nu}x^\nu(1-\epsilon).
$$
Therefore, for all $x < x_l := \left((1+\epsilon)^{-1}\nu\log{p}\right)^{1/\nu}$, we have
\begin{equation} \label{eq:AGG-quantiles-proof-1}
    -\log{p} = -\frac{1}{\nu}x_l^\nu(1+\epsilon) \le \log{\overline{F}(x_l)} \le \log{\overline{F}(x)},
\end{equation}
and for all $x > x_u := \left((1-\epsilon)^{-1}\nu\log{p}\right)^{1/\nu}$, we have
\begin{equation} \label{eq:AGG-quantiles-proof-2}
    \log{\overline{F}(x)} \le \log{\overline{F}(x_u)} \le -\frac{1}{\nu}x_u^\nu(1-\epsilon) = -\log{p}.
\end{equation}
By definition of generalized inverse,
\begin{equation*}
    u_p := F^\leftarrow(1-1/p) = \inf\{x:\overline{F}(x)\le 1/p\} = \inf\{x:\log{\overline{F}(x)} \le -\log{p}\}.
\end{equation*}
We know from relations \eqref{eq:AGG-quantiles-proof-1} and \eqref{eq:AGG-quantiles-proof-2} that 
$$
[x_u, +\infty) \subseteq \{x:\log{\overline{F}(x)} \le -\log{p}\} \subseteq [x_l, +\infty),
$$
and so $x_l\le u_p \le x_u$, and the expression for the quantiles follows.
\end{proof}



\section{Rapid variation and relative stability}
\label{subsec:RS}

The behavior of the maxima of identically distributed random variables has been studied extensively in the literature (see, e.g., \cite{leadbetter2012extremes,resnick2013extreme,embrechts2013modelling,de2007extreme} 
and the references therein). 
% We show in this subsection that the maxima of errors with rapidly varying tails can be bounded above using quantiles of their marginal distribution, regardless of their dependence structure; this was a key step in the proof of Theorem \ref{thm:sufficient}.
The concept of rapid variation plays an important role in the light-tailed case.

\begin{definition}[Rapid variation] \label{def:rapid-variation}
The survival function of a distribution, $\overline{F}(x) = 1 - F(x)$, is said to be rapidly varying if
\begin{equation}\label{e:def:rapid-variation}
\lim_{x\to\infty} \frac{\overline{F}(tx)}{\overline{F}(x)} 
    = \begin{cases}
    0, & t > 1\\
    1, & t = 1\\
    \infty, & 0 < t < 1
\end{cases}.
\end{equation}
\end{definition}

When $F(x)<1$ for all finite $x$, \citet{gnedenko1943distribution} showed that the distribution $F$ has rapidly varying tails if and only if the maxima of independent observations from $F$ are \emph{relatively stable} in the following sense.
\begin{definition}[Relative stability] \label{def:RS}
Let $\epsilon_p = \left(\epsilon_p(i)\right)_{i=1}^p$ be a sequence of random variables with identical marginal distributions $F$. 
Define the sequence $(u_p)_{p=1}^\infty$ to be the $(1-1/p)$-th generalized quantile of $F$, i.e., 
\begin{equation} \label{eq:quantiles}
    u_p = F^\leftarrow(1 - 1/p).
\end{equation}
The triangular array ${\cal E} = \{\epsilon_p, p\in\N\}$ is said to have relatively stable (RS) maxima if
\begin{equation} \label{eq:RS-condition}
    \frac{1}{u_{p}} M_p := \frac{1}{u_{p}} \max_{i=1,\ldots,p} \epsilon_p(i) \xrightarrow{\P} 1,\quad \text{as }\;p\to\infty.
\end{equation}
\end{definition}

In the case of independent and identically distributed $\epsilon_p(i)$'s, \citet{barndorff1963limit} and \citet{resnick1973almost} obtained necessary and sufficient conditions for the \emph{almost sure stability} of maxima, where the convergence in \eqref{eq:RS-condition} holds almost surely.

While relative stability (and almost sure stability) is well-understood in the independent case, the role of dependence has not been fully explored.
We start this investigation with a small refinement of Theorem 2 in \citet{gnedenko1943distribution} valid under arbitrary dependence.

\begin{proposition}[Rapid variation and relative stability] \label{prop:rapid-varying-tails}
Assume that the array ${\cal E}$ consists of identically distributed random 
variables with cumulative distribution function $F$, where $F(x)<1$ for all finite $x>0$. 
\begin{enumerate}
    \item If $F$ has rapidly varying right tail, then for all $\delta>0$,
        \begin{equation} \label{eq:rapid-varying-tails}
            \P\left[\frac{1}{u_p} M_p\le1+\delta\right] \ge 1 - \frac{\overline F((1+\delta)u_p)}{\overline F(u_p)} \to 1.
        \end{equation}
    \item If, in addition, the array ${\cal E}$ has independent entries, then it is relatively stable if and only if $F$ has rapidly varying tail.
    \label{prop:rapid-varying-tails_part-ii}
\end{enumerate}
\end{proposition}

\begin{proof}[Proof of Proposition \ref{prop:rapid-varying-tails}] 
By the union bound and the fact that 
$p\overline F(u_p) \le 1$, we have
\begin{align}\label{e:prop:rapid-varying-tails_part-i-1}
\P [ M_p > (1+\delta)u_p] \le p \overline F((1+\delta)u_p)
 \le \frac{\overline F((1+\delta)u_p)}{\overline F(u_p)}.
\end{align}
In view of \eqref{e:def:rapid-variation} (rapid variation) and the fact that $u_p\to\infty$, as $p\to\infty$, the right-hand side of \eqref{e:prop:rapid-varying-tails_part-i-1} vanishes 
as $p\to\infty$, for all $\delta>0$.  This completes the proof of \eqref{eq:rapid-varying-tails}. Part 2 is a re-statement of the classic result due to Gnedenko in \cite{gnedenko1943distribution}.
\end{proof}

We next demonstrate that Gaussian, Exponential, Laplace, and Gamma distributions all have rapidly varying tails. 

% \begin{corollary} \label{cor:AGG-is-RS}
% If $F\in\text{AGG}(\nu)$, $\nu>0$, an independent array ${\cal E}$ is relatively stable. Further, $\text{AGG}(\nu)$ is the only class of model with $u_{p} \sim \left(\nu\log{p}\right)^{1/\nu}$.
% \end{corollary}
% 
% \begin{proof}[Proof of Corollary \ref{cor:AGG-is-URS}]
% By Proposition \ref{prop:rapid-varying-tails}, it is enough to show that in the AGG model,  $\overline{F}$ has rapidly varying tail. 
% By definition of the AGG tails \eqref{def:AGG}, we have
% $$
% \lim_{t\to\infty} \frac{\log{\left(\overline{F}(tx)\Big/\overline{F}(t)\right)}}{-\frac{1}{\nu}t^\nu(x^\n% u-1)} = 1,
% $$
% where the denominator tends to $+\infty$ or $-\infty$ depending on whether $0<x<1$ or $x>1$.
% Therefore, we must have $\overline{F}(tx) / \overline{F}(t)$ converging to $\infty$ or 0 in the correct range of $x$'s; the case where $x=1$ is trivial.
% The last claim follows from the expression for AGG quantiles; see Proposition \ref{prop:quantile}.
% \end{proof}

\begin{example}[Generalized AGG] \label{exmp:AGG}
A distribution is said to have \emph{Generalized AGG} right tail, if $\log{\overline{F}}$ is regularly varying,
\begin{equation} \label{eq:GAGG}
    \log{\overline{F}(x)} = - x^\nu L(x),
\end{equation}
where $\nu>0$ and $L: (0,+\infty)\to(0,+\infty)$ is a slowly varying function. (A function is said to be slowly varying if $\lim_{x\to\infty}L(tx)/L(x) = 1$ for all $t>0$.) Note that the $\text{AGG}(\nu)$ model corresponds to the special case where $L(x)\to 1/\nu$, as $x\to\infty$.

Relation \eqref{eq:rapid-varying-tails} holds for all arrays $\cal E$ with \emph{generalized} AGG marginals; if the entries are independent, the maxima are relatively stable. 
This follows directly from Proposition \ref{prop:rapid-varying-tails}, once we show that $F$ has rapidly varying tail. 
Indeed, by \eqref{eq:GAGG}, we have
$$
\log{\left(\overline{F}(tx)\Big/ \overline{F}(x)\right)} = - L(x)x^\nu\left(t^\nu\frac{L(tx)}{L(x)} - 1\right),
$$
which converges to $-\infty$, 0, and $+\infty$, as $x\to\infty$, when $t>1$, $t=1$, and $t<1$, respectively, since $x^\nu L(x)\to\infty$ as $x\to\infty$ by definition of $L$.
\end{example}


The \ac{AGG} class encompasses a wide variety of rapidly varying tail models such as Gaussian and double exponential distributions. The larger class \eqref{eq:GAGG} is needed, however, for the Gamma distribution.

More generally, distributions with heavier tails (e.g., log-normal) and lighter tails (e.g., Gompertz) outside the generalized AGG class \eqref{eq:GAGG} may also possess rapidly varying tails;
heavy-tailed distributions like the Pareto and t-distributions, on the other hand, do not.
These alternative classes of models are will be introduced when we study the phase transitions in Chapter \ref{chap:exact-support-recovery}. 
%For brevity, we focus here on the $\text{AGG}(\nu)$ models.


\section{Exercises}

\fbox{ TO DO: Verify and add}

\begin{enumerate}

 \item (Inspired by \cite{resnick1973almost}) Let $\epsilon(i)\sim F,\ i=1,2,\cdots$ have arbitrary dependence 
 and $u_p = F^{\leftarrow}(1-1/p)$. 
 
 {\bf (a)}   If for some $\delta_p\to 0$,
 we have $\sum_{p} \overline{F}((1+\delta_p)u_p) <\infty$, then show that
 $$
 \limsup_{p\to\infty} \frac{M_p}{u_p} \le 1,\ \ \mbox{ almost surely,} 
 $$
 where $M_p:= \max_{i=1,\cdots,p} \epsilon(i)$.\\
 
 {\bf (b)} Show that the condition of part {\bf (a)} holds for the Generalized AGG$(\nu),\ \nu>0$ distributions.\\
 
 {\em Hint:} In part {\bf (a)}, argue that the events $\{ M_p > (1+\delta_p) u_p,\ \mbox{ infinitely often}\}$ and
  $\{\epsilon(p) > )(1+\delta_p)u_p, \ \mbox{infinitely often}\}$ are equal. Appeal to the Borel Zero-One Law.\\
 
 \item  Show that independent realizations from a Generalized AGG$(\nu),\ \nu>0$ distribution
  are {\em almost surely stable}. That is, let $\epsilon(i)$'s be independent with distribution function $F$ as in Example \ref{exmp:AGG}.\\

{\bf (a)} Show that there exists a sequence $\delta_p\to 0$, such that 
$$
\sum_p \overline F((1+\delta_p)u_p) <\infty\ \ \mbox{ and }\ \  \sum_p F^p((1-\delta_p)u_p) <\infty.
$$
 {\bf (b)} By appealing to the Borel Zero-One law, conclude that in this case
 $$
 \frac{1}{u_p} M_p {\longrightarrow} 1,\ \ \mbox{ almost surely}.
 $$

{\em Comment:} 
\cite{resnick1973almost} provide a general necessary and sufficient condition for almost sure stability of independent maxima.\\
 
 \item Prove Gnedenko's result in part (2) of Proposition \ref{prop:rapid-varying-tails}.\\
 
\item Suppose that $F$ is heavy-tailed, i.e., $\overline F(x) \sim x^{-\alpha} L(x),$ as $x\to\infty$, for some $\alpha>0$ and a
slowly varying function $L$.  That is, $L(tx)/L(t)\to 1$, as $t\to\infty$, for all $x>0$.\\

{\bf (a)} Show that if $u_p$ is such that $p\overline F(u_p)\to 1$ as $p\to\infty$, then $p \overline F(x u_p)\to x^{-\alpha}$.\\

{\bf (b)} Let $\epsilon(i)$'s be independent realizations from $F$. With $u_n$ as in part {\bf (a)}, show that
$$ 
\frac{M_p}{u_p} \stackrel{d}{\longrightarrow} Z,\ \ \mbox{ as }p\to\infty,
$$ 
where $M_p = \max_{i=1,\cdots,p} \epsilon(i)$ and $\P(Z\le x) = e^{-1/x^\alpha},\ x>0$.\\

{\em Comment:} In the heavy-tailed case, rather than concentration of maxima (relative stability),
we encounter {\em dispersion of maxima} where under rescaling $M_p$ converges in distribution to a proper random variable.\\

\item Let  $M_p,\ p=1,2,\cdots$  be an infinite sequence of random variables such that
$$
\frac{M_p}{u_p} \stackrel{\P}{\longrightarrow} 1\ \ \mbox{ and }\ \ \frac{M_p}{v_p} \stackrel{d}{\longrightarrow} \xi,
$$
for some deterministic positive sequences  $u_p$ and $v_p$ and some non-zero random variable $\xi$.

Show that $\lim_{p\to\infty} u_p/v_p = c$ and $\xi = c$, almost surely.\\

{\em Hint:} Using Skorokhod's theorem there exist $\widetilde M_p$ and $\widetilde \xi$ on another probability space
such that $M_p/v_p\stackrel{d}{=}\widetilde M_p/v_p \stackrel{a.s.}{\to} \widetilde \xi \stackrel{d}{=}\xi$.
Then, recall the following criterion.

\begin{lemma} We have $\eta_p\stackrel{\P}{\to} \eta,\ p\to\infty$ if and only if, for every sequence $p_n\to \infty$, there is a further
subsequence $\{\widetilde p_n\}\subset\{p_n\}$, such that $\eta_{\widetilde p_n} \stackrel{a.s.}{\to} \eta$ almost surely, as $\widetilde p_n\to
\infty$.
\end{lemma}

\item Prove that the Benjamini-Hochberg procedure in Definition \ref{def:BH} controls FDR for independent data.\\

{\em Hint:} \fbox{ Give a good hint. }

\item  
\end{enumerate}


\section{Exercises}
\label{sec:Exercises-Background}
\fbox{ TO DO: Verify and add}

\begin{enumerate}

 \item (Inspired by \cite{resnick1973almost}) Let $\epsilon(i)\sim F,\ i=1,2,\cdots$ have arbitrary dependence 
 and $u_p = F^{\leftarrow}(1-1/p)$. 
 
 {\bf (a)}   If for some $\delta_p\to 0$,
 we have $\sum_{p} \overline{F}((1+\delta_p)u_p) <\infty$, then show that
 $$
 \limsup_{p\to\infty} \frac{M_p}{u_p} \le 1,\ \ \mbox{ almost surely,} 
 $$
 where $M_p:= \max_{i=1,\cdots,p} \epsilon(i)$.\\
 
 {\bf (b)} Show that the condition of part {\bf (a)} holds for the Generalized AGG$(\nu),\ \nu>0$ distributions.\\
 
 {\em Hint:} In part {\bf (a)}, argue that the events $\{ M_p > (1+\delta_p) u_p,\ \mbox{ infinitely often}\}$ and
  $\{\epsilon(p) > (1+\delta_p)u_p, \ \mbox{infinitely often}\}$ are equal. Appeal to the Borel Zero-One Law.\\
 
 \item  Show that independent realizations from a Generalized AGG$(\nu),\ \nu>0$ distribution
  are {\em almost surely stable}. That is, let $\epsilon(i)$'s be independent with distribution function $F$ as in Example \ref{exmp:AGG}.\\

{\bf (a)} Show that there exists a sequence $\delta_p\to 0$, such that 
$$
\sum_p \overline F((1+\delta_p)u_p) <\infty\ \ \mbox{ and }\ \  \sum_p F^p((1-\delta_p)u_p) <\infty.
$$
 {\bf (b)} By appealing to the Borel Zero-One law, conclude that in this case
 $$
 \frac{1}{u_p} M_p {\longrightarrow} 1,\ \ \mbox{ almost surely}.
 $$

{\em Comment:} 
\cite{resnick1973almost} provide a general necessary and sufficient condition for almost sure stability of independent maxima.\\
 
 \item Prove Gnedenko's result in part (2) of Proposition \ref{prop:rapid-varying-tails}.\\
 
\item Suppose that $F$ is heavy-tailed, i.e., $\overline F(x) \sim x^{-\alpha} L(x),$ as $x\to\infty$, for some $\alpha>0$ and a
slowly varying function $L$.  That is, $L(tx)/L(t)\to 1$, as $t\to\infty$, for all $x>0$.\\

{\bf (a)} Show that if $u_p$ is such that $p\overline F(u_p)\to 1$ as $p\to\infty$, then $p \overline F(x u_p)\to x^{-\alpha}$.\\

{\bf (b)} Let $\epsilon(i)$'s be independent realizations from $F$. With $u_n$ as in part {\bf (a)}, show that
$$ 
\frac{M_p}{u_p} \stackrel{d}{\longrightarrow} Z,\ \ \mbox{ as }p\to\infty,
$$ 
where $M_p = \max_{i=1,\cdots,p} \epsilon(i)$ and $\P(Z\le x) = e^{-1/x^\alpha},\ x>0$.\\

{\em Comment:} In the heavy-tailed case, rather than concentration of maxima (relative stability),
we encounter {\em dispersion of maxima} where under rescaling $M_p$ converges in distribution to a proper random variable.\\

\item Let  $M_p,\ p=1,2,\cdots$  be an infinite sequence of random variables such that
$$
\frac{M_p}{u_p} \stackrel{\P}{\longrightarrow} 1\ \ \mbox{ and }\ \ \frac{M_p}{v_p} \stackrel{d}{\longrightarrow} \xi,
$$
for some deterministic positive sequences  $u_p$ and $v_p$ and some non-zero random variable $\xi$.

Show that $\lim_{p\to\infty} u_p/v_p = c$ and $\xi = c$, almost surely.\\

{\em Hint:} Using Skorokhod's theorem there exist $\widetilde M_p$ and $\widetilde \xi$ on another probability space
such that $M_p/v_p\stackrel{d}{=}\widetilde M_p/v_p \stackrel{a.s.}{\to} \widetilde \xi \stackrel{d}{=}\xi$.
Then, recall the following criterion.

\begin{lemma} We have $\eta_p\stackrel{\P}{\to} \eta,\ p\to\infty$ if and only if, for every sequence $p_n\to \infty$, there is a further
subsequence $\{\widetilde p_n\}\subset\{p_n\}$, such that $\eta_{\widetilde p_n} \stackrel{a.s.}{\to} \eta$ almost surely, as $\widetilde p_n\to
\infty$.
\end{lemma}

\item Prove that the Benjamini-Hochberg procedure in Definition \ref{def:BH} controls FDR for independent data.\\

{\em Hint:} \fbox{ Give a good hint. }

\item  
\end{enumerate}


