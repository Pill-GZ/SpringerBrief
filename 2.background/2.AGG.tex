
We introduce a fairly general class of distributions known as \ac{AGG} in this section, and state some properties about the tails of the 
\index{AGG}{AGG} distributions.

\begin{definition} \label{def:AGG}
A distribution $F$ is called \index{asymptotic generalized Gaussian}{asymptotic generalized Gaussian} 
with parameter $\nu>0$ (denoted $\text{AGG}(\nu)$) if
\begin{enumerate}
    \item $F(x)\in(0,1)$ for all $x\in\R$, and \smallskip
    \item $\log{\overline{F}(x)} \sim -\frac{1}{\nu}x^\nu$ and $\log{F(-x)} \sim -\frac{1}{\nu}(-x)^\nu,$ \label{eq:AGG}
\end{enumerate}
where $\overline{F}(x) = 1 - F(x)$ is the survival function, and $a(x)\sim b(x)$ is taken to mean $\lim_{x\to\infty} a(x)/b(x) = 1$.
\end{definition}

The AGG models include, for example, the Gaussian distribution ($\nu = 2$), and the Laplace distribution ($\nu = 1$) as special cases. 
Since the requirement is only placed on the tail behavior, this class encompasses a large variety of light-tailed models, and is commonly used in the literature on high-dimensional testing \citep{cai2007estimation, arias2017distribution}.


% On the other hand, the AGG models are themselves special cases of a more general class of tail models as we will see in Example \ref{exmp:AGG}.
% For simplicity of exposition, however, we shall focus on the $\text{AGG}(\nu)$ distributions, where the quantiles have explicit expressions.

\begin{proposition} \label{prop:quantile}
The $(1/p)$-th upper quantile of $\text{AGG}(\nu)$ is
\begin{equation} \label{eq:AGG-quantiles}
    u_{p} := F^\leftarrow(1-1/p) \sim \left(\nu\log{p}\right)^{1/\nu},\quad \text{as }\;p\to\infty,
\end{equation}
where $F^\leftarrow(q) = \inf_x\{x:F(x)\ge q\},\ q\in (0,1)$.
\end{proposition}
% The proof of Proposition \ref{prop:quantile} can be found in Section \ref{suppsec:AGG} of the supplement \citep{gao2018supplement}.
% We remark here that the family of density is log-concave for $\nu \ge 1$, therefore the oracle procedure is the thresholding procedure (See Section \ref{subsec:oracle}).


% \begin{proposition}[$(1 - 1/p)$-th quantile] \label{prop:quantile}
% Let 
% \begin{equation} \label{eq:quantiles-Appendix}
% u_p = F^{\leftarrow}(1-1/p),    
% \end{equation}
% then,
% \begin{equation}
% \lim_{p\to\infty} \frac{u_p}{\left(\nu\log{p}\right)^{1/\nu}} = 1.
% \end{equation}
% \end{proposition}
% 
\begin{proof}[Proof of Proposition \ref{prop:quantile}]
By definition of AGG, for any $\epsilon>0$, there is a constant $C(\epsilon)$ such that for all $x\ge C$, we have
$$
-\frac{1}{\nu}x^\nu(1+\epsilon) \le \log{\overline{F}(x)} \le -\frac{1}{\nu}x^\nu(1-\epsilon).
$$
Therefore, for all $x < x_l := \left((1+\epsilon)^{-1}\nu\log{p}\right)^{1/\nu}$, we have
\begin{equation} \label{eq:AGG-quantiles-proof-1}
    -\log{p} = -\frac{1}{\nu}x_l^\nu(1+\epsilon) \le \log{\overline{F}(x_l)} \le \log{\overline{F}(x)},
\end{equation}
and for all $x > x_u := \left((1-\epsilon)^{-1}\nu\log{p}\right)^{1/\nu}$, we have
\begin{equation} \label{eq:AGG-quantiles-proof-2}
    \log{\overline{F}(x)} \le \log{\overline{F}(x_u)} \le -\frac{1}{\nu}x_u^\nu(1-\epsilon) = -\log{p}.
\end{equation}
By definition of generalized inverse,
\begin{equation*}
    u_p := F^\leftarrow(1-1/p) = \inf\{x:\overline{F}(x)\le 1/p\} = \inf\{x:\log{\overline{F}(x)} \le -\log{p}\}.
\end{equation*}
We know from relations \eqref{eq:AGG-quantiles-proof-1} and \eqref{eq:AGG-quantiles-proof-2} that 
$$
[x_u, +\infty) \subseteq \{x:\log{\overline{F}(x)} \le -\log{p}\} \subseteq [x_l, +\infty),
$$
and so $x_l\le u_p \le x_u$, and the expression for the quantiles follow.
\end{proof}

