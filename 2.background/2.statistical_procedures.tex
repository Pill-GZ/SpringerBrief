
We review some popular procedures for signal detection and signal support recovery tasks in this section.

{\bf Signal detection.}
One of the commonly used statistics in sparse signal detection problems such as \eqref{eq:global-test-additive} and \eqref{eq:global-test-chisq} are the $L_q$ norms of the observations $x$,
\begin{equation} \label{eq:Lq-norm}
   L_q(x) = \left(\sum_{i=1}^p{|x(i)|^q}\right)^{1/q}.
\end{equation}
Typical choices of $q$ include $q=1, 2$ and $\infty$, where $L_\infty(x)$ is interpreted as the limit of $L_q(x)$ norms as $q\to\infty$, and is equivalent to $\max_{i}|x(i)|$.
Test procedures based on \eqref{eq:Lq-norm} may then be written as $T(\mathcal{R}(x)) = \mathbbm{1}_{(t,+\infty)}(L_q(x))$, where the cutoff $t$ can be chosen to control the Type I error at desired levels.

While \eqref{eq:Lq-norm} measures the deviation of the data from the origin in an omnidirectional manner, statistics that are tailored to the alternatives can be used in the hopes of power improvement if the directions of the alternatives are known.
For example, in the additive error model \eqref{eq:model-additive}, suppose we want to test for positive mean shifts, i.e., one-sided alternative
\begin{equation} \label{eq:global-test-one-sided}
    %\mathcal{H}_0: \mu(i) = 0, \;i=1,\ldots,p,
    %\quad\text{v.s.}\quad 
    \mathcal{H}_1: \mu(i)> 0, \; \text{for some }i\in\{1,\ldots,p\}.
\end{equation}
Then, one might consider monitoring the sum (or equivalently, the arithmetic average) of the observations, 
\begin{equation} \label{eq:sum-statistic}
    S(x) := \sum_{i=1}^p{x(i)},
\end{equation}
or the maximum of the observations,
\begin{equation} \label{eq:max-statistic}
    M(x) := \max_{i=1,\ldots,p}{x(i)}.
\end{equation}
Other tests based on the empirical \ac{CDF} are also available.
Assuming the same one-sided alternative, let 
\begin{equation}
    q(i) = 1 - \sup\{F_i(y)\,:\,y<x(i)\}, \quad i = 1,\ldots,p,
\end{equation}
be the p-values of the individual observations, where $F_i$ is the \ac{CDF} of the $i$-th component $x(i)$ under $\mathcal{H}_0$.
We define empirical \ac{CDF} of the p-values as
\begin{equation}
    \widehat{F}_p(t) = 
    \frac{1}{p} \sum_{i=1}^p \mathbbm{1}_{[0, t]}(q(i)).
\end{equation}
Viewed as random elements in the space of c\`adl\`ag functions with the Skorohod $J_1$ topology, the centered and scaled \ac{CDF}s converge weakly to a Brownian bridge,
\begin{equation*}
    \left\{\sqrt{p}\left(\widehat{F}_p(t) - t\right)\right\}_{t\in[0,1]} 
    % \stackrel{J_1}{\longrightarrow} 
    \implies
    \left\{\mathbb{B}(t)\right\}_{t\in[0,1]},\ \ \ \mbox{ as } p\to\infty,
\end{equation*}
under the global null $\mathcal{H}_0$ and mild continuity assumptions on the $F_i$'s \citep{skorokhod1956limit}. 
Therefore, goodness-of-fit statistics such as Kolmogorov-Smirnov distance \citep{smirnov1948table}, Cramer-von Mises-type statistics \citep{cramer1928composition, anderson1952asymptotic} that measure the departure from this limiting behavior can be used for testing $\mathcal{H}_0$ against $\mathcal{H}_1$.
Of particular interest is the higher criticism (\ac{HC}) statistic, first proposed by \cite{tukey1976lecture},
\begin{equation} \label{eq:HC-statistic}
    HC(x) = 
    \max_{0\le t\le\alpha_0}\frac{\widehat{F}_p(t) - t}{\sqrt{t(1 - t)/p}}.
\end{equation}

Each of the above statistics $L_q$, $S$, $M$, or $HC$, gives rise to a decision rule, whereby the null hypothesis is rejected if 
the statistic exceeds a suitably calibrated threshold. The choice of the threshold is typically determined based on large-sample limit 
theorems. For example, as shown in Theorem 1.1 of \cite{donoho2004higher}, under the null hypothesis
$$
 \frac{HC(x)}{\sqrt{2\log\log(p)}} \longrightarrow  1,\ \ \mbox{ in probability,}
$$ 
as $p\to\infty$.  Thus, one decision rule is to reject ${\cal H}_0$, if $HC(x)> t(p,\alpha_p)$, where 
$t(p,\alpha_p) = \sqrt{2\log\log(p)}(1+o(1))$.  As we will see, this yields an optimal signal detection procedure 
\citep[see also Theorem 1.2 in ][]{donoho2004higher}. The performance of these statistics in high-dimensional 
sparse signal detection problems will be reviewed in Section \ref{sec:asymptotics}, and analyzed in Chapter \ref{chap:phase-transitions}.

\medskip

{\bf Signal support recovery.}
In signal support recovery tasks, we shall study the performance of five procedures, all of which belong to the broad class of thresholding procedures.
\begin{definition}[Thresholding procedures]
A thresholding procedure for estimating the support 
$S:=\{i\, :\, \lambda(i)\neq0\}$ is one that takes on the form
\begin{equation} \label{eq:thresholding-procedure}
    \widehat{S} = \left\{i\,|\,x(i) \ge t(x)\right\},
\end{equation}
where the threshold $t(x)$ may depend on the data $x$.
\end{definition}

Examples of thresholding procedures include ones that aim to control FWER \eqref{eq:FWER-FWNR} --- Bonferroni's \citep{dunn1961multiple}, Sid\'ak's \citep{vsidak1967rectangular}, Holm's \citep{holm1979simple}, and Hochberg's procedure \citep{hochberg1988sharper} --- as well as procedures that target FDR \eqref{eq:FDR-FNR}, such as the Benjamini-Hochberg \cite{benjamini1995controlling} and the Barber-Cand\`es procedure \citep{barber2015controlling}.
Indeed, the class of thresholding procedures \eqref{eq:thresholding-procedure} is so general that it contains most (but not all) of the statistical procedures in the multiple testing literature.
% \cite{roquain2011type}.


% \subsection{FWER-controlling procedures}
% \label{subsec:FWER-controlling-procedures}

Under the assumption that the data $x(i)$'s under the null have a common marginal distribution $F$, we review five thresholding procedures for support recovery, starting with the well-known Bonferroni's procedure which aims at controlling family-wise error rates.
\begin{definition}[Bonferroni's procedure] \label{def:Bonf}
Bonferroni's procedure with level $\alpha$ is the thresholding procedure that uses the threshold
\begin{equation} \label{eq:Bonferroni-procedure}
    t_p = F^{\leftarrow}(1 - \alpha/p).
\end{equation}
where  $F^{\leftarrow}(u)=\inf{\left\{x:F(x)\ge u\right\}}$ is the generalized inverse function.
\end{definition}
% It is easy to see that the family-wise error rate (FWER) is controlled at level $\alpha$ by applying the union bound, regardless of the error-dependence structure (see e.g.\ Relation \eqref{eq:Bonferroni-FWER-control}, below).
The Bonferroni procedure is deterministic, i.e. non data-dependent, and only depends on the dimension of the problem and the null distribution.
A closely related procedure is Sid\'ak's procedure \citep{vsidak1967rectangular},
which is a more aggressive (and also deterministic) thresholding procedure that uses the threshold
\begin{equation} \label{eq:Sidak-procedure}
    t_p = F^{\leftarrow}((1 - \alpha)^{1/p}).
\end{equation}
% can be shown to control FWER in the case independent errors.

The third procedure, strictly more powerful than Bonferroni's, is the so-called Holm's procedure \citep{holm1979simple}.
On observing the data $x$, its coordinates can be ordered from largest to smallest
$x(i_1) \ge x(i_2)  \ge \ldots \ge x(i_p)$,
where $(i_1, \ldots, i_p)$ is a permutation of $\{1, \ldots, p\}$. 
Denote these order statistics as $x_{[1]}, x_{[2]}, \ldots, x_{[p]}$.
\begin{definition}[Holm's procedure]
Let $i^*$ be the largest index such that
$$
\overline{F}(x_{[i]}) \le \alpha / (p-i+1),\quad \text{for all }\;i\le i^*.
$$
Holm's procedure with level $\alpha$ is the thresholding procedure with threshold
\begin{equation} \label{eq:Holm-procedure}
    t_p(x) = x_{[i^*]}.
\end{equation}
\end{definition}
In contrast to the Bonferroni procedure, Holm's procedure is data-dependent.
% It can be shown that Holm's procedure also controls FWER at $\alpha$ level, regardless of dependence in the data.
A closely related, more aggressive (and also data-dependent) thresholding procedure is Hochberg's procedure \citep{hochberg1988sharper}.
%\begin{definition}[Hochberg's procedure]
%Hochberg's procedure 
It replaces the index $i^*$ in Holm's procedure with the largest index such that
$$
\overline{F}(x_{[i]}) \le \alpha / (p-i+1).
$$
Notice that both Holm's and Hochberg's procedures compare p-values to the same thresholds $\alpha / (p-i+1)$.
However, Holm's procedure only rejects the set of hypotheses whose p-values are all smaller than their respective thresholds.
On the other hand, Hochberg's procedure rejects the set of hypotheses as long as the largest of their p-values fall below its threshold, and therefore, can be more powerful than Holm's procedure. 
%where  $\overline{F}(x)=1-F(x)$ is the survival function.
%\end{definition}

It can be shown that both Bonferroni's and Holm's procedures control FWER at their nominal levels, regardless of dependence in the 
data \citep{holm1979simple}. In contrast, Sid\'ak's and Hochberg's procedures control FWER at nominal levels when data are independent \citep{vsidak1967rectangular, hochberg1988sharper}. \stilian{DISCUSS}{Explain in the Supplement?}

Last but not least, we review the \ac{BH} procedure, which aims at controlling \ac{FDR} in 
\eqref{eq:FDR-FNR}, proposed by  \cite{benjamini1995controlling}.

Recall the order statistics of our observations $x_{[1]} \ge x_{[2]}  \ge \ldots \ge x_{[p]}$.
\begin{definition}[Benjamini-Hochberg's procedure] \label{def:BH}
Let $i^*$ be the largest index such that
$$
\overline{F}(x_{[i]}) \le \alpha i/p.
$$
The Benjamini-Hochberg (BH) procedure with level $\alpha$ is the thresholding procedure with threshold
\begin{equation} \label{eq:BH-procedure}
    t_p(x) = x_{[i^*]},
\end{equation}
\end{definition}
The \ac{BH} procedure is shown to control the FDR at level $\alpha$ when the $x(i)$'s are independent \citep{benjamini1995controlling}. 
\stilian{DISCUSS}{Again, explain in the Supplement?} Variations of this procedure have been proposed to control the \ac{FDR} under certain models of dependent observations \citep{benjamini2001control}.


% We now turn to discuss the framework for anlayzing the asymptotic risks in high dimensions.

The performance of these procedures in high-dimensional sparse signal support recovery problems will be reviewed in Section \ref{sec:asymptotics}, and analyzed in Chapters \ref{chap:phase-transitions}, \ref{chap:exact-support-recovery}, and \ref{chap:GWAS}.
