
We end this chapter with several facts about univariate and multivariate Gaussian distributions that will be used in the rest of the manuscript. \\

{\bf Relative stability.} We first state the relative stability of iid standard Gaussian random variables.  Since the standard Gaussian distribution falls in the 
class of asymptotically generalized Gaussians (AGG; see Definition \ref{def:AGG}), by Example \ref{exmp:AGG}, we know that the triangular array ${\cal E} = \{\left(\epsilon_p(i)\right)_{i=1}^p, p\in\N\}$ has relatively stable (RS) maxima in the sense of \eqref{eq:RS-condition}, i.e.,
\begin{equation} \label{eq:relative-stability-Gaussian-maxima}
    \frac{1}{u_{p}} \max_{i=1,\ldots,p} \epsilon_p(i) \xrightarrow{\P} 1,\quad \text{as }\;p\to\infty,
\end{equation}
where $u_p$ is the $(1/p)$-th upper quantile as defined in \eqref{eq:AGG-quantiles}.
Similarly, since the array ${\cal E}$ has distributions symmetric around 0, it also has relatively stable minima
\begin{equation} \label{eq:relative-stability-Gaussian-minima}
    \frac{1}{u_{p}} \min_{i=1,\ldots,p} \epsilon_p(i) \xrightarrow{\P} -1,\quad \text{as }\;p\to\infty.
\end{equation}
The convergence in \eqref{eq:relative-stability-Gaussian-maxima} also holds almost surely.\\

{\bf Mill's ratio.} We give next the well-known bounds for the Mill's ratio of Gaussian tails.
Let $\Phi$ denote the CDF of the standard Gaussian distribution and $\phi$ its density.
One can show that for all $x>0$ we have
\begin{equation} \label{eq:Mills-ratio}
    \frac{x}{1+x^2}\phi(x) \le \overline{\Phi}(x) = 1-\Phi(x) \le \frac{1}{x}\phi(x),
\end{equation}
using e.g., integration by parts.  Note that this fact may be used to verify the rapid variation of $\Phi$, which entails 
the relative stability property above.\\

{\bf Stochastic monotonicity.} The third fact is the stochastic monotonicity of the Gaussian location family. 
In fact, for all location families $\{F_\delta(x)\}_\delta$ where $F_\delta(x) = F(x-\delta)$, we have,
\begin{equation} \label{eq:stochastic-monotonicity-Gaussian}
    F_{\delta_1}(t) \ge F_{\delta_2}(t), \quad \text{for all}\quad t\in\mathbb{R}\quad\text{and all}\quad \delta_1 \le \delta_2.
\end{equation}
Relation \eqref{eq:stochastic-monotonicity-Gaussian} holds, of course, when $F$ is the standard Gaussian distribution. \\

{\bf Slepian's lemma and the Sudakov-Fernique inequality.} 
The following two results will be instrumental in our characterization of uniform relative stability for 
Gaussian triangular arrays in Chapter \ref{chap:URS}.  
% Many more details and deeper insights can be found for example in the monographs 
%\cite{leadbetter2012extremes} and \cite{adler2009random}, among others. 
The first is the celebrated result due to \citet{slepian1962one}.

\begin{theorem}[Slepian's lemma]\label{thm:Slepian-lemma} Let $\epsilon = (\epsilon(i))_{i=1}^p$ and $\eta = (\eta(i))_{i=1}^p$ be 
two multivariate normally distributed random vectors with zero means $\E [\epsilon(i)] = \E[\eta(i)] = 0$. 

 If for all $i,j=1,\cdots,p$, we have
$$
\E [\epsilon(i)^2] = \E[\eta(i)^2],\ \ \ \mbox{ and }\ \ \ {\rm Cov}(\epsilon(i),\epsilon(j)) \le {\rm Cov}(\eta(i),\ \eta(j)),
$$
 then $\epsilon \stackrel{{\rm st}}{\ge} \eta$, i.e.,
$$
\P [ \epsilon(i) \le x_i,\ i=1,\cdots,p] \le \P [ \eta(i) \le x_i,\ i=1,\cdots,p].
$$
\end{theorem}


This result implies in particular that $M_\epsilon:= \max_{i=1,\cdots,p}\epsilon(i)$ dominates stochastically $M_\eta:= \max_{i=1,\cdots,p} \eta(i)$ in the
sense that
\begin{equation}\label{e:Slepian-lemma-maxima}
 \P[ M_\eta > u] \le \P [M_\epsilon > u],\ \ \mbox{ for all }u\in\mathbb R.
\end{equation}
In this case, we shall write $M_\eta \stackrel{d}{\le} M_\epsilon$.  This result shows, for example, that the maximum of iid Gaussians is 
stochastically larger the maximum of any positively correlated Gaussian vector with the same marginal distributions. 

Slepian's lemma can be obtained as a corollary from the general Normal Comparison Lemma 
\citep[see, e.g., Theorem 4.2.1 on page 81 in][]{leadbetter2012extremes}.  See also Ch.\ 2 in \cite{adler2009random}.

The following result, known as the Sudakov-Fernique inequality, is similar in spirit to Slepian's lemma but it does not assume that the Gaussian 
vectors are centered and yields a weaker conclusion -- an inequality between expectations. For a proof, many insights, and, in fact, a more 
general result, see e.g., Theorem 2.2.5 on page 61 in \cite{adler2009random}.

\begin{theorem} [Sudakov-Fernique Inequality] \label{th:Sudakov-Fernique} 
Let $\epsilon = (\epsilon(i))_{i=1}^p$ and $\eta = (\eta(i))_{i=1}^p$ be 
two multivariate normally distributed random vectors.

If for all $i,j=1,\cdots,p$, we have 
$$
\E [ \epsilon(i) ] = \E [\eta(i)] \ \ \mbox{ and }\ \ \E [ (\eta(i)-\eta(j))^2] \le \E [ (\epsilon(i)-\epsilon(j))^2],
$$
then for $M_\epsilon = \max_{i=1,\cdots,p}\epsilon(i)$ and $M_\eta = \max_{i=1,\cdots,p} \eta(i)$, we have 
$$
 \E [ M_\eta ] \le \E [M_\epsilon].
$$
\end{theorem}


